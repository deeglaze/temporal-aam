%
As noted in \autoref{sec:sort}, temporal contracts are associated with structural contracts that label function components within them.
%
For simple exposition, we will consider tuples as the main organizational tool for contracting the interactions between multiple functions.
%
Since we consider monitor constructions as a more basic notion of equality, we also see each temporal monitor construction as starting its own \emph{timeline}, which sees its own filtered view of events in the system.
%
Thus, as values flow through contract boundaries, they are considered on different timelines.
%
\dfm formalized their semantics in terms of a nondeterministic machine that defined its interactions on all event streams, and thus their machine was on a single timeline.
%
The semantics of temporal contracts that we propose uses pointer-equality of monitors for comparisons of non-primitive data.
%
Our semantics makes interaction between temporal contract monitors explicit, allowing us to verify whole programs.
%%

%%
A denotational semantics for temporal contracts is difficult to get right, since their behavior is implicitly prefix-closed.
%
Liveness-looking properties have a hidden, insidious behavior with negation: contracts that state anything about how a trace may not end would allow just such traces since extensions to such bad traces are acceptable, and prefix closure will throw the ``bad traces'' back into what is acceptable.
%
For example, $\mevent\mevent \in \prefixes(\denotetcon{\stnot{\mevent}}{\mtimeline}{\menv})$, and because of prefix closure, $\mevent \in \prefixes(\denotetcon{\stnot{\mevent}}{\mtimeline}{\menv})$!
%
Temporal contract satisfaction should not be contingent on future observations; an effective monitor should blame \emph{as soon as} a contract is not satisfied.
%
We thus have a different semantics for temporal contracts that ping-pongs between \emph{full traces} and \emph{partial traces}; a negated temporal contract will reject all full traces of the given contract, as well as any extension of such traces.
%
This semantics of negation does not satisfy double-negation elimination (DN), but we find that to be an acceptable trade-off; we do not need DN in order to implement or verify temporal contracts.
%
We claim that this semantics is what \dfm meant their system to mean, as it matches up with the expectations of their prose and the test cases in their implementation.
%
Furthermore, our implementation catches more ``bad'' behavior than their monitoring system, meaning we have not just formalized their implementation.
%%

%%
The denotational semantics in \autoref{fig:tcontract-denotation} lends itself nicely to online monitoring via a derivative parsing approach.
%
For each event $\mevent$ we come across on a timeline $\mtimeline$, we set $\mTMons(\mtimeline)$ to $\derive{\mevent}{\mTMons(\mtimeline)}$, as long as the derivative isn't empty.
%
We can show that the partial trace semantics is prefix-closed (\autoref{thm:prefix-closed}), and thus we can pinpoint the cause of an error as soon as it is sent to the monitor - a necessary condition for effective blame management in this realm.
%%

%%
We made the design choice to require a function contract monitor wrapping for named values to consider their calls or returns as events --- this choice is reflected in the denotation of events in \autoref{fig:event-denotation}.
%
The reason for this is that control should flow back to the timeline considered in order for an event to affect that timeline.
%
It is easy enough to amend the structural contracts to reflect the fact that a binding is considered a function in the temporal contract.

\begin{theorem}[Prefix closure]
  $\prefixes(\denotetcone{\mtcon}{\menv}) = \denotetcone{\mtcon}{\menv}$
\end{theorem}

\FloatBarrier
\begin{figure}
 \begin{align*}
   \denotetcone{\stOr{\mtcon \ldots}}{\menv} &= \bigcup\denotetcone{\mtcon}{\menv}\ldots
\\
    \denotetcone{\stAnd{\mtcon \ldots}}{\menv} &= \bigcap\denotetcone{\mtcon}{\menv}\ldots
\\
    \denotetcone{\stmany{\mtcon}}{\menv} &= \denotetconfulle{\stmany{\mtcon}}{\menv} \semseq \denotetcone{\mtcon}{\menv}
\\
    \denotetcone{\stnot{\mtcon}}{\menv} &= \semneg{\denotetconfulle{\mtcon}{\menv}}
\\
    \denotetcone{\stseq{\mtcon_0}{\mtcon_1}}{\menv} &= \denotetcone{\mtcon_0}{\menv} \cup \denotetconfulle{\mtcon_0}{\menv}\semseq \denotetcone{\mtcon_1}{\menv}
\\
    \denotetcone{\stbind{\mevent}{\mtcon}}{\menv} &= \setof{\epsilon} \cup \setbuild{\mevent' \mtrace}{\menv' = \matches(\mevent, \mevent', \menv), \mtrace \in \denotetcone{\mtcon}{\menv'}}
\\
    \denotetcone{\epsilon}{\menv} &= \setof{\epsilon}
\\
    \denotetcone{\mevent}{\menv} &= \setof{\epsilon} \cup \denotetconfulle{\mevent}{\menv}
\\[2pt]
    \denotetconfulle{\mtcon,\menv}{\menv} &= \denotetconfulle{\mtcon}{\menv}
\\
    \denotetconfulle{\stOr{\mtcon \ldots}}{\menv} &= \bigcup\denotetconfulle{\mtcon}{\menv}\ldots
\\
    \denotetconfulle{\stAnd{\mtcon \ldots}}{\menv} &= \bigcap\denotetconfulle{\mtcon}{\menv}\ldots
\\
    \denotetconfulle{\stnot{\mtcon}}{\menv} &= \semneg{\denotetconfulle{\mtcon}{\menv}}
\\
    \denotetconfulle{\stseq{\mtcon_0}{\mtcon_1}}{\menv} &= \denotetconfulle{\mtcon_0}{\menv} \semseq \denotetconfulle{\mtcon_1}{\menv}
\\
    \denotetconfulle{\stmany{\mtcon}}{\menv} &= \setbuild{\mtrace^i}{i \le \omega, \mtrace \in \denotetconfulle{\mtcon}{\menv}}
\\
    \denotetconfulle{\stbind{\mevent}{\mtcon}}{\menv} &= \setbuild{\mevent' \mtrace}{\menv' = \matches(\mevent, \mevent', \menv), \mtrace \in \denotetconfulle{\mtcon}{\menv'}}
\\
    \denotetconfulle{\epsilon}{\menv} &= \setof{\epsilon}
\\
    \denotetconfulle{\mevent}{\menv} &= \setbuild{\mevent'}{\menv' = \matches(\mevent, \mevent', \menv)}
\\[2pt]
    \semneg{\Pi} &= \setbuild{\mtrace}{\forall \mtrace' \in \Pi\setminus\setof{\epsilon}. \mtrace' \not\le \mtrace}
\\
    \Pi \semseq \Pi' &= \setbuild{\mtrace \cdot \mtrace'}{\mtrace \in \Pi, \mtrace' \in \Pi'}
 \end{align*}
  \caption{Denotational Semantics of Temporal Contracts}
  \label{fig:tcontract-denotation}
\end{figure}

%
We say $\mexp$ satisfies its temporal contract if its event trace filtered by the timeline ($\denote{\mexp}_\mtimeline$) is in the denotation of the temporal contract for that timeline ($\denotetcon{\mtcon})$, where $\bot$ is an empty environment of temporal bindings).
%
