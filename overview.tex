\begin{figure}
  \begin{align*}
    \mscon \in \SContract &::= \sflat{\mexp} \alt \sarr{\mtoplevelname}{\mscon}{\mscon} \alt \sconsc{\mscon}{\mscon}
\\
    \mexp \in \Expr &::= \sTMon{k}{l}{j}{\mscon}{\mtcon}{\mexp}
                    \alt \sSMon{k}{l}{j}{\mtimeline}{\mscon}{\mexp}
                    \alt \text{other forms}
\\
\mmlab,k,l,j \in \Label&\text{ an infinite set} \\
\mtimeline \in \Timeline &\text{ an infinite set}
  \end{align*}
  \caption{Syntax of structural contracts with labels}
  \label{fig:scontract-syntax}
\end{figure}

\begin{figure}
  \begin{align*}
  \motcon \in \TContract^\circ &::=
      \mevent \alt \snonevent{\mevent}
 \alt \epsilon
 \alt \stnot{\motcon}
 \alt \stseq{\motcon}{\motcon}
 \alt \stmany{\motcon{}} \\
&\alt \stOr{\isset{\motcon}}
 \alt \stAnd{\isset{\motcon}}
 \alt \stbind{\mevent}{\motcon}
\\
\mtcon \in \TContract &= \text{same rules as } \TContract^\circ \text{ for } \mtcon \text{ plus } \alt \motcon, \menv \\
\menv \in \Env &= \Var \to \wp(\Value) \\
\mevent \in \Event &::= \scevev{\mname}{\mvpat} \alt \sany \\
\mcev \in \FunctionEvent &::= {\tt call} \alt {\tt ret} \\
\mvpat \in \VPat &::= \mval \alt \sbind{\mvar} \alt \mname \alt \scons(\mvpat,\mvpat) \alt \snegpat{\mvpat} \alt \sany \alt \snone \\
\mvar \in \Var &\quad\text{an infinite set} \\
\mname \in \Name &::= \mvar \alt \mtoplevelname
  \end{align*}
  \caption{Syntax of temporal contracts}
  \label{fig:tcontract-syntax}
\end{figure}

Temporal contracts provide a declarative language for monitoring the temporal ordering of events that pass through module boundaries.
%
We present and analyze a slightly different formulation than \dfm's temporal contracts that allows for more precise specification of value-use.
%
The syntax is presented in \autoref{fig:tcontract-syntax}.
%
We notate sets of some element type with metavariable $e$ as $\isset{e}$, and lists as $\many{e}$.
%
We write the interpolation of sets and lists into many arguments as $e\ldots$, following the rules of \citet{dvanhorn:Kohlbecker1987Macrobyexample}.
%
The if-then-else syntax we use follows Dijkstra ($\mathit{guard} \to \mathit{then}, \mathit{else}$), where anything non-$\bot$ is considered true.
%%

%%
Temporal contracts ($\mtcon$) include events ($\mevent$) (for $\mevent$ction), negated events ($\snonevent{\mevent}$), event matching scoped over a following contract ($\stbind{\mevent}{\mtcon}$), concatenation ($\stseq{\mtcon}{\mtcon}$) (often represented using juxtaposition), negated contracts ($\stnot{\mtcon}$), Kleene closure of events ($\stmany{\mtcon}$), union ($\stOr{\isset{\mtcon}}$), intersection ($\stAnd{\isset{\mtcon}}$), the empty temporal contract ($\epsilon$), and an open temporal contract closed by an environment ($\motcon, \menv$).
%
We consider the fail contract $\bot$ as a macro for $\stOr{\setof{}}$; \dfm's universal contract $\sddd$ was intended to be interpreted lazily (though was not in the formal semantics), but laziness poses challenges to the derivative parsing approach we have taken, so we leave the study of laziness to future work.
%
The difference between $\snonevent{\mevent}$ and $\stnot{\mtcon}$ is that the first must be an event and force time to step forward once, whereas the second may match arbitrarily many events.
%%

%%
Events themselves are expressed as patterns denoting calls ($\scallev{\mname}{\mvpat}$) or returns ($\sretev{\mname}{\mvpat}$), with respect to a particular function $\mname$ and with its argument or result matching a pattern $\mvpat$.
%
If $\mname$ is a label ($\mtoplevelname$), we simply check that the monitor wrapping the function has the same label (attached via the structural contract).
%
However, if $\mname$ is a variable ($\mvar$), then we consult a binding environment that the monitoring system builds as we pass binding events to determine if the function is exactly equal to the value bound.
%
Our distinction between variables (bound by temporal contract) and labels (bound by structural contract) is one aspect of what separates our semantics from \dfm's.
%
Patterns can match values ($\mval$), variable bindings and references ($\sbind{\mvar}$, $\mvar$), labeled functions ($\mtoplevelname$), structured data ($\scons(\mvpat,\mvpat)$), negated patterns that do not bind ($\snegpat{\mvpat}$), anything or nothing ($\sany$, $\snone$).
%%

%%
\dfm's semantics for referring to functions is problematic; we give a slightly different account that captures the spirit of their prose describing their system, and indeed mirrors their actual implementation.
%
Consider one of their motivating examples (recalled---using our syntax and semantics---and discussed in \autoref{sec:sort}), which was to protect the comparator passed to a sort function from escaping the scope of the call.
%
What this should mean is the particular binding introduced by a call to $sort$ cannot be called after $sort$ returns, \ie, each constructed monitor around given comparators should not be called after $sort$ returns.
%
However, the flat use of \emph{labels} instead of \emph{bindings} would cause a second call to a supposedly-correct $sort$ to fail, since it internally calls the comparator of the same label, but of a different monitor construction.
%
\dfm's implementation also includes call/return matching via a different binding at the call site to eventually use in the matching return.
%
We leave call/return matching to future work, as it is not necessary to verify the motivating examples.
%%

%%
\iflong{
\subsection{Sort example} \label{sec:sort}
%
\renewcommand*{\arraystretch}{1.2}
\newcommand*{\call}[1]{\scallev{#1}{\_}}
\newcommand*{\ret}[1]{\sretev{#1}{\_}}
\begin{align*}
 &\begin{array}{ l @{\quad}l@{\ } c @{\ }l }
 SortContract =
 &sort &:   &(cmp\ :\ Pos\ \to Pos\ \to\ Bool) \\
 &     &    &(List\ Pos) \\
 &     &\to &(List\ Pos) \\
 \end{array}
 \\
 &~\begin{array}{ @{~}r@{} l @{}l }
  \text{where}\quad
  &\stnot{(&\sddd~ \call{sort}~ \stmany{\snonevent{\ret{sort}}}~ \call{sort}~ \sddd)} \\
  \cap\quad
  &\stnot{(&\sddd~ \scallev{sort}{?cmp}~ \sddd~ \\
  &&\ret{sort} \sddd~ \call{cmp}~ \sddd)}
 \end{array}
 %\caption{Sort example}
 %\label{fig:sort}
\end{align*}
%
%%
This example presents the contract for a hypothetical $sort$ function which takes two arguments: a comparator and a list (of positive numbers).
%
The notation ``$\mathit{name}\ :\ Domain\ \to\ Range$'' describes a function contract where the argument satisfies the $Domain$ contract and the result satisfies the $Range$ contract.
%
The ``$\mathit{name}\ :$'' prefix denotes the name of the function, for use in the temporal aspect of the contract.
%%
%
%%
$SortContract$'s temporal component, given by the ``where'' clauses following its structural (arrow) contract.
%
The first of these clauses states that a second call to $sort$ may not occur (hence the negation of the trace) if there is no intervening return from $sort$ ($\stmany{\snonevent{\ret{sort}}}$).
%
This is specifying a particular safety property (as evidenced by the negation of the trace): $sort$ is supposed to be \emph{non-re-entrant}.
%
The second temporal clause specifies a higher-order property; it states that, given a call to $sort$, its associated $cmp$ argument cannot be called after $sort$ returns.
%%
%
%%
Note also that the negated clauses of the temporal contract are prefixed and suffixed by ``$\sddd$''.
%TODO: Provide decent and intuitive explanation that doesn't require any forward-reference
%%
%
%%
\subsection{File example} \label{sec:file}
%
\begin{align*}
 &FileSystemContract\, =\, open\, :\, String\, \to\, FileContract \\
 &FileContract =~ Record \\
 &\begin{array}{ @{\quad~}l@{\ :} @{~}l@{\ \to\ } l }
  read & Unit & String \\
  write & String & Unit \\
  close & Unit & Unit
 \end{array} \\
 &\text{where}\quad \sddd~ \ret{close}
 %\caption{File example}
 %\label{fig:file}
\end{align*}
%
%%
This example gives the contract for a hypothetical file system, which can be used to open files by giving the $open$ function a filename (a $String$); the client is then given a file handle contracted by $FileContract$.
%
A file handle, in turn, is a record of functions which interact with the file: $read$, $write$, and $close$, all which perform the expected behaviors.
%%
%
%%
The temporal contract is what is interesting: it is not phrased in terms of a negation, but rather an affirmation.
%
Its goal is to state that a user of the file is forbidden from making use of the file handle (through the use of its component functions) after the user has $close$d the file.
%
It is phrased such that there is no ``$\sddd$'' at the end of its trace; this means that the last reference one can make to such a monitored record is returning from $close$; after that, it cannot be used.
%
%TODO: Do we need to define prefix-closure?
Note that this is not a \emph{liveness property}; this does not mean that a return from $close$ \emph{must} happen, as traces are \emph{prefix-closed}.
%
Instead, the property is a \emph{safety property}, though expressed in the affirmative.
%%
%
%%
\subsection{TCP example} \label{sec:tcp}
%
\newcommand*{\tcpc}{\mathit{TCPConnection}}
\newcommand*{\tcpcc}{\mathit{TCPConnectionContract}}
\newcommand*{\tcpsock}{\mathit{TCPSocket}}
\newcommand*{\tcpsendc}{\mathit{TCPSendContract}}
\newcommand*{\tcprecvc}{\mathit{TCPRecvContract}}
\newcommand*{\tcpdata}{\mathit{TCPData}}
\newcommand*{\tcpstyle}[1]{\texttt{#1}}
%\FloatBarrier
\begin{figure}
 \newcommand*{\send}[1]{\scallev{send}{#1}}
 \newcommand*{\rcv}[1]{\sretev{recv}{#1}}
 \newcommand*{\notclose}{\snonevent{\call{close}}}
 \newcommand*{\tcpsyn}{\tcpstyle{SYN}}
 \newcommand*{\tcpack}{\tcpstyle{ACK}}
 \newcommand*{\tcpsynack}{\tcpstyle{SYN\&ACK}}
 \newcommand*{\tcpfin}{\tcpstyle{FIN}}
 %\newcommand*{\tcpfinack}{\tcpstyle{FIN\&ACK}}
 $\tcpcc\, =~ Record$ \\
 $\begin{array}{ @{\quad~}l@{\ :} @{~}l @{\ \to\ }l }
  open & \tcpsock & \tcpsendc \\
  listen & \tcpsock & \tcprecvc
 \end{array}$ \\
 $\tcpdata =$
 $~ \tcpsyn \mid \tcpack \mid \tcpsynack \mid \tcpfin \mid \tcpstyle{Data}(\_)$
%
 $\tcpc =~ Record$ \\
 $\begin{array}{ @{\quad~}l@{\ :} @{~}l@{\ \to\ } l }
  send & \tcpdata & Unit \\
  recv & Unit & \tcpdata \\
  timeout & Duration & Unit \\
  close & Unit & Unit
 \end{array}$ \\
%
 $\tcpsendc =~ \tcpc$ \\
 $\begin{array}{ @{~}r@{\quad} l @{}l }
  \text{where}
  &&\send{\tcpsyn}~ \notclose \\
  &&\rcv{\tcpsynack}~ \notclose \\
  &&\send{\tcpack}~ \stmany{\notclose} \\
%
  &&\cup
  \begin{aligned}
   &\left(
    \begin{aligned}
     &\rcv{\tcpfin}~ \send{\tcpack} \\
     &\send{\tcpfin}~ \rcv{\tcpack}
    \end{aligned}
   \right) \\
   &\left(
    \begin{aligned}
     &\call{close}~ \send{\tcpfin} \\
     &\rcv{\tcpack}~ \rcv{\tcpfin}~ \send{\tcpack}
    \end{aligned}
   \right)
  \end{aligned} \\
  &&\ret{timeout}~ \ret{close}
 \end{array}$
%
 $\tcprecvc =~ \tcpc$ \\
 $\begin{array}{ @{~}r@{\quad} l @{}l @{}l }
  \text{where}
  &&\rcv{\tcpsyn}~ \notclose \\
  &&\send{\tcpsynack}~ \notclose \\
  &&\rcv{\tcpack}~ \stmany{\notclose} \\
%
  &&\cup
  \begin{aligned}
   &\left(
    \begin{aligned}
     &\call{close}~ \send{\tcpfin} \\
     &\rcv{\tcpack}~ \rcv{\tcpfin}~ \send{\tcpack}
    \end{aligned}
   \right) \\
   &\left(
    \begin{aligned}
     &\rcv{\tcpfin}~ \send{\tcpack} \\
     &\call{close}~ \send{\tcpfin}~ \rcv{\tcpack}
    \end{aligned}
   \right)
  \end{aligned} \\
  &&\ret{timeout}~ \ret{close}
 \end{array}$
%
 \caption{TCP example}
 \label{fig:tcp}
\end{figure}
%
% TODO: Source this from WP
\begin{figure}
 \centering
 \fontsize{4}{5} \selectfont
 \def \svgwidth{\columnwidth}
 \input{tcp-fsm.pdf_tex}
 \caption{Simplified TCP FSM for \autoref{fig:tcp}}
 \label{fig:tcp-fsm}
\end{figure}
%
%%
In \autoref{fig:tcp} we show the contract for a hypothetical TCP connection module.
%
A client of this module may $open$ a $\tcpsock$ for initiating a connection or may $listen$ to a $\tcpsock$ for passively connecting.
%
A $\tcpc$ is defined similarly to our file system example; it is a record of functions which interact with the connection: $send$, $recv$, $timeout$, and $close$.
%
Notably, $send$ and $recv$ interact with data of the form $\tcpdata$, which can be one of the special packets used in the TCP protocol or can simply be some amount of user data ($\tcpstyle{Data}(\_)$).
%
%TODO: maybe express timing out in a better manner?
The $timeout$ function is unimportant to our discussion; it is used internally by the module to set timeouts for interactions with the other end of the TCP connection; clients do not directly use it and it is included solely for the ability to reason over timeouts in the temporal contract.
%%
%
%%
A socket on the ``sending'' end of the TCP connection (having used $open$) is obliged to use the connection according to the temporal component of $\tcpsendc$.
%
Likewise, a socket on the ``receiving'' end of the TCP connection (having used $listen$) is constrained by $\tcprecvc$.
%
The contracts are very involved, but they implement a simplified version of the TCP connection lifecycle given in \autoref{fig:tcp-fsm}.
%
Note that the temporal clause in each contract is phrased in the affirmative, as in \autoref{sec:file}; however, unlike that example, this property is indeed a (restricted) kind of \emph{liveness property}.
%
It states that the client uses the socket in a manner consistent with the TCP protocol and can expect the other end to likewise adhere.
%
The key difference between this property and a traditional liveness property is that it speaks of a liveness property with respect to \emph{a particular contract-monitoring} of a $\tcpc$; prefix-closure is still present at the top level (cf. \autoref{fig:tcontract-denotation}) and thus the protocol noted in the temporal contract may not occur if no $\tcpc$ is used.
%%
}