Behavioral contracts allow languages to internalize a notion of ``type checking'' at run time, so that untyped or statically-typed-but-currently-failing-the-type-checker programs can still be run/tested with expected invariants checked on-the-fly.
%
Contracts go beyond typical type systems because they give full access to the language, allowing expression of strong invariants that a traditional type checker could not verify.
%
Contract violations during testing or even after deployment help correct both the implementation and specification of software without having to provide a proof of correctness up-front.
%%

%%
Temporal higher-order contracts provide a language for specifying and dynamically monitoring adhereance to stateful protocols in a higher-order language.
%
Just as higher-order function contracts internalize an extra-linguistic type-checking process to the dynamic semantics, we claim that temporal contracts internalize the extra-linguistic process of model-checking.
%
To demonstrate, we propose taking a language with temporal higher-order contracts which uses a simple automaton --- the pattern-matching state machine (PMSM) --- for its dynamic semantics.
%
We then abstract this into a finite model using abstracting abstract machines (AAM) and then reason over it for statically verifying the temporal constraints.
%
This approach affords us a simple and direct proof of soundness.