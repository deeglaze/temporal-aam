Behavioral contracts allow languages to internalize a notion of ``type checking'' at run time, so that untyped or statically-typed-but-currently-failing-the-type-checker programs can still be run and tested with expected invariants checked on-the-fly.
%
Contracts go beyond typical type systems because they give full access to the language, allowing expression of strong invariants that a traditional type checker could not verify.
%
Contract violations during testing or even after deployment help correct both the implementation and specification of software without having to provide a proof of correctness up-front.
%%

%%
Temporal higher-order contracts provide a language for specifying and dynamically monitoring adhereance to stateful protocols in a higher-order language.
%
In this work, we present a concrete and abstract semantics for these contracts to improve specification expressiveness and also provide a method to statically verify that specifications hold.
%
We do so by introducing a notion of derivative for regular expressions with back references, and a sound, straight-forward abstraction that leaves us with a correct model-checker.