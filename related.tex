%%
The main separating factor between this work and the model-checking literature is that temporal property verification is \emph{extra-linguistic}, meaning there is no mechanism in the object language that could monitor for the properties.
%
Furthermore, work that focuses on the linguistic mechanism does not also focus on the verification mechanism.
%%

%%
\paragraph{Runtime monitoring:}
Monitoring sequences of actions at runtime is a mature and active area of research.
%
Temporal higher-order contracts are themselves a runtime monitoring system.
%
The notion of an action is reminiscent of aspect-oriented programming's notion of a \emph{join-point}, and thus we see several systems built on AspectJ \citep{ianjohnson:aspectj} that offer a domain-specific language for running \emph{advice} when the action trace matches a specified pattern, \eg, Tracematches \citep{ianjohnson:Allan05addingtrace} and J-LO~\citep{ianjohnson:jlo}.
%
Tracematches use a language similar to temporal contracts but do not support negation; they also only provide a way to execute given \emph{advice}, and not monitor the satisfaction of a specification.
%
J-LO on the other hand is a monitoring system based on LTL propositions with binding constructs (named DLTL).
%
J-LO's goal is closer to temporal contracts, but its language is not; conversely, tracematches match the language and not the goal.
%
Both systems are also tied to Java's class structure, so they cannot natively express properties of higher-order functions or refinements on values.

\paragraph{Higher-order model-checking:}
Java and C++ both have several high-quality model-checking tools \citep{ianjohnson:bandera, ianjohnson:java-pathfinder, ianjohnson:LLBMC}, some of which are bounded model-checkers; meaning they cannot fully verify temporal properties---only present possible counter-examples.
%
Bandera~\citep{ianjohnson:bandera} is a collection of tools that uses static analysis techniques to extract a finite model from a Java program to feed to select back-end model-checkers.
%
Similar to our approach, Bandera employs flow analysis in order to produce compact models.
%
Unsimilarly, it does not synthesize checkers for runtime monitoring the expressed properties, nor does it natively support higher-order functions.
%
A complete separation of model generation and model-checking also degrades precision, since the more in-depth constraint solving typical model-checkers do can help prune the control-flow space; our approach is amenable to integrated constraint solving and is left to future work.
%%

%%
A technique that specifically targets higher-order languages, higher-order recursion schemes (HORS)~\citep{ianjohnson:Knapik:2002:HPT:646794.704852}, is rooted in simply-typed, call-by-name lambda terms, but has model-checking solutions that have been extended to call-by-value~\citep{ianjohnson:DBLP:journals/jacm/Kobayashi13} and untyped~\citep{dvanhorn:Tsukada2010Untyped} languages, through heavy type-theoretic machinery.
%
Model-checking an untyped HORS is undecidable, and such model checkers make various approximations biased to soundly model-check programs in traditional type systems rather than traditional untyped languages.
%
Our technique is lighter weight and more transparently correct since it follows from a systematic transformation of a standard semantics.
%
Additionally, the AAM approach makes extensions to more complex language features straightforward, whereas in HORS one would need to CPS, double CPS, or perform a functional encoding of a new form of data; all of which impose additional proof obligations and points-of-failure for the analysis implementer.
%
Finally, HORS do not synthesize runtime monitors or have a notion of blame, unlike our system.
%%

%%
\paragraph{Behavioral contract verification:}
There is more closely related work in this area.
%
There have been many successful efforts in the realm of first-order contract verification~\citep{ianjohnson:fahndrich:contracts:2011,ianjohnson:vcc:2009}, but the techniques employed are inherently first-order: the only values are booleans.
%
\citet{ianjohnson:Flanagan:2006:HTC:1111037.1111059}'s notion of hybrid type checking is one way to state the problem: dynamic types are essentially flat contracts, and are treated as subtypes of anything during static checking.
%
If an external theorem prover can prove that the flat contracts always hold, the dynamic checks can be safely removed.
%
\citet{dvanhorn:Xu2012Hybrid} describes a higher-order contract verification system for OCaml by inlining all contract monitors and relying on a system of simplifications further enhanced by an SMT solver to optimize away dynamic checks.
%
\citet{dvanhorn:TobinHochstadt2012Higherorder} use AAM on a module semantics with higher-order contracts and is the most related to this work.
%
However, whereas their work focuses on a concrete semantics for handling unknown values and an external theorem prover to show contract containment, our work focuses on an orthogonal issue of temporal contract monitoring.
%
Our techniques should smoothly integrate with theirs when considering partial programs, and is left to future work (\cref{sec:conclusion}).
%%
