\section{Introduction}

Software systems are large, consist of many modules, and have invariants that are either outright inexpressible or too costly to express (and prove) in the language's static type system --- if it has one.
%
When this is the case, one might hope to rely on software contracts~\cite{dvanhorn:meyer-eiffel} to give dynamic guarantees about the behavior of one's system.
%
In modern, higher-order languages, the question of ``who violated the contract?'' becomes non-trivial, and we graduate to higher-order contracts ~\citep{dvanhorn:Findler2002Contracts} to blame the correct party responsible for any violation of these invariants.
%
A proposed system of \emph{temporal higher-order contracts}~\citep{ianjohnson:dfm:icfp2011} (henceform DFM) provides a linguistic mechanism to describe temporal properties of behavioral values flowing through the program.
%
%when traditionally such properties are checked extra-linguistically by model-checking an abstracted form of the program.
%
Example temporal properties are, ``a file can only be closed if it has been opened'' and, in the higher-order setting, ``if function A is given a function B, then that B may not be called once A returns.''
%
Such invariants are important for interfaces that have set-up and tear-down protocols to follow, or even pure interfaces that have particular compositions of calls needed to construct some object.
%%

%%
There are downsides to software contracts, however.
%
Since contracts are strictly for runtime monitoring, they do not themselves ensure correctness --- they help the process of constructing correct programs.
%
In addition, contracts can introduce considerable overhead to the program.
%
These may be acceptable costs to pay at the beginning of development, but model-checking provides an additional level of confidence in correctness or pinpoints means for failure.
%
A sound model checker can justify safely removing contract checking and have a performance-and-correctness return on the initial investment.
%%

%%
Temporal contracts pose an additional challenge over higher-order function contracts to statically verify, as they monitor extended interactions with a module, and not just localized pieces.
%
We propose a framework that is composable with techniques to verify functional contracts, that has low technical overhead (\eg, no translation to a model-checker's language necessary).
%
The technique uses the Abstracting Abstract Machines (AAM) approach \citep{dvanhorn:VanHorn2010Abstracting} to check for reachability of a temporal contract blame.
%%

%%
AAM was originally targeted towards describing flow analyses, but is robust enough to apply to model-checking safety properties of higher-order programs.
%
It is well-known that control-flow analysis can be stated in terms of model-checking~\citep{ianjohnson:analysis-is-mc}, but this observation is misleading; one must use a quadratic number of model-checking queries to discover what a monovariant flow analysis can discover in one run.
%
In general, the queries are at least linear in the size of the abstract value space, which is exponential in some polyvariant analyses such as 1-CFA. % TODO: cite Van Horn, Mairson?
%
The value space tends to be sparse in practice, but an ahead-of-time query generator cannot discover it without a polyvariant analysis.
%
The realm of temporal contract verification is similar: to use a model-checker, we would have to run a data-flow analysis up front in order to collect values that might be used as bindings, enumerate all possible binding/reference insantiations of the temporal contract, and then translate those contracts into LTL queries --- much more work than just running the program abstractly and reporting reachable blame states.
%

%%
Our contributions in this paper are as follows:
\begin{itemize}
 \item{an alternative to \dfm's temporal contract semantics amenable to analysis yet faithful to their examples;}
 \item{a notion of derivative for regular expressions with back-references;}
 \item{a sound abstraction to computably verify temporal contract satisfaction.}
\end{itemize}
%%

%%
\section{Overview of temporal higher-order contracts}
%%

%\FloatBarrier
\begin{figure}
  \begin{align*}
    \mscon \in \SContract &::= \sflat{\mexp} \alt \sarr{\mtoplevelname}{\mscon}{\mscon} \alt \sconsc{\mscon}{\mscon}
\\
    \mexp \in \Expr &::= \sTMon{k}{l}{j}{\mscon}{\mtcon}{\mexp}
                    \alt \sSMon{k}{l}{j}{\mtimeline}{\mscon}{\mexp}
                    \alt \text{other forms}
\\
\mmlab,k,l,j \in \Label&\text{ an infinite set} \\
\mtimeline \in \Timeline &\text{ an infinite set}
  \end{align*}
  \caption{Syntax of structural contracts with labels}
  \label{fig:scontract-syntax}
\end{figure}

\begin{figure}
  \begin{align*}
  \motcon \in \TContract^\circ &::=
      \mevent \alt \snonevent{\mevent}
 \alt \epsilon
 \alt \stnot{\motcon}
 \alt \stseq{\motcon}{\motcon}
 \alt \stmany{\motcon{}} \\
&\alt \stOr{\isset{\motcon}}
 \alt \stAnd{\isset{\motcon}}
 \alt \stbind{\mevent}{\motcon}
\\
\mtcon \in \TContract &= \text{same rules as } \TContract^\circ \text{ for } \mtcon \text{ plus } \alt \motcon, \menv \\
\menv \in \Env &= \Var \to \wp(\Value) \\
\mevent \in \Event &::= \scevev{\mname}{\mvpat} \alt \sany \\
\mcev \in \FunctionEvent &::= {\tt call} \alt {\tt ret} \\
\mvpat \in \VPat &::= \mval \alt \sbind{\mvar} \alt \mname \alt \scons(\mvpat,\mvpat) \alt \snegpat{\mvpat} \alt \sany \alt \snone \\
\mvar \in \Var &\quad\text{an infinite set} \\
\mname \in \Name &::= \mvar \alt \mtoplevelname
  \end{align*}
  \caption{Syntax of temporal contracts}
  \label{fig:tcontract-syntax}
\end{figure}

Temporal contracts provide a declarative language for monitoring the temporal ordering of events that pass through module boundaries.
%
We present and analyze a slightly different formulation than \dfm's temporal contracts that allows for more precise specification of value-use.
%
The syntax is presented in \autoref{fig:tcontract-syntax}.
%
We notate sets of some element type with metavariable $e$ as $\isset{e}$, and lists as $\many{e}$.
%
We write the interpolation of sets and lists into many arguments as $e\ldots$, following the rules of \citet{dvanhorn:Kohlbecker1987Macrobyexample}.
%
The if-then-else syntax we use follows Dijkstra ($\mathit{guard} \to \mathit{then}, \mathit{else}$), where anything non-$\bot$ is considered true.
%%

%%
Temporal contracts ($\mtcon$) include events ($\mevent$) (for $\mevent$ction), negated events ($\snonevent{\mevent}$), event matching scoped over a following contract ($\stbind{\mevent}{\mtcon}$), concatenation ($\stseq{\mtcon}{\mtcon}$) (often represented using juxtaposition), negated contracts ($\stnot{\mtcon}$), Kleene closure of events ($\stmany{\mtcon}$), union ($\stOr{\isset{\mtcon}}$), intersection ($\stAnd{\isset{\mtcon}}$), the empty temporal contract ($\epsilon$), and an open temporal contract closed by an environment ($\motcon, \menv$).
%
We consider the fail contract $\bot$ as a macro for $\stOr{\setof{}}$; \dfm's universal contract $\sddd$ was intended to be interpreted lazily (though was not in the formal semantics), but laziness poses challenges to the derivative parsing approach we have taken, so we leave the study of laziness to future work.
%
The difference between $\snonevent{\mevent}$ and $\stnot{\mtcon}$ is that the first must be an event and force time to step forward once, whereas the second may match arbitrarily many events.
%%

%%
Events themselves are expressed as patterns denoting calls ($\scallev{\mname}{\mvpat}$) or returns ($\sretev{\mname}{\mvpat}$), with respect to a particular function $\mname$ and with its argument or result matching a pattern $\mvpat$.
%
If $\mname$ is a label ($\mtoplevelname$), we simply check that the monitor wrapping the function has the same label (attached via the structural contract).
%
However, if $\mname$ is a variable ($\mvar$), then we consult a binding environment that the monitoring system builds as we pass binding events to determine if the function is exactly equal to the value bound.
%
Our distinction between variables (bound by temporal contract) and labels (bound by structural contract) is one aspect of what separates our semantics from \dfm's.
%
Patterns can match values ($\mval$), variable bindings and references ($\sbind{\mvar}$, $\mvar$), labeled functions ($\mtoplevelname$), structured data ($\scons(\mvpat,\mvpat)$), negated patterns that do not bind ($\snegpat{\mvpat}$), anything or nothing ($\sany$, $\snone$).
%%

%%
\dfm's semantics for referring to functions is problematic; we give a slightly different account that captures the spirit of their prose describing their system, and indeed mirrors their actual implementation.
%
Consider one of their motivating examples (recalled---using our syntax and semantics---and discussed in \autoref{sec:sort}), which was to protect the comparator passed to a sort function from escaping the scope of the call.
%
What this should mean is the particular binding introduced by a call to $sort$ cannot be called after $sort$ returns, \ie, each constructed monitor around given comparators should not be called after $sort$ returns.
%
However, the flat use of \emph{labels} instead of \emph{bindings} would cause a second call to a supposedly-correct $sort$ to fail, since it internally calls the comparator of the same label, but of a different monitor construction.
%
\dfm's implementation also includes call/return matching via a different binding at the call site to eventually use in the matching return.
%
We leave call/return matching to future work, as it is not necessary to verify the motivating examples.
%%

%%
\iflong{
\subsection{Sort example} \label{sec:sort}
%
\renewcommand*{\arraystretch}{1.2}
\newcommand*{\call}[1]{\scallev{#1}{\_}}
\newcommand*{\ret}[1]{\sretev{#1}{\_}}
\begin{align*}
 &\begin{array}{ l @{\quad}l@{\ } c @{\ }l }
 SortContract =
 &sort &:   &(cmp\ :\ Pos\ \to Pos\ \to\ Bool) \\
 &     &    &(List\ Pos) \\
 &     &\to &(List\ Pos) \\
 \end{array}
 \\
 &~\begin{array}{ @{~}r@{} l @{}l }
  \text{where}\quad
  &\stnot{(&\sddd~ \call{sort}~ \stmany{\snonevent{\ret{sort}}}~ \call{sort}~ \sddd)} \\
  \cap\quad
  &\stnot{(&\sddd~ \scallev{sort}{?cmp}~ \sddd~ \\
  &&\ret{sort} \sddd~ \call{cmp}~ \sddd)}
 \end{array}
 %\caption{Sort example}
 %\label{fig:sort}
\end{align*}
%
%%
This example presents the contract for a hypothetical $sort$ function which takes two arguments: a comparator and a list (of positive numbers).
%
The notation ``$\mathit{name}\ :\ Domain\ \to\ Range$'' describes a function contract where the argument satisfies the $Domain$ contract and the result satisfies the $Range$ contract.
%
The ``$\mathit{name}\ :$'' prefix denotes the name of the function, for use in the temporal aspect of the contract.
%%
%
%%
$SortContract$'s temporal component, given by the ``where'' clauses following its structural (arrow) contract.
%
The first of these clauses states that a second call to $sort$ may not occur (hence the negation of the trace) if there is no intervening return from $sort$ ($\stmany{\snonevent{\ret{sort}}}$).
%
This is specifying a particular safety property (as evidenced by the negation of the trace): $sort$ is supposed to be \emph{non-re-entrant}.
%
The second temporal clause specifies a higher-order property; it states that, given a call to $sort$, its associated $cmp$ argument cannot be called after $sort$ returns.
%%
%
%%
Note also that the negated clauses of the temporal contract are prefixed and suffixed by ``$\sddd$''.
%TODO: Provide decent and intuitive explanation that doesn't require any forward-reference
%%
%
%%
\subsection{File example} \label{sec:file}
%
\begin{align*}
 &FileSystemContract\, =\, open\, :\, String\, \to\, FileContract \\
 &FileContract =~ Record \\
 &\begin{array}{ @{\quad~}l@{\ :} @{~}l@{\ \to\ } l }
  read & Unit & String \\
  write & String & Unit \\
  close & Unit & Unit
 \end{array} \\
 &\text{where}\quad \sddd~ \ret{close}
 %\caption{File example}
 %\label{fig:file}
\end{align*}
%
%%
This example gives the contract for a hypothetical file system, which can be used to open files by giving the $open$ function a filename (a $String$); the client is then given a file handle contracted by $FileContract$.
%
A file handle, in turn, is a record of functions which interact with the file: $read$, $write$, and $close$, all which perform the expected behaviors.
%%
%
%%
The temporal contract is what is interesting: it is not phrased in terms of a negation, but rather an affirmation.
%
Its goal is to state that a user of the file is forbidden from making use of the file handle (through the use of its component functions) after the user has $close$d the file.
%
It is phrased such that there is no ``$\sddd$'' at the end of its trace; this means that the last reference one can make to such a monitored record is returning from $close$; after that, it cannot be used.
%
%TODO: Do we need to define prefix-closure?
Note that this is not a \emph{liveness property}; this does not mean that a return from $close$ \emph{must} happen, as traces are \emph{prefix-closed}.
%
Instead, the property is a \emph{safety property}, though expressed in the affirmative.
%%
%
%%
\subsection{TCP example} \label{sec:tcp}
%
\newcommand*{\tcpc}{\mathit{TCPConnection}}
\newcommand*{\tcpcc}{\mathit{TCPConnectionContract}}
\newcommand*{\tcpsock}{\mathit{TCPSocket}}
\newcommand*{\tcpsendc}{\mathit{TCPSendContract}}
\newcommand*{\tcprecvc}{\mathit{TCPRecvContract}}
\newcommand*{\tcpdata}{\mathit{TCPData}}
\newcommand*{\tcpstyle}[1]{\texttt{#1}}
%\FloatBarrier
\begin{figure}
 \newcommand*{\send}[1]{\scallev{send}{#1}}
 \newcommand*{\rcv}[1]{\sretev{recv}{#1}}
 \newcommand*{\notclose}{\snonevent{\call{close}}}
 \newcommand*{\tcpsyn}{\tcpstyle{SYN}}
 \newcommand*{\tcpack}{\tcpstyle{ACK}}
 \newcommand*{\tcpsynack}{\tcpstyle{SYN\&ACK}}
 \newcommand*{\tcpfin}{\tcpstyle{FIN}}
 %\newcommand*{\tcpfinack}{\tcpstyle{FIN\&ACK}}
 $\tcpcc\, =~ Record$ \\
 $\begin{array}{ @{\quad~}l@{\ :} @{~}l @{\ \to\ }l }
  open & \tcpsock & \tcpsendc \\
  listen & \tcpsock & \tcprecvc
 \end{array}$ \\
 $\tcpdata =$
 $~ \tcpsyn \mid \tcpack \mid \tcpsynack \mid \tcpfin \mid \tcpstyle{Data}(\_)$
%
 $\tcpc =~ Record$ \\
 $\begin{array}{ @{\quad~}l@{\ :} @{~}l@{\ \to\ } l }
  send & \tcpdata & Unit \\
  recv & Unit & \tcpdata \\
  timeout & Duration & Unit \\
  close & Unit & Unit
 \end{array}$ \\
%
 $\tcpsendc =~ \tcpc$ \\
 $\begin{array}{ @{~}r@{\quad} l @{}l }
  \text{where}
  &&\send{\tcpsyn}~ \notclose \\
  &&\rcv{\tcpsynack}~ \notclose \\
  &&\send{\tcpack}~ \stmany{\notclose} \\
%
  &&\cup
  \begin{aligned}
   &\left(
    \begin{aligned}
     &\rcv{\tcpfin}~ \send{\tcpack} \\
     &\send{\tcpfin}~ \rcv{\tcpack}
    \end{aligned}
   \right) \\
   &\left(
    \begin{aligned}
     &\call{close}~ \send{\tcpfin} \\
     &\rcv{\tcpack}~ \rcv{\tcpfin}~ \send{\tcpack}
    \end{aligned}
   \right)
  \end{aligned} \\
  &&\ret{timeout}~ \ret{close}
 \end{array}$
%
 $\tcprecvc =~ \tcpc$ \\
 $\begin{array}{ @{~}r@{\quad} l @{}l @{}l }
  \text{where}
  &&\rcv{\tcpsyn}~ \notclose \\
  &&\send{\tcpsynack}~ \notclose \\
  &&\rcv{\tcpack}~ \stmany{\notclose} \\
%
  &&\cup
  \begin{aligned}
   &\left(
    \begin{aligned}
     &\call{close}~ \send{\tcpfin} \\
     &\rcv{\tcpack}~ \rcv{\tcpfin}~ \send{\tcpack}
    \end{aligned}
   \right) \\
   &\left(
    \begin{aligned}
     &\rcv{\tcpfin}~ \send{\tcpack} \\
     &\call{close}~ \send{\tcpfin}~ \rcv{\tcpack}
    \end{aligned}
   \right)
  \end{aligned} \\
  &&\ret{timeout}~ \ret{close}
 \end{array}$
%
 \caption{TCP example}
 \label{fig:tcp}
\end{figure}
%
% TODO: Source this from WP
\begin{figure}
 \centering
 \fontsize{4}{5} \selectfont
 \def \svgwidth{\columnwidth}
 \input{tcp-fsm.pdf_tex}
 \caption{Simplified TCP FSM for \autoref{fig:tcp}}
 \label{fig:tcp-fsm}
\end{figure}
%
%%
In \autoref{fig:tcp} we show the contract for a hypothetical TCP connection module.
%
A client of this module may $open$ a $\tcpsock$ for initiating a connection or may $listen$ to a $\tcpsock$ for passively connecting.
%
A $\tcpc$ is defined similarly to our file system example; it is a record of functions which interact with the connection: $send$, $recv$, $timeout$, and $close$.
%
Notably, $send$ and $recv$ interact with data of the form $\tcpdata$, which can be one of the special packets used in the TCP protocol or can simply be some amount of user data ($\tcpstyle{Data}(\_)$).
%
%TODO: maybe express timing out in a better manner?
The $timeout$ function is unimportant to our discussion; it is used internally by the module to set timeouts for interactions with the other end of the TCP connection; clients do not directly use it and it is included solely for the ability to reason over timeouts in the temporal contract.
%%
%
%%
A socket on the ``sending'' end of the TCP connection (having used $open$) is obliged to use the connection according to the temporal component of $\tcpsendc$.
%
Likewise, a socket on the ``receiving'' end of the TCP connection (having used $listen$) is constrained by $\tcprecvc$.
%
The contracts are very involved, but they implement a simplified version of the TCP connection lifecycle given in \autoref{fig:tcp-fsm}.
%
Note that the temporal clause in each contract is phrased in the affirmative, as in \autoref{sec:file}; however, unlike that example, this property is indeed a (restricted) kind of \emph{liveness property}.
%
It states that the client uses the socket in a manner consistent with the TCP protocol and can expect the other end to likewise adhere.
%
The key difference between this property and a traditional liveness property is that it speaks of a liveness property with respect to \emph{a particular contract-monitoring} of a $\tcpc$; prefix-closure is still present at the top level (cf. \autoref{fig:tcontract-denotation}) and thus the protocol noted in the temporal contract may not occur if no $\tcpc$ is used.
%%
}
%%
\section{Semantics of Temporal Contracts} \label{sec:temporal-semantics}

%% FIXME
We present and analyze a slightly different formulation than \dfm's temporal contracts that allows for more precise specification of value-use.
%
But first, we must discuss why we do not import \dfm's semantics directly.
%
\subsection{\dfm's semantics}
\newcommand{\denotedfm}[2]{\denote{#1}_{#2}}
\begin{figure}
\setlength{\abovedisplayskip}{0pt}
\setlength{\belowdisplayskip}{4pt}
\setlength{\abovedisplayshortskip}{0pt}
\setlength{\belowdisplayshortskip}{8pt}
  \begin{align*}
    \denotedfm{\bullet}{\bullet} &: \TContract^\circ \times \MEnv \to \wp(\mathit{Trace})
    \\
    \mmenv \in \MEnv &::= \epsilon \alt \mmenv, \mvar : \msendrec \mscon
    \\
    \msendrec &\in \setof{{\tt send}, {\tt recv}} \text{, and }\sim \text{ a notion of ``matches.''}
  \end{align*}
  \begin{align*}
    \denotedfm{\scevdfm{\mtoplevelname}{\mvpat}}{\mmenv} &=
      \setbuild{\msendrec.\scevdfm{y}{d}}{\mmenv(y) \equiv \msendrec'\sarr{\mtoplevelname}{\_}{\_}, \mvpat \sim d}
    \\
    \denotedfm{\snonevent{\mevent}}{\mmenv} &= \Action \setminus \denotedfm{\mevent}{\mmenv}
    \\
    \denotedfm{\stseq{\motcon_0}{\motcon_1}}{\mmenv} &= \denotedfm{\motcon_0}{\mmenv} \mathrel{\cdot} \denotedfm{\motcon_1}{\mmenv}
    \\
    \denotedfm{\stmany{\motcon{}}}{\mmenv} &= \stmany{\denotedfm{\motcon}{\mmenv}}
    \\
    \denotedfm{\stnot{\motcon}}{\mmenv} &= \mathit{Trace} \setminus \denotedfm{\motcon}{\mmenv}
    \\
    \denotedfm{\stOr{\isset{\motcon}}}{\mmenv} &= \bigcup\denotedfm{\motcon}{\mmenv}\ldots
    \\
    \denotedfm{\sddd}{\mmenv} &= \mathit{Trace}
    \\
    \denotedfm{\stbind{\scevdfm{\mtoplevelname}{? \mvar}}{\motcon}}{\mmenv} &= \setbuild{\msendrec.\scevdfm{y}{c}\mtrace}{\mmenv(y) \equiv \msendrec'\sarr{\mtoplevelname}{\_}{\_}, \mtrace \in \denotedfm{\motcon[\mvar := c]}{\mmenv}}
  \end{align*}
  \caption{\dfm's semantics of temporal contracts}
  \label{fig:dfm-semantics}
\end{figure}
%
The semantics presented in \autoref{fig:dfm-semantics} is a recollection of the denotational semantics that \dfm{} gives to temporal contracts.
%
They use a module semantics based on an $\mathit{EF}$ machine that tracks the bindings shared across module boundaries, $\mmenv$ (originally $E$ but renamed to distinguish from evaluation contexts), and a stack of module boundaries to return to, $F$.
%
Regardless of how this machine works, the denotation of a temporal contract attached to a structural contract, $\denote{\mscon\ {\tt where}\ \motcon}$, is generated by traces of $\mathit{EF}$ that are driven by sent or received calls and return actions (roughly):
\begin{align*}
 \left\{
   \begin{array}{l}
    {\tt send.ret}(\mathit{start},h)\mtrace \in \prefixes(\denotedfm{\motcon}{\mmenv}) :
 \\ \qquad\langle \mmenv_0, \mathit{start}\rangle \Rightarrow^\mtrace \langle \mmenv, F\rangle \wedge \mmenv_0 = \epsilon, h : {\tt send}S
\end{array}\right\}
\end{align*}

The use of $\prefixes$ in this definition is problematic, and negation is the culprit.
%
Contracts that state anything about how a trace may not end would allow just such traces since \emph{extensions} to such ``bad traces'' are acceptable, and prefix closure will throw the ``bad traces'' back into what is acceptable.
%
%Additionally, if one writes a contract more carefully to reject extensions of bad traces, there isn't an obvious operational interpretation that allows early failure.
%
For example, the denotation of temporal contracts from \dfm{} allows $\mevent\mevent \in \prefixes(\denote{\stnot{\mevent}})$, and because of prefix closure, $\mevent \in \prefixes(\denote{\stnot{\mevent}})$.
%
Temporal contract failure should not be contingent on future observations; an effective monitor should blame \emph{as soon as} a contract is violated.
%%

%%
\dfm's semantics for referring to functions is additionally problematic; we give a slightly different account that captures the spirit of their prose describing their system, and more closely reflects their implementation.
%
The temporal component of the example discussed in \autoref{sec:sort} was originally the following:
\begin{align*}
 \stnot{(\sddd~ \sortid(\_)~ \stmany{\snonevent{\retof{\sortid}(\_)}}~ \sortid(\_))} 
 \ \cap\  \stnot{\sddd~ \retof{\sortid}(\_) \sddd~ \mathit{cmp}(\_)}
\end{align*}

In contrast to our restatement, the flat use of \emph{labels} instead of \emph{bindings} would cause a second call to a supposedly-correct $\sortid$ to fail, since it internally calls the comparator of the same label, but of a different monitor construction.
%
Their implementation works around this by additionally adding a monitor-wrapping action, that generates a new label to pair with the function label to uniquely identify it.
%%

\subsection{Our semantics}
%
As noted in \autoref{sec:sort}, temporal contracts are associated with structural contracts that label function components within them.
%
For simple exposition, we will consider tuples as the main organizational tool for contracting the interactions between multiple functions.
%
Since we consider monitor constructions as a more basic notion of equality, we also see each temporal monitor construction as starting its own \emph{timeline}, which sees its own filtered view of actions in the system.
%
Thus, as values flow through contract boundaries, they are considered on different timelines.
%
\dfm{} formalized their semantics in terms of a nondeterministic machine that defined its interactions on all action streams, and thus their machine was on a single timeline.
%
The semantics of temporal contracts that we propose uses a weak equality for comparisons of non-primitive data, which we evaluate as structural equality up to closures, where we use pointer-equality.
%
Our semantics makes interaction between temporal contract monitors explicit, allowing us to verify whole programs.
%%

%
To combat the problems introduced by \dfm's original use of $\prefixes$ on top of a semantics of full traces, we give a different semantics for temporal contracts that alternates between \emph{full traces} and \emph{partial traces}.
%
Additionally, a negated temporal contract will reject all non-empty full traces of the given contract, as well as any extension of such traces.
%
This semantics of negation does not satisfy double-negation elimination (DNE), but we find that to be an acceptable trade-off; we do not need DNE in order to implement or verify temporal contracts.
%
We claim that this semantics is what \dfm{} intended their system to mean, as it matches up with the expectations of their prose, the test cases in their implementation\footnote{The functional test cases, in particular, since our model does not handle Racket's object system.}, and additionally raises blame on programs that \dfm{} were surprised their implementation accepted.
%
Since our semantics catches more ``bad'' behavior than their monitoring system, we have not simply formalized their implementation.
%%

%%
The denotational semantics in \autoref{fig:tcontract-denotation} lends itself nicely to online monitoring via a derivative parsing approach.
%
For each action $\mevent$ we come across on a timeline $\mtimeline$, we set $\mTMons(\mtimeline)$ to $\derive{\mevent}{\mTMons(\mtimeline)}$, as long as the derivative isn't failing.
%
We can show that the partial trace semantics is prefix-closed (\autoref{thm:prefix-closed}), and thus we can pinpoint the cause of an error as soon as it is sent to the monitor --- a necessary condition for effective blame management in this realm.
%%

%%
We made the design choice that call actions for bound variables ($\scallev{\mvar}{\mvpat}$) will only be sent to the temporal monitor if the value bound to $\mvar$ is itself contracted on the monitor's timeline.
%
The reason for this is that control should flow back to the timeline considered in order for an action to affect that timeline.
%
It is easy enough to amend the structural contracts to reflect the fact that a binding is considered a function in the temporal contract.

\begin{theorem}[Prefix closure]\label{thm:prefix-closed}
  $\prefixes(\denotetcone{\motcon}{\menv}) = \denotetcone{\motcon}{\menv}$
\end{theorem}

%\FloatBarrier
\newcommand*{\tconsemfigs}[4]{
 \iftwocolumn{\begin{figure}#1 #2\end{figure}
              \begin{figure}#3 #4\end{figure}}
             {\begin{figure}
              \begin{minipage}[b]{.55\linewidth}#1 #2\hrule height 0pt\end{minipage}
              \begin{minipage}[b]{.40\linewidth}#3 #4\hrule height 0pt\end{minipage}
             \end{figure}}}
\tconsemfigs{
    \begin{align*}
      \denotetconbothe{\stOr{\isset{\motcon}}}{\menv} &=
      \bigcup\denotetconbothe{\motcon}{\menv}\ldots
      \\
      \denotetconbothe{\stAnd{\isset{\motcon}}}{\menv} &=
      \bigcap\denotetconbothe{\motcon}{\menv}\ldots
      \\
      \denotetconbothe{\epsilon}{\menv} &= \setof{\epsilon}
      \\
      \denotetconbothe{\stnot{\motcon}}{\menv} &=
      \semneg{\denotetconfulle{\motcon}{\menv}}
      \\[2pt]
      \denotetcone{\stmany{\motcon{}}}{\menv} &=
      \denotetconfulle{\stmany{\motcon{}}}{\menv} \mathrel{\cdot}
      \denotetcone{\motcon}{\menv}
      \\
      \denotetcone{\stseq{\motcon_0}{\motcon_1}}{\menv} &=
      \denotetcone{\motcon_0}{\menv} \cup
      \denotetconfulle{\motcon_0}{\menv}\mathrel{\cdot}
      \denotetcone{\motcon_1}{\menv}
      \\
     \denotetcone{\stbind{\mevent}{\motcon}}{\menv} &=
      \setof{\epsilon}\cup
        \{ \mdata \mtrace : \mdata,\menv' \in \denoteevent{\mevent}{\menv},
        \\&\phantom{= \setof{\epsilon}\cup \{ \mdata\mtrace : } \mtrace \in \denotetcone{\motcon}{\menv'} \}
     \\
      \denotetcone{\mevent}{\menv} &= \setof{\epsilon}\cup\denotetconfulle{\mevent}{\menv}
      \\[2pt]
      \denotetconfulle{\stseq{\motcon_0}{\motcon_1}}{\menv} &=
      \denotetconfulle{\motcon_0}{\menv} \mathrel{\cdot}
      \denotetconfulle{\motcon_1}{\menv}
      \\
      \denotetconfulle{\stmany{\motcon{}}}{\menv} &= \denotetconfulle{\motcon}{\menv}^*
      \\
     \denotetconfulle{\stbind{\mevent}{\motcon}}{\menv} &=
     \setbuild{\mdata \mtrace}{\mdata,\menv' \in \denoteevent{\mevent}{\menv}, \mtrace \in \denotetconfulle{\motcon}{\menv'}}
     \\
      \denotetconfulle{\mevent}{\menv} &= \setbuild{\mdata}{\mdata,\menv' \in \denoteevent{\mevent}{\menv}}
      \\[2pt]
      \denoteevent{\mevent}{\menv} &= \setbuild{\mdata,\menv'}{\menv'_\must = \matches(\mevent, \mdata, \menv)}
      \\
      \semneg{\Pi} &= \setbuild{\mtrace}{\forall \mtrace' \in
        \Pi\setminus\setof{\epsilon}. \mtrace' \nleq \mtrace}
    \end{align*}}{\caption{Denotational Semantics of Temporal Contracts ($B$ means both $P$ and $F$)}\label{fig:tcontract-denotation}}
  {\begin{align*}
      \derivee{\mdata}{\epsilon}{\menv} &= \bot
      \\
      \derivee{\mdata}{\mevent}{\menv} &= \left\{
        \begin{array}{ll}
          \epsilon & \text{if } \menv'_\must = \matches(\mevent, \mdata, \menv) \\
          \bot & \text{otherwise}
        \end{array}\right.
      \\
      \derivee{\mdata}{\stbind{\mevent}{\motcon}}{\menv} &=
      \left\{\begin{array}{ll}
          \motcon,\menv' & \text{if } \menv'_\must = \matches(\mevent, \mdata, \menv) \\
          \bot & \text{otherwise}
        \end{array}\right.
      \\
      \derivee{\mdata}{\stseq{\motcon_0}{\motcon_1}}{\menv} &=
      \stOr{\setof{\derivee{\mdata}{\motcon_0}{\menv},\
          \stseq{\nullable(\motcon_0)}{\derivee{\mdata}{\motcon_1}{\menv}}}}
      \\
      \derivee{\mdata}{\stOr{\isset{\motcon}}}{\menv} &=
      \stOr{\derivee{\mdata}{\motcon}{\menv}\ldots}
      \\
      \derivee{\mdata}{\stAnd{\isset{\motcon}}}{\menv} &=
      \stAnd{\derivee{\mdata}{\motcon}{\menv}\ldots}
      \\
      \derivee{\mdata}{\stmany{\motcon{}}}{\menv} &=
      \stseq{\derivee{\mdata}{\motcon}{\menv}}{\stmany{\motcon{}}}
      \\
      \derivee{\mdata}{\stnot{\motcon}}{\menv} &=
      \nullable(\derivee{\mdata}{\motcon}{\menv}) \to \bot,
      \stnot{\derivee{\mdata}{\motcon}{\menv}}
      \\[2pt]
      \nullable(\epsilon) &= \nullable(\stmany{\motcon{}}) =
      \nullable(\stnot{\motcon}) = \epsilon
      \\
      \nullable(\stbind{\mevent}{\motcon}) &= \nullable(\mevent) = \bot
      \\
      \nullable(\stOr{\isset{\motcon}}) &=
      \bigvee{\nullable(\motcon)\ldots}
      \\
      \nullable(\stAnd{\isset{\motcon}}) &=
      \bigwedge{\nullable(\motcon)\ldots}
      \\
      \nullable(\stseq{\motcon_0}{\motcon_1}) &=
      \nullable(\motcon_0)\wedge \nullable(\motcon_1)
      \\
      \nullable(\motcon,\menv) &= \nullable(\motcon)
    \end{align*}}{\caption{Derivatives of Temporal Contracts}\label{fig:tcon-deriv}}

The semantics and derivatives here are simplified to the concrete case, for space.
%
The extension to support the approximate matching semantics uses the appropriate logical connectives across $\cup$, $\cap$ and $\neg$ to compute the overall valuation of the temporal contract following a derivation.
%
Multiple possible matches (given from $\matches$) lead to multiple possible derivations, and the temporal contract connectives are lifted over these sets of derivatives.
%
Thus, derivatation produces a set of possible derivations and a valuation indicating how sure we are to \emph{not} blame.
%%

\newcommand*{\matchsemfigs}[4]{
 \iftwocolumn{\begin{figure} #1 #2 \end{figure}
              \begin{figure} #3 #4 \end{figure}}
             {\begin{figure}
              \begin{minipage}[b]{.50\linewidth}#1 #2\hrule height 0pt\end{minipage}
              \begin{minipage}[b]{.45\linewidth}#3 #4\hrule height 0pt\end{minipage}
             \end{figure}}} 
\matchsemfigs{
\setlength{\abovedisplayskip}{0pt}
\setlength{\belowdisplayskip}{4pt}
\setlength{\abovedisplayshortskip}{0pt}
\setlength{\belowdisplayshortskip}{8pt}
  \begin{align*}
    \matches &: \Pattern \times \Qualified \times \Env \to \wp(\MatchResult) \\
    \Delta &: \wp(\MatchResult) \to \MatchResult \\
    \delta &: \wp(\Valuation) \to \Valuation
  \end{align*}
    \begin{align*}
      \mvaluation \in \Valuation &::= \may \alt \must \alt \bot \\
      \mpat \in \Pattern &::= \VPat \text{ rules plus } \alt \mconstructor(\many{\mpat}) \\
      S \subset \MatchResult &::= \menvs_\mvaluation \\
      \menvs_\bot &= \emptyset_\bot = \emptyset_\mvaluation \\
      \mdata \in \Qualified &= \isset{\mdata} \alt \mval \alt \mconstructor(\many{\mdata}) \\
      \mconstructor \in \Constructors &= \setof{{\tt call}, {\tt ret}, {\tt cons}}
      \\[2pt]
      \must \wedge \must &= \must \\
      \bot \wedge \_ &= \_ \wedge \bot = \bot \\
      \may \wedge \_ &= \_ \wedge \may = \may
      \\[2pt]
      \neg \must &= \bot \\
      \neg \may &= \may \\
      \neg \bot &= \must
    \\[2pt]
    \menvs \bowtie \menvs' &= \setbuild{\combinef{\menv}{\menv'}}{\menv \in \menvs, \menv' \in \menvs'} \\
    \combinef{\menv}{\menv'} &= \lambda x. x \in \dom(\menv') \to \menv'(x), \menv(x)
    \end{align*}}
  {\caption{Spaces and functions for matching}\label{fig:matchspace}}
  {\begin{align*}
%    \matches &: \Pattern \times \Qualified \times \Env \to \MatchResult \\
    \matches(\sany, \_, \menv) &= \menv_\must \\
    \matches(\snone, \_, \menv) &= \emptyset_\bot \\
    \matches(\mtoplevelname, \snlam{\mtoplevelname}{\mvar}{\mexp}, \menv) &= \menv_\must \\
    \matches(\snegpat{\mvpat}, \mdata, \menv) &= \menv_{\neg \mvaluation} \\
    \text{where }& \matches(\mvpat, \mdata, \menv) = \menv'_\mvaluation \\
    \matches(\sbind{\mvar}, \mdata, \menv) &= \menv[\mvar \mapsto \mdata]_\must \\
    \matches(\mvar, \mdata, \menv) &= \menv_{\menv(\mvar) \simeq \mdata} \\
    \matches(\mconstructor(\many{\mpat}), \mconstructor(\many{\mdata}), \menv) &= (\Bowtie \menvs \ldots)_{\bigwedge \mvaluation \ldots} \\
    \text{where }& \menvs_\mvaluation \ldots = \matches(\mvpat, \mdata, \menv) \ldots \\
    \matches(\mpat, \isset{\mdata}, \menv) &= \Delta\setbuild{\matches(\mpat, \mdata, \menv)}{\mdata \in \isset{\mdata}} \\
    \matches(\mdata, \mdata', \menv) &= \menv_{\mdata \simeq \mdata'} \\
    \matches(\mpat, \mdata, \menv) &= \emptyset_\bot \quad
    \text{otherwise}
    \\[2pt]
    \Delta S &= (\bigcup\limits_{\menvs_{\_} \in S}\menvs)_{\delta \setbuild{\mvaluation}{{\_}_\mvaluation \in S}} \\
    \delta \emptyset &= \bot \\
    \delta \setof{\mvaluation} &= \mvaluation \\
    \delta \isset{t} &= \may \quad \text{otherwise}
  \end{align*}}{\caption{Semantics of matching}\label{fig:matchsem}}

We define our semantics of matching in \autoref{fig:matchsem} in anticipation of abstraction, where we do not always know when two values are equal.
%
When matching against a set of possible values, we might have a $\must$ match and a $\bot$ match, in which case the entire match should be considered $\may$ matching (this is the significance of the $\Delta$ metafunction).
%
The $\triangleleft$ operator extends the left environment with the bindings of the right, though the order doesn't matter considering that binding patterns may not bind the same variable twice.
%
Matching against sets of values makes it possible that we have several possible matches.
%
Thus $\matches$ returns a set of environments possible from matching a given pattern against some data.
%
At the leaves, when considering values of the language equal, $\matches$ appeals to a weak equality function, $\simeq$, where $\mval \simeq \mval' = \must$ implies $\mval = \mval'$ in the concrete semantics, and $\mval \simeq \mval' = \bot$ implies $\mval \neq \mval'$ in the concrete semantics.
%%

%%
We say $\mexp$ satisfies a temporal contract $\mtcon$ if its action trace filtered by the timeline to which the contract is attached is in the denotation of the temporal contract ($\denotetcon{\mtcon}$).
%
Since monitors are generated during reduction, the proof is mostly technical that our monitoring system ensures an expression either satisfies its contract or blames, and hinges mainly on the correctness of derivatives:
%

\begin{theorem}[Full]\label{thm:full}
 $\denotetconfull{\derivee{\mdata}{\motcon}{\menv}} = \setbuild{\mtrace}{\mdata\mtrace \in \denotetconfulle{\motcon}{\menv}}$
\end{theorem}

\begin{theorem}[Partial]\label{thm:partial}
 $\denotetcon{\derivee{\mdata}{\motcon}{\menv}} = \setbuild{\mtrace}{\mdata\mtrace \in \denotetcone{\motcon}{\menv}}$
\end{theorem}

\begin{theorem}[Top level full]\label{thm:top-full}
 $\denotetconfull{\derive{\mdata}{\mtcon}} = \setbuild{\mtrace}{\mdata\mtrace \in \denotetconfull{\mtcon}}$
\end{theorem}

\begin{theorem}[Top level partial]\label{thm:top-partial}
 $\denotetcon{\derive{\mdata}{\mtcon}} = \setbuild{\mtrace}{\mdata\mtrace \in \denotetcon{\mtcon}}$
\end{theorem}

The latter three fall out of the first from simple inductions.
%
The first theorem depends on the following lemma, which follows from a simple induction.

\begin{lemma}[Nullability]\label{lem:nullability}
  $\nullable(\motcon) = \epsilon \iff \epsilon \in \denotetconfulle{\motcon}{\menv}$
\end{lemma}

\autoref{thm:full} has a straightforward proof except in the $\neg$ case, shown below:
%
\begin{byCases}
  \iftwocolumn{}
  {\fontsize{8pt}{9pt}\selectfont}
  \case{\motcon \equiv \stnot{\motcon{}'}}{
    \begin{byCases}
      \case{H : \nullable(\derivee{\mdata}{\motcon{}'}{\menv}) = \epsilon}{
        \begin{pfsteps*}
          \item{$\denotetconfull{\derivee{\mdata}{\motcon}{\menv}} = \emptyset$} \BY{computation}
          \item{$\epsilon \in \denotetconfull{\derivee{\mdata}{\motcon{}'}{\menv}}$}
            \BY{$H$, lemma \ref{lem:nullability}} \pflabel{deriveeps}
          \item{$\mdata \in \denotetconfulle{\motcon{}'}{\menv}$} \BY{IH, \pfref{deriveeps}}
        \end{pfsteps*}
        To show $\setbuild{\mtrace}{\mdata\mtrace \in \denotetconfulle{\motcon}{\menv}} = \emptyset$, we suppose $\mtrace \in \semneg{\denotetconfulle{\motcon{}'}{\menv}}$ and show $\mtrace \nequiv \mdata\mtrace'$:
        \begin{byCases}
          \case{\mtrace \equiv \mdata\mtrace'}{
            Since $\mdata \in \denotetconfulle{\motcon{}'}{\menv}$, by definition of $\neg$, contradiction.}
          \otherwise{$\mtrace$ not prefixed by $\mdata$}
        \end{byCases}}
      \case{H : \nullable(\derivee{\mdata}{\motcon{}'}{\menv}) = \bot}{
\newcommand{\lhs}{\semneg{\setbuild{\mtrace}{\mdata\mtrace \in \denotetconfulle{\motcon{}'}{\menv}}}}
\newcommand{\rhs}{\semneg{\denotetconfulle{\motcon{}'}{\menv}}}
        \begin{pfsteps*}
          \item{$\epsilon \notin \denotetconfull{\derivee{\mdata}{\motcon{}'}{\menv}}$}
             \BY{lemma \ref{lem:nullability}} \pflabel{derivenoeps}
          \item{$\setbuild{\mtrace}{\mdata\mtrace \in \denotetconfulle{\motcon{}'}{\menv}} = \denotetconfull{\derivee{\mdata}{\motcon{}'}{\menv}}$} \BY{IH} \pflabel{IH}
          \item{$\mdata \notin \denotetconfulle{\motcon{}'}{\menv}$} \BY{\pfref{derivenoeps}, \pfref{IH}}
          \item{Goal is $\lhs = \setbuild{\mtrace}{\mevent\mtrace \in \rhs}$} \BY{computation}
        \end{pfsteps*}
        We prove this goal by bi-containment:
        \begin{byCases}
          \case{\mathit{Hs} : \mtrace \in \lhs}{
            \begin{pfsteps*}
              \item{$\forall \mtrace' \in \setbuild{\mtrace}{\mevent\mtrace \in \denotetconfulle{\motcon{}'}{\menv}}\setminus\setof{\epsilon}. \mtrace' \nleq \mtrace$}
                  \BY{$\mathit{Hs}$ and inversion} \pflabel{Hinv}
               \item{Suppose $\mtrace' \in \denotetconfulle{\motcon{}'}{\menv}$} \pflabel{let}
               \item{$\mtrace' \nleq \mevent\mtrace$}
                 \BY{\pfref{Hinv}, \pfref{let}, prefix cancellation} \pflabel{concl}
               \item{$\mtrace \in \rhs$} \BY{\pfref{concl}}
            \end{pfsteps*}}
          \case{\mathit{Hs} : \mtrace \in \rhs}{
            \begin{pfsteps*}
              \item{$\forall \mtrace' \in \denotetconfulle{\motcon{}'}{\menv}\setminus\setof{\epsilon}. \mtrace' \nleq \mevent\mtrace$}
                \BY{$\mathit{Hs}$, inversion} \pflabel{Hsinv1}
              \item{Suppose $\mtrace' \in \setbuild{\mtrace}{\mevent\mtrace \in \denotetconfulle{\motcon{}'}{\menv}} \setminus\setof{\epsilon}$} \pflabel{let}
              \item{$\mevent\mtrace' \in \denotetconfulle{\motcon{}'}{\menv}$} \BY{\pfref{let}} \pflabel{in}
              \item{$\mtrace' \nleq \mtrace$} \BY{\pfref{Hsinv1}, \pfref{in}, prefix cancellation} \pflabel{concl}
              \item{$\mtrace \in \lhs$} \BY{\pfref{concl}}
            \end{pfsteps*}}
        \end{byCases}}
    \end{byCases}}
\end{byCases}
%z

An interesting corollary relating paths to repeated derivation:
\begin{corollary}
  $\mtrace \in \denotetcon{\mtcon} \iff \nullable(\derive{\mtrace}{\mtcon}) = \epsilon$
\end{corollary}
% %
% Depends on
% \begin{lemma}[Flat emptiness]
%   $\flatempty(\mtcon) \implies \denotetconfull{\mtcon} = \emptyset$
% \end{lemma}
% where $\flatempty$ and $\flatempty_\menv$ are defined as (where $[_\menv]$ denotes the definition applies to both)
% \begin{align*}
%   \flatempty[_\menv](T) &= \bot \quad\text{if } T\equiv\epsilon,\, T\equiv(\stmany{\mtcon}),\,\text{or }T\equiv\stnot{\mtcon} \\
%   \flatempty[_\menv](\stbind{\mevent}{\mtcon}) &= \flatempty[_\menv](\mevent) \\
%   \flatempty[_\menv](\stOr{\isset{\mtcon}}) &= \bigwedge \flatempty[_\menv](\mtcon)\ldots \\
%   \flatempty[_\menv](\stAnd{\isset{\mtcon}}) &= \bigvee \flatempty[_\menv](\mtcon)\ldots \\
%   \flatempty[_\menv](\stseq{\mtcon_0}{\mtcon_1}) &= \flatempty[_\menv](\mtcon_0) \vee \flatempty[_\menv](\mtcon_1) \\
%   \flatempty(\mtcon,\menv) &= \flatempty_\menv(\mtcon) \\[2pt]
%   \flatempty[_\menv](\snone) &= \top \\
%   \flatempty_\menv(\mvar) &= (\menv(\mvar) \overset{?}{=} \emptyset) \\
%   \flatempty[_\menv](\snegpat{\mvpat}) &= \top \quad\text{if } \mvpat\equiv\sany,\, \mvpat\equiv\sbind{\mvar}, \text{ or } \mvpat\equiv\snegpat{\mvpat'} \text{ and } \flatempty[_\menv](\mvpat') \\
%   \flatempty[_\menv](\mvpat) &= \bot \quad\text{otherwise}
% \end{align*}
% %
% Our implementation reduces temporal contracts at construction time, so that the following invariant holds of represented contracts:
% \begin{equation*}
%  \flatempty(\mtcon) \iff \mtcon = \bot
% \end{equation*}


%%
\section{Semantics}\label{sec:technical}

Now that we have a clear view of the behavior of temporal contracts, we nail down a formal semantics that we use to prove the correctness of our monitoring system.
%
The semantics we present is in the style of Felleisen's reduction semantics~\citep{ianjohnson:Felleisen:2009:SEP:1795772}, which can be systematically transformed into an abstract machine in the form presented in~\citet{dvanhorn:VanHorn2010Abstracting}.
%%

The language itself is standard ISWIM, just extended with contract-monitoring forms, whose syntax and semantics we will focus on.
%
%%
\subsection{Syntax}
%\FloatBarrier

\begin{figure}
\begin{align*}
\mexp \in \Expr &::=
      \snlam{\mtoplevelname}{\mvar}{\mexp}
% These are already in overview
% \alt \sSMon{k}{l}{k}{\mtimeline}{\mscon}{\mexp}
% \alt \sTMon{k}{l}{j}{\mscon}{\mtcon}{\mexp}
 \alt \stblame{k}{j} 
 \alt \scev{k}{j}{\mtimeline}{\mval_f}{\mexp}
 \alt \text{other forms}
\end{align*}
\caption{Syntax additions}
\label{fig:syntax}
\end{figure}

%
A structural monitor $\sSMon{k}{l}{j}{\mtimeline}{\mscon}{\mexp}$ is like a traditional monitor \cite{ianjohnson:dthf:complete}: the structural contract is given by $\mscon$ and the contract parties are given by labels: $k$ is the server, $l$ is the client, and $j$ is the contract itself.
%
However, our structural monitors additionally feature a timeline ($\mtimeline$), used to temporally reason over values as they flow through contract boundaries (cf. \autoref{sec:temporal-semantics}).
%

%
A temporal monitor $\sTMon{k}{l}{j}{\mscon}{\mtcon}{\mexp}$ has a structural component as well ($\mscon$), but importantly it also considers a temporal contract given by $\mtcon$.
%
\iflong{At this point we would like to note that our usage of labels differs slightly from existing literature \cite{ianjohnson:dthf:complete}: we have a notion of the ``top level label'', denoted by $\toplevel$; this is similar to \citeauthor{ianjohnson:dthf:complete}'s use of $l_0$.}
%%

%%
Expressions of the form $\sret{k}{j}{\mtimeline}{\mval}{\mexp}$ denote the production of a return event on the $\mtimeline$ timeline, blaming $k$ for failures, where $\mval$ is the function being returned from, and $\mexp$ has evaluated to the return value.
%
Call events are similar.
%
The $\stblame{l}{j}$ expression denotes a temporal contract failure (of the contract $j$), which is a fatal error which blames $l$.
%%
%
After arguments to a call, or return values are wrapped, we get a blessing from the temporal monitor that the call or return is acceptable.
%
\newcommand*{\namefmt}[1]{\textit{\textsc{#1}}}
This is embodied in the $\namefmt{wrap}$ and $\namefmt{event}$ rules in \autoref{fig:reduction}.
%
Functions wrapped in arrow contracts are annotated with the name in the contract ($\snlam{\mtoplevelname}{\mvar}{\mexp}$) for matching purposes, but otherwise these functions have the same behavior as non-annotated functions.
%
%
\iflong{
Our evaluation contexts in \autoref{fig:ctx} follow straightforwardly and are ownership-sensitive \cite{ianjohnson:dthf:complete}, though we leave out discussion of this topic, as we feel it is orthogonal to our work.}
%
%

%%
\subsection{Reduction} \label{sec:reduction}

A machine state ($\mstate$) is composed of the expression ($\mexp$), and store of temporal contracts ($\mTMons : \Timeline \to \wp(\TContract)$).
%
\begin{figure*}
\newcommand*{\update}[3]{#1^\prime = #1[#2 \mapsto #1(#2)\mathrel{\sqcup}{#3}]}
\newcommand*{\name}[1]{&\text{[\namefmt{#1}]}}
\newcommand*{\where}{\text{where }}
\newcommand*{\cwhere}{\phantom{\where}}
\centering
$\begin{array}{@{}l @{\ }c@{\ } l@{} r}
\mstate := \chevron{\mctx{\mmlab}[\mathit{redex}], \mTMons}
&\machstep&
\mstate^\prime := \chevron{\mctx{\mmlab^\prime}[\mathit{reduct}], \mTMons^\prime}
\name{rule-name}

\iflong*{
\\ \hline % If true
\sif{ \sOwn{\strue}{\mmlab} }{\mexp_1}{\mexp_2}
&\machstep&
\mexp_1
\name{if-true}
%
\\ % If false
\sif{ \sOwn{\sfalse}{\mmlab} }{\mexp_1}{\mexp_2}
&\machstep&
\mexp_2
\name{if-false}
%
\\ % Apply
\sapp{ \sOwn{\slam{\mvar}{\mexp}}{\mmlab} }{ \sOwn{\mval}{\mmlab} }
&\machstep&
\sown{ \sapp{\{ \sown{\mval}{\mmlab}/\mvar \}}{\mexp} }{\mmlab}
\name{apply}

\\
}
% Temporal monitor
{\\ \hline}
\sTMon{k}{\mmlab}{j}{\mscon}{\mtcon}{\mval}
&\machstep&
\sSMon{k}{\mmlab}{j}{\mtimeline}{\mscon}{\mval}
\name{tmon} \\
&&\where \update{\mTMons}{\mtimeline}{\setof{\mtcon}} \\
&&\cwhere \mtimeline = \salloc{\mstate}

\\ % Arrow monitor
\sSMon{k}{l}{j}{\mtimeline}{\sarr{\mtoplevelname}{\mscon_D}{\mscon_R}}{\mval}
&\machstep&
\self \qquad \text{given } \mval \equiv \slam{\mvar}{\mexp}
\name{wrap} \\
&&\where \self = \snlam{\mtoplevelname}{\mvar}{\sret{k}{j}{\mtimeline}{\self}{\mexp_{\mathit{rng}}}} \\
&&\cwhere \mexp_{\mathit{rng}} = \sSMon{k}{l}{j}{\mtimeline}{\mscon_R}{
                               \sapp{\mval}{\mexp_{\mathit{call}}}} \\
&&\cwhere \mexp_{\mathit{call}} = \scall{l}{j}{\mtimeline}{\self}{\mexp_{\mathit{arg}}} \\
&&\cwhere \mexp_{\mathit{arg}} = \sSMon{l}{k}{j}{\mtimeline}{\mscon_D}{\mvar}

\\ % Arrow monitor fail
\ditto
&\machstep&
\sblame{k}{j} \qquad\text{given } \mval \nequiv \slam{\mvar}{\mexp}
\name{wrap-fail}

\\

\iflong{
\\ % Flat monitor
\sSMon{k}{\mmlab}{j}{\mtimeline}{ \sown*{\sflat{\mexp}}{\mmlab^{\prime\prime}} }{ \sOwn{\mval}{\mmlab^{\prime\prime}} }
&\machstep&
\schk*{k}{j}{\mtimeline}{ \sapp{\mexp}{\mval} }{\mval}
\name{smon-flat}
%
\\ % Cons monitor
\sSMon{k}{\mmlab}{j}{\mtimeline}{ \sconsc{\mscon_A}{\mscon_D} }{ \sOwn{\vcons{\mval_A}{\mval_D}}{\mmlab} }
&\machstep&
\sown{\vcons{ \sSMon{k}{\mmlab}{j}{\mtimeline}{\mscon_A}{\mval_A} }{ \sSMon{k}{\mmlab}{j}{\mtimeline}{\mscon_D}{\mval_D} }}{\mmlab}
\name{smon-cons}
}

\iflong{
\\ % Check true
\schk{k}{j}{ \sOwn{\strue}{j} }{\mval}
&\machstep&
\mval
\name{chk-true}
%
\\ % Check false
\schk{k}{j}{ \sOwn{\sfalse}{j} }{\mval}
&\machstep&
\sblame{k}{j}
\name{chk-false}
}

\\ % Call/Return
\scev{k}{j}{\mtimeline}{\mval_f}{\sown{\mval}{o}}
&\machstep&
\sown{\mval}{k}
\name{event} \\
&&\where \mtcons_\mvaluation = \Delta\setbuild{\derive{\scevev{\mval_f}{\mval}}{\mtcon}}{\mtcon\in\mTMons(\mtimeline)} \\
&&\cwhere \mTMons' = \mTMons[\mtimeline \mapsto \gamma_1?(\mtimeline) \to \mtcons, \mTMons(\mtimeline) \sqcup \mtcons] \\
&&\text{given } \mvaluation \neq \bot \\

\\ % Call/Return fail
\ditto
&\machstep&
\stblame{k}{j}
\quad\where \ditto, \text{given } \mvaluation \neq \must
\name{TCon-fail}

\\ \hline % Blame from structural monitor
\mctx{\mmlab}[\sblame{k}{j}]
&\machstep&
\sblame{k}{j}
\name{blame}

\\ % Blame from temporal monitor
\mctx{\mmlab}[\stblame{k}{j}]
&\machstep&
\stblame{k}{j}
\name{TCon-blame}

\end{array}$
\caption{Reduction rules}
\label{fig:reduction}
\end{figure*}

%
We give our reduction relation in \autoref{fig:reduction}.
%
In the interest of space, we only present and discuss the rules which we feel are most relevant to our discussion of temporal contracts.
%

%
The \namefmt{tmon} rule stores the given temporal contract inside the store ($\mTMons$), associated with an allocated timeline ($\mtimeline$) and joined with any pre-existing temporal contracts on that timeline (as abstract allocation cannot always produce fresh timelines).
%
The structural component of the contract ($\mscon$) is then used to rewrap the monitored value ($\mval$) inside of a structural monitor (${\tt smon}$) associated with the timeline.
%

%
%The \namefmt{wrap} rules handle the reduction of structural monitors.
%
\iflong{%
Flat contracts (in \namefmt{smon-flat}) on constants are easily translated into a check on the contract.
%
Cons-list contracts (in \namefmt{smon-cons}) are also simple given our evaluation contexts: they are translated into an owned $\mathtt{cons}$, with each component appropriately wrapped by its respective structural contract: $\mscon_A$ for the \texttt{car} and $\mscon_D$ for the \texttt{cdr}.
%
}
%
Since temporal contracts monitor the calls and returns of functions, we use new syntactic forms to communicate the call and return events (with the corresponding wrapped values) to the temporal monitor.
%
When an event is permissible by the temporal monitor, we reduce to the wrapped value.
%
An event is permissible if the temporal contracts on the timeline do not derive to the empty contract.
%
\iflong{Finally, note that the evaluation context's owner becomes the callee after this step (in line with the call).}
%%
%
%%
Since multiple temporal contracts can be on a timeline, and a timeline might be aliased in the abstract, we have two points to focus on in the $\namefmt{event}$ rule.
%
We reuse $\Delta$ from the definition of matching in order to combine the possible outcomes of derivation due to weak equality; any difference in valuation means we must conservatively blame.
%
When a timeline corresponds to one concrete timeline (given by $\gamma_1?$, always true in the concrete), we can destructively update the timeline state.
%
If the timeline is aliased, then we abstractly cannot decide which contract belongs to which alias of the timeline (they can even be identical contracts on two different timelines), so we join with the derivation to simultaneously step and not step all contracts.
%
%
\iflong{
An expression's event trace is defined as the string of events that would be passed to the temporal monitor, were we to leave in the ${\tt call}$ and ${\tt ret}$ redexes.
%
We restrict it to a particular timeline in the following corecursive definition:
\begin{align*}
 \denote{\mexp}_\mtimeline &= \prefixes\left(\left\{
   \begin{array}{ll}
      \setbuild{\scevev{\mval_f}{\mval}\mtrace}
               {\mtrace \in \denote{\mctx{\mmlab^\prime}[\mval]}_\mtimeline}
       &\text{if } \mexp\equiv\mctx{\mmlab^\prime}[\scev{k}{j}{\mtimeline}{\mval_f}{\mval}] \\
      \bigcup\setbuild{\denote{\mexp'}}{\mexp \machstep \mexp'} &\text{otherwise}\end{array}\right.\right)
\end{align*}
%%
Since timelines get allocated freshly and temporal contracts get introduced in later computation, our monitor's correctness condition must take into account all possible monitor constructions along execution.
%
Therefore, we state correctness as a correspondence between blame and (full) adherence to contracts after any amount of computation has produced some amount of temporal monitors.
%
The well-formedness criterion ($\wf$) requires emptiness to mean blame, and forces sane initial conditions; if an expression adheres to a temporal contract, it can never lead to an empty monitor by definition.
\begin{align*}
 \wf(\chevron{\mexp, \mTMons}) &= \forall \mtimeline\in\dom(\mTMons). \mTMons(\mtimeline) = \emptyset \implies \mexp \equiv \mctx{\mmlab^\prime}[\stblame{k}{j}] \\
 \adhere(\chevron{\mexp, \mTMons}) &= \forall \mtimeline\in\dom(\mTMons). \denote{\mexp}_\mtimeline \subseteq \bigcup\limits_{\mtcon \in \mTMons(\mtimeline)}{\denotetcon{\mtcon}} \\
 \good(\mtimeline, \mstate) &= \forall \mstate'\equiv\chevron{E[\scev{k}{j}{\mtimeline}{\mval_f}{\mval}], \mTMons}. \mstate \machstep^* \mstate' \implies \\
                            &\qquad \forall \mTMons'. \mstate' \centernot\machstep \chevron{E[\stblame{k}{j}], \mTMons'} \\
 \alias(\mtimeline, \mstate) &= \exists \mstate'\equiv\chevron{E[\sTMon{k}{l}{j}{\mscon}{\mtcon}{\mval}], \mTMons}. \\
                           &\qquad\mstate \machstep^* \mstate' \machstep \chevron{E[\sSMon{k}{l}{j}{\mtimeline}{\mscon}{\mval}], \mTMons\sqcup[\mtimeline \mapsto \mTMons(\mtimeline) \sqcup \setof{\mtcon}]} \\
 \blame(\mtimeline, \mstate) &= \exists \mstate'\equiv\chevron{E[\scev{k}{j}{\mtimeline}{\mval_f}{\mval}], \mTMons}, \mTMons'. \\
                             &\qquad\mstate \machstep^* \mstate' \machstep \chevron{E[\stblame{k}{j}], \mTMons'} \\
\end{align*}
%
%%
\begin{lemma}[Well-formedness preserved]\label{lem:wf}
  If $\wf(\mstate)$ and $\mstate \machstep \mstate'$ then $\wf(\mstate')$.
\end{lemma}
\begin{proof}
  Cases on $\machstep$.
\end{proof}
%
\begin{lemma}[Adherence means aliasing or no blame]
  For all $\mstate\equiv\chevron{\mexp, \mTMons}$,
  if $\adhere(\mstate)$ then
  $\forall \mtimeline \in \dom(\mTMons). \alias(\mtimeline, \mstate) \vee \good(\mtimeline,\mstate)$.
\end{lemma}
%
\begin{lemma}[Later blame means eventual non-adherence]
  For all $\mstate\equiv\chevron{\mexp, \mTMons}$,
  if $\exists \mtimeline \in \dom(\mTMons). \blame(\mtimeline,\mstate)$ then
  $\exists \mstate'. \neg\adhere(\mstate') \wedge \mstate \machstep^* \mstate'$.
\end{lemma}
%
\begin{theorem}[Monitor soundness]
  If $\overline{\mstate} = \chevron{\mexp,\mTMons} \machstep^* \chevron{\mctx{\mmlab^\prime}[\stblame{k}{j}], \mTMons'}$, $\wf(\chevron{\mexp,\mTMons})$ and $\inv(\chevron{\mexp, \mTMons})$ then $\exists \mstate \in \overline{\mstate}.\neg\inv(\mstate)$
\end{theorem}
Proof by induction and \autoref{thm:top-partial}.
\iflong{
  \begin{proof}
    \begin{byCases}
      \case{\text{Base}}{Vacuously true by definition of $\inv$.}
      \case{\text{Induction step }\overline{\mstate} \equiv
        \chevron{\mexp, \mTMons} \machstep^* \chevron{\mexp'', \mTMons''} \machstep \chevron{\mexp', \mTMons'}}{
        By cases on the final step
         $\chevron{\mctx{\mmlab^\prime}[\mexp''], \mTMons''} \machstep
          \chevron{\mctx{\mmlab^{\prime\prime}}[\mexp'], \mTMons'}$:
        \begin{byCases}
          \case{\scev{k}{j}{\mtimeline}{\mval_f}{\sown{\mval}{o}} \machstep \stblame{k}{j}}{
            Absurd by definition of $\inv$ and \autoref{thm:top-partial}.}
          \otherwise{No blame}
        \end{byCases}
      }
    \end{byCases}
  \end{proof}}
%
\begin{theorem}[Monitor completeness]
  For all $\mstate\equiv\chevron{\mexp,\mTMons}$,
  if $\neg\inv(\mstate)$ and $\wf(\mstate)$. then $\exists k,j, \mTMons'. \mstate \machstep^* \chevron{\mctx{\mmlab^\prime}[\stblame{k}{j}], \mTMons'}$.
\end{theorem}
\iflong{
  \begin{proof}
    Trivial in the case that $\exists \mtimeline. \mTMons(\mtimeline) = \emptyset$.
    By definition of $\inv$, there must be some timeline $\mtimeline$, temporal contract $\mtcon$, and event trace $\mtrace$ such that $\mtrace \notin \denotetcon{\mTMons(\mtimeline)}$
  \end{proof}
}
%
Notice that the invariant about adherence is not limited by the number of steps it is allowed to take.
%
Since the expression throughout reduction adheres to each contract on each created timeline, after no amount of steps will the monitor be able to blame.
}
%%

\section{Abstract semantics}

We use the AAM approach to soundly approximate our concrete semantics.
%
This first requires a CESK-like machine to transform, but there are known ways to systematically transform Felleisen-style reduction semantics into corresponding CESK machines ~\citep{dvanhorn:Danvy-Nielsen:RS-04-26}.
%
The important aspect of the abstraction process is not this transformation, but the fact that the space of values becomes \emph{finite}, meaning the space of temporal contract derivatives is finite (proven via simple coinduction).
%
In our semantics with just closures, conses, booleans and integers, we consider:
\begin{itemize}
 \item syntactically equal closures and conses to be ``\emph{maybe}-equal'';
 \item equal booleans and integers to be ``\emph{must}-equal'';
 \item integer comparisons with the abstract $\mathbf{Int}$ element to be ``\emph{maybe}-equal'' (assuming a simple flat abstraction of the integers);
 \item and all other possibilities to be ``\emph{never}-equal''.
\end{itemize}
%
Any time a temporal contract is in a \emph{may} state, we blame conservatively.
%
Since timelines can be abstractly allocated more than once, each time we send an action to the monitor, we have to both step and not step the contract, which can kill precision and performance.
%
On top a basic abstraction from AAM, we evaluate the following techniques to improve the precision of the analysis:
\begin{itemize}
\item{($\mu$) use abstract counting ~\citep{dvanhorn:Might:2006:GammaCFA} to improve equality checking of allocated data and provide a non-trivial $\gamma_1?$;}
\item{($\Gamma$) use abstract garbage collection~\citep{dvanhorn:Might:2006:GammaCFA} to remove dead temporal monitors;}
\item{($\Gamma_\tau$) use abstract garbage collection to remove bindings to dead values in live temporal monitors.
%
This means we reduce $\mtcon,\menv$ to $\mtcon,\menv'$, where $\menv'$ restricts variables to the values in $\menv$ that touch only reachable addresses.
%
The concrete semantics would operate the same way regardless of GC, but this is akin to using weak boxes to maintain values in temporal contract environments;}
\item{($\Xi$) separate continuation management to a pushdown abstraction~\citep{dvanhorn:Vardoulakis2011CFA2}.}%; or}
%\item{use a polyvariant allocation strategy such as $m$-CFA ~\citep{dvanhorn:Might2010Resolving} or polymorphic splitting~\citep{dvanhorn:wright-jagannathan-toplas98}.}
\end{itemize}
%%

\section{Evaluation}
%%

\begin{figure}
  \input{bench-overview}
  \caption{Benchmark results. Numbers are run-time (sec) / $\frac{\text{spurious blames}}{\text{possible blames}}$}
  \label{fig:evaluation}
\end{figure}
%%
We built temporal higher-order contracts into our existing framework for analyzing a subset of Scheme \footnote{Model and benchmarks available online \url{http://github.com/dvanhorn/oaam/tree/thocon}}.
%
Our benchmarks are a collection of examples from \dfm's paper and implementation:
\paragraph{{\tt sort1}:}{\dfm's motivating example of a non-reentrant sort function where the given comparison function may not be called after sort returns\footnote{If sort stashes the comparator in a global and we call it afterwards, \dfm's implementation does not raise blame, whereas ours does.}.
%
The structural contract for {\tt sort1} is weak: {\tt (listof integer?) -> any/c}.
}
\paragraph{{\tt sort2}:}{like {\tt sort1}, but with a stronger structural contract that will confuse a regular analysis: {\tt (listof integer?) -> (listof integer?)}.}
\paragraph{{\tt sort3}:}{like {\tt sort2}, but call sort more than once (mapped over a list of lists), in order to confuse non-$\Gamma_\tau$ analyses.}
\paragraph{{\tt malloc}:}{A pair of two functions, malloc and free, where malloc returns an ever-growing counter and free does nothing. They are contracted so that free may not be called with addresses that were previously freed and not subsequently returned by malloc.}
\paragraph{{\tt file}:}{A function, {\tt open}, which given a file path, produces a list of read, write and close functions. They are contracted such that none of these functions may be called after close. Using this interface, we copy the contents of files in one list and write them to the files in the other.}
%\paragraph{{\tt FTP}:}{An FTP server built on TCP. \textbf{TODO: Alex}}
%%

%%
The analysis results and run times for our benchmark programs are in \autoref{fig:evaluation}, categorized by the additional machinery as labeled in the previous section ($m$ means used more than 2GiB limit).
%
Analyses without garbage collection used the global store widening to accelerate convergence.
%%

%%
The minor variations to the sort example motivated our additional analytical machinery.
%, \textbf{TO BE SEEN} and proved strong enough to verify the more complicated example of the FTP interaction.
%
No technique we employed could fully verify the {\tt malloc} example, since the protocol depends on being able to prove equality between values that are mutated and grown without bound; our abstraction for numbers is simple, so the mutated counter immediately jumps to \textbf{Int}$\top$, where comparisons conservatively both succeed and fail.
%
The {\tt file} example could be verified had we used a polyvariant allocation strategy, since it involved opening two files at the same time with the same handle allocation code.
%%

%%
Notice that without a pushdown abstraction, abstract garbage collection searched a state space orders of magnitude more than with.
%
The $\Gamma$ and $\Gamma_\tau$ implementations do not use a pushdown abstraction, but they still can verify {\tt sort2}, which was meant to confound non-pushdown analyses.
%
The reason for this is that in AAM, continuations are allocated on the heap, so GC can reclaim them for better return flow predicition; recursive calls do not get the same precision boost.
%
The topic of making abstract GC more performant and feasible is an active area of research, most recently extended to pushdown abstractions \citep{dvanhorn:Earl2012Introspective}.
%
Although greater precision can increase the possible state space, clever abstractions combined with typical program structure can actually reduce the explored state space, as is evident by the dearth of visited blame sites in our most precise implementation.

%%
\section{Related Work}
%%

%%
Model-checking and contract verification (proving functional correctness) are huge fields and it is important to view our work in the greater context of these worlds of research.
%
The main separating factor between this work and the model-checking literature is that temporal property verification is \emph{extra-linguistic}, meaning there is no mechanism in the object language that could monitor for the properties.
%
Furthermore, work that focuses on the linguistic mechanism does not also focus on the verification mechanism.
%%

%%
\paragraph{Runtime monitoring:}
Monitoring sequences of actions at runtime is a mature and active area of research.
%
This area has similarities to temporal contracts due to the use of runtime monitoring and of languages for describing execution traces, but nothing in the area has the concept of blame or has a static story.
%
The notion of an action is reminiscent of aspect-oriented programming's notion of a \emph{join-point}, and thus we see several systems built on AspectJ \citep{ianjohnson:aspectj} that offer a domain-specific language for running \emph{advice} when the action trace matches a specified pattern, \eg, Tracematches \citep{ianjohnson:Allan05addingtrace} and J-LO ~\citep{ianjohnson:jlo}.
%
Tracematches use a language similar to temporal contracts but do not support negation; they also have a different purpose: execute advice at more specific times based on the program history, and not to offer a high-level specification system with blame that contracts provide.
%
J-LO on the other hand offers a monitoring system based on LTL propositions with binding constructs that tracks the satisfaction of the LTL proposition with the assumption that future portions of the proposition hold.
%
If the \emph{now} portion of the proposition fails to hold, the monitor raises a failure (it does not blame).
%
The language for temporal contracts is an extension of regular expressions with back references, as many desirable properties are difficult to express in LTL with back-references (DLTL ~\cite{ianjohnson:jlo})\footnote{A sentiment expressed by DLTL's inventor~\citep{boddenadmission}}.
%
Temporal contracts and DLTL can talk about value flow and use over time via binding in the specification --- this is not something that LTL can locally express, and DLTL does not currently have any model checking tool-support.
%
J-LO's goal is closer to temporal contracts, but its language is not; conversely, tracematches match the language and not the goal.
%
In both cases, the only static analysis is on the specification itself in order to improve runtime performance, and not on the monitored program's adherence to the specification.
%
Both systems are also tied to Java's class structure, so they cannot express properties of higher-order behavior or refinements on values.

\paragraph{Higher-order model-checking:}
Java and C++ both have several high-quality model-checking tools \citep{ianjohnson:bandera, ianjohnson:java-pathfinder, ianjohnson:LLBMC}, some of which are bounded model-checkers; meaning they cannot fully verify temporal properties --- only present possible counter-examples.
%
Bandera~\citep{ianjohnson:bandera} is a collection of tools that uses static analysis techniques to extract a finite model from a Java program to feed to select back-end model-checkers.
%
Similar to our approach, Bandera employs flow analysis in order to produce compact models.
%
Unsimilarly, it does not synthesize checkers for runtime monitoring the expressed properties, nor does it natively support higher-order functions.
%
A complete separation of model generation and model-checking also degrades precision, since the more in-depth constraint solving typical model-checkers do can help prune the control-flow space; our approach is amenable to integrated constraint solving and is left to future work.
%%

%%
A technique that specifically targets higher-order languages, higher-order recursion schemes (HORS)~\citep{ianjohnson:Knapik:2002:HPT:646794.704852}, is rooted in simply-typed, call-by-name lambda terms, but has model-checking solutions that have been extended to call-by-value ~\citep{ianjohnson:DBLP:journals/jacm/Kobayashi13} and untyped ~\citep{dvanhorn:Tsukada2010Untyped} languages, through heavy type-theoretic machinery.
%
Model-checking an untyped HORS is undecidable, and such model checkers make various approximations biased to soundly model-check programs in traditional type systems rather than traditional untyped languages.
%
Our technique is lighter weight and more transparently correct since it follows from a systematic transformation of a standard semantics.
%
Additionally, the AAM approach makes extensions to more complex language features straightforward, whereas in HORS one would need to CPS, double CPS, or perform a functional encoding of a new form of data; all of which impose additional proof obligations and points-of-failure for the analysis implementor.
%
Finally, HORS do not synthesize runtime monitors or have a notion of blame, unlike our system.
%%

%%
\paragraph{Behavioral contract verification:}
In the world of static sotware contract verification, there is more closely related work.
%
There have been many successful efforts in the realm of first-order contract verification~\citep{ianjohnson:fahndrich:contracts:2011,ianjohnson:vcc:2009}, but the techniques employed are inherently first-order: the only values are booleans.
%
\citet{ianjohnson:Flanagan:2006:HTC:1111037.1111059}'s notion of hybrid type checking is one way to state the problem: dynamic types are essentially flat contracts, and are treated as subtypes of anything during static checking.
%
If an external theorem prover can prove that the flat contracts always hold, the dynamic checks can be safely removed.
%
\citet{dvanhorn:Xu2012Hybrid} describes a higher-order contract verification system for OCaml by inlining all contract monitors and relying on a system of simplifications further enhanced by an SMT solver to optimize away dynamic checks.
%
\citet{dvanhorn:TobinHochstadt2012Higherorder} use AAM on a module semantics with higher-order contracts and is the most related to this work.
%
However, whereas their work focuses on a concrete semantics for handling unknown values and an external theorem prover to show contract containment, our work focuses on an orthogonal issue of temporal contract monitoring.
%
Our techniques should smoothly integrate with theirs when considering partial programs, and is left to future work (\cref{sec:conclusion}).
%%

%%
\section{Conclusion and future work} \label{sec:conclusion}
%%

%%
We demonstrated that a linguistic construction for monitoring temporal properties of higher-order programs (temporal higher-order contracts) can be transparently abstracted to provide a sound verification algorithm, almost for free, given the AAM approach.
%
Our preliminary evaluation provides evidence that this can be an effective approach to verifying programs with finite protocols, given the right combination of existing analysis machinery.
%
The way forward is clear: build a quality temporal monitoring library and contract a large project to find weaknesses in the language of temporal contracts, and finally evaluate the verification algorithm on this large example.
%
Potential weaknesses we can identify:
\begin{itemize}
\item{the blame story needs better understanding, with deotonic logic:
%
currently, the party that generates the action that causes a contract to fail gets blamed, and there is no way to express that some action must happen before the end of the monitor's lifetime.
%
The action-emitting party might be innocent, just working at the behest of a different party that is obliged not to violate the contract.
%
Instead of attaching a temporal contract to a single structural contract with the {\tt tmon} form, we instead allow structural contracts to \emph{emit} temporal contracts as obligations of a given party, and additionally blame for unfinished obligations at monitor collection time, end of execution, or after a specified amount of time;
}
%
\item{calls and returns don't match in the temporal contracts themselves:
%
the actions are currently interpreted without a notion of matching, which could turn out to be too flat and limiting for proper specifications --- the implementation from \dfm{} allows labeling calls so that return actions can refer to the matching call.
%
It is unclear whether an analysis can precisely match calls and returns in both the contracts and control-flow, due to the several stacks that apparently entails, but at least a ``good enough'' approximation should exist;
}
%
\item{some specifications are better suited to DLTL or state machines:
%
some examples in \dfm's test suite included using side effects in the temporal monitor's matching to encode state transitions.
To more adequately handle this idiom without sophisticated handling of side-effects, we may wish to mix and match ways in which we express properties.
%
For example, a contract in a different language is considered primitive, and only continues when the contract is accepting (much like sequence contracts work now).
}
%
\item{whole program analysis is intractible at scale:
%
we can verify on a per-module basis if we add temporal contracts to a semantics of partial programs, such as in \citet{dvanhorn:TobinHochstadt2012Higherorder}.
%
The primary difficulty there is a sound definition of {\tt havoc} in the presence of temporal contracts, which simulates all possible interactions with a module.
%
Additional concerns lie in adequately maintaining enough information about unknown values to prune search space, without also exploding the search space with all the distinguished unknown values, but there is existing work on this that we can leverage~\citep{ianjohnson:DBLP:journals/cacm/DilligDA10}.
}
\end{itemize}
%
%
