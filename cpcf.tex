
\section{Introduction}\label{sec:intro}

Software systems are large, consist of many modules, and have invariants that are either outright inexpressible or too costly to express (and prove) in the language's static type system --- if it has one.
%
When this is the case, one might hope to rely on software contracts~\cite{dvanhorn:meyer-eiffel} to give dynamic guarantees about the behavior of one's system.
%
In modern higher-order languages, the question of ``who violated the contract?'' becomes non-trivial, and we need higher-order contracts~\citep{dvanhorn:Findler2002Contracts} to blame the correct party responsible for any violation of these invariants.
%
Disney, Flannagan and McCarthy proposed a system of \emph{temporal higher-order contracts}~\citep{ianjohnson:DBLP:conf/icfp/DisneyFM11} (DFM) to provide a linguistic mechanism for describing temporal properties of behavioral values flowing through the program.
%
%when traditionally such properties are checked extra-linguistically by model-checking an abstracted form of the program.
%
Example temporal properties are, ``a file can only be closed if it has been opened'' and, in the higher-order setting, ``if function A is given a function B, then B may not be called once A returns.''
%
Such invariants are important for interfaces that have set-up and tear-down protocols to follow, or even pure interfaces that have particular compositions of calls needed to construct some object.
%%

%%
There are downsides to software contracts: no static guarantees, and to a lesser extent, runtime overhead.
%
Since contracts are runtime monitors, they do not themselves ensure correctness---though their blame reporting helps the process of constructing correct programs.
%
Verification technology provides an additional level of confidence in correctness or pinpoints means for failure; its early use can even accelerate development with its bug-reporting capabilities.
%
A sound model-checker can justify safely removing contract checking and have a performance-and-correctness return on the initial investment in contract design and runtime overhead.
%%

%%
Temporal contracts pose an additional challenge over higher-order function contracts to statically verify, as they monitor the progression of interactions with a module over time, and not just localized interactions at module boundaries.
%
We propose a framework that is composable with techniques to verify functional contracts, and has low technical overhead (\eg, no translation to a model-checker's language necessary).
%
The technique uses the Abstracting Abstract Machines (AAM) approach \citep{dvanhorn:VanHorn2010Abstracting} to check for reachability of a temporal contract blame.
%%

%%
AAM was originally targeted towards describing flow analyses, but is robust enough to apply to model-checking safety properties of higher-order programs.
%
We develop a new semantics for temporal contracts that has an operational interpretation by means of regular expression derivatives~\citep{ianjohnson:Brzozowski1964}.
%
This allows us to add the temporal property checking as part of the language's semantics itself, and we can soundly and finitely abstract it with little effort.
%
Model checking a program with respect to its temporal contracts then just amounts to running it, post abstraction, to witness the absence of blame.
%%

%%
We chose this method over translating to an existing model-checker for a few reasons, the most important of which is the lack of linguistic support for the form of binding we have in temporal contracts.
%
Beyond the language is the technical and run-time overhead required to use existing model-checkers.
%
It is well-known that control-flow analysis can be stated in terms of model-checking~\citep{ianjohnson:analysis-is-mc}, but this observation comes with a cost: one must use a quadratic number of model-checking queries to discover what a monovariant flow analysis can discover in one run.
%
The query-reducing techniques used amount to running a flow analysis, so we simply solve the entire problem with a flow analysis; we do not need to feed multiple queries to a model-checker that has some representation of the model we discover in the analysis.
%
\iflong{
In general, the queries are at least linear in the size of the abstract value space, which is exponential in some polyvariant analyses such as 1-CFA. % TODO: cite Van Horn, Mairson?
%
For temporal contracts, to emulate binding in a logic that lacks it, we would (roughly) have to enumerate binding and reference pairs over the entire value space.
%
We overcome the need to generate queries for a model-checker by embedding the temporal checking in an abstract semantics, and checking on-the-fly.
%%
}
Our contributions in this paper are as follows:
\begin{itemize}
 \item{an alternative to \dfm's temporal contract semantics amenable to analysis yet faithful to their examples;}
 \item{a notion of derivative for regular expressions with back-references;}
 \item{a sound abstraction to computably verify temporal contract satisfaction.}
\end{itemize}
%%

%%
The outline of the paper by section:

%%
\begin{itemize}
\item{\autoref{sec:overview} provides an intuition for how temporal
  contracts work, and presents their syntax.}
%
\item{\autoref{sec:temporal-semantics} reviews the semantics proposed in
  \dfm, discusses problem points, and defines a new semantics along
  with a proven-correct notion of derivative for regular expressions
  with back-references in the scope of our non-standard semantics of
  negation.}
%
\item{\autoref{sec:technical} describes an operational semantics of the
  language forms needed to express temporal contracts, using the
  aforementioned derivative function to give an operational
  interpretation of ``stepping'' temporal contracts.}
%
\item{\autoref{sec:abstract-semantics} introduces the process of
  abstracting the previously defined semantics to a finite
  approximation.
%
  It further describes how to use existing analysis techniques to enhance the basic abstraction to improve the precision of the analysis in ways that specifically target temporal contracts.}
%
\item{\autoref{sec:evaluation} is an initial evaluation of our technique on an assortment of benchmarks.}
%
\item{\autoref{sec:related} puts our work into the greater context of verification, model-checking and abstract interpretation.}
%
\item{Finally, \autoref{sec:conclusion} concludes with a discussion of directions for future work.}
%
\end{itemize}
\section{Overview of temporal higher-order contracts}\label{sec:overview}
%%

%\FloatBarrier
Temporal contracts provide a declarative language for monitoring the temporal ordering of actions that pass through module boundaries.
%
We begin with a simple example that exhibits the expressiveness of temporal higher-order contracts to frame the discussion.
%
The following example comes from the original work on temporal higher-order contracts, and its behavior led us to explore alternative semantics.
%
\newcommand{\sortid}{\mathit{sort}}
\subsection{\dfm's sort example} \label{sec:sort}
%
This example presents the contract for a hypothetical $\sortid$ function which takes two arguments: a comparator and a list (of positive numbers), is non-re-entrant, and furthermore cannot make its given comparator available to be called after it's done sorting.
%
We express the function component of the contract (with labels given to function components) like so:
\begin{align*}
  \sortid : (\mathit{cmp} : \mathit{Pos} \to \mathit{Pos} \to \mathit{Bool})\ (List\ Pos) \to (List\ Pos)
\end{align*}

We can express the temporal component of the contract in a natural way with the following:
\renewcommand*{\arraystretch}{1.2}
\newcommand*{\call}[1]{\scallev{#1}{\_}}
\newcommand*{\ret}[1]{\sretev{#1}{\_}}
\begin{align*}
  &\stnot{(\sddd {\tt call}(\sortid, \_,\_)~ \stmany{\snonevent{{\tt ret}(\sortid, \_)}}~ {\tt call}(\sortid,\_,\_))}
  \\\cap & \stnot{(\sddd~ \stbind{{\tt call}(\sortid, ?\mathit{cmp}, \_)}{\sddd~ {\tt ret}(\sortid, \_)~ \sddd~ {\tt call}(\mathit{cmp}, \_, \_)})}
\end{align*}
%
The first clause expresses non-re-entrance, phrased as a negation of a trace reentering the function: after a call to $\sortid$ and some actions that aren't returns from $\sortid$, there is another call to $\sortid$.
%
We use $\_$ or $\sany$ as a wild card to match any argument or return value.
%%

%%
The second clause of the temporal component specifies a higher-order property; given a call to $\sortid$, its associated $\mathit{cmp}$ argument cannot be called after $\sortid$ returns.
%
We use angle brackets around actions that we want to bind values from, using the ? binding form.
%
Since $\mathit{cmp}$ will be wrapped with its higher-order contract at each call, which creates new values, the bindings for $\mathit{cmp}$ will be distinct across execution.
%%
These contracts are attached to a sort implementation with a special expression form, {\tt tmon}, that additionally requires labels for who are the contracted parties.
%
The intention of the regular-expression notation is to say, ``as long as the trace is a prefix of these strings of actions, the temporal contract is satisfied.''
%
For example, $A$ satisfies the contract $ABC$, but $ABD$ doesn't.
%%

Since the semantics is prefix closed, we can take the state of a regular expression parser as an indicator for whether the contract system should blame.
%
We chose regular expression derivatives for this purpose due to their simplicity, though we extended them to allow for back-referencing values captured in binding forms.
%
Consider the following faulty trace for an interaction with $\sortid$ that violates the higher-order component of the temporal contract:
\newcommand*{\wrapv}{\mathit{wrap}}
\begin{align*}
  {\tt call}(\sortid, \le, {\tt `}(2\ 1))~{\tt call}(\wrapv, 2, 1) ~{\tt ret}(\sortid, {\tt `}(1\ 2))) ~{\tt call}(\wrapv, 0, 1) \\
  \text{where } \wrapv  = \lambda x y. ({\tt if}\ ({\tt and}\ ({\tt pos?}\ x)\ ({\tt pos?}\ y))\ (\le\ x\ y)\ \sblame{\mathit{client}}{\mathit{contract}})
\end{align*}

The contract system applies the regular expression derivative for each action in the trace as it arrives, and blames as soon as derivation fails.
%
The derivatives, cumulatively (previous derivation has one less prime), are
\begin{align*}
  \derive{{\tt call}(\sortid, \le, {\tt `}(2\ 1))}{\mtcon} &= \stnot{\stOr{\setof{\mtcon_0, \stmany{\snonevent{{\tt ret}(\sortid, \_)}} ~{\tt call}(\sortid, \_, \_)}}},
    \\ &\cap \stnot{\stOr{\setof{\mtcon_1, ({\sddd~ {\tt ret}(\sortid, \_)~ \sddd~ {\tt call}(\mathit{cmp}, \_, \_)}, \menv)}}} \\
%
  \derive{{\tt call}(\le, \wrapv, 2, 1)}{\mtcon'} &= \mtcon' \\
%
  \derive{{\tt ret}(\sortid, {\tt `}(1\ 2))}{\mtcon''} &= \stnot{\mtcon_0} \cap \stnot{\stOr{\setof{\mtcon_1, (\sddd~ {\tt call}(\mathit{cmp}, \_, \_), \menv)}}} \\
%
  \derive{{\tt call}(\wrapv, 0, 1)}{\mtcon'''} &= \stnot{\mtcon_0} \cap \stnot{\stOr{\setof{\mtcon_1, \epsilon}}} = \stnot{\mtcon_0}\cap \bot = \bot \\
%
  \text{where } \mtcon_0 & = \sddd {\tt call}(\sortid, \_,\_)~ \stmany{\snonevent{{\tt ret}(\sortid, \_)}}~ {\tt call}(\sortid,\_,\_) \\
  \mtcon_1 &= \sddd~ \stbind{{\tt call}(\sortid, ?\mathit{cmp}, \_)}{\sddd~ {\tt ret}(\sortid, \_)~ \sddd~ {\tt call}(\mathit{cmp}, \_, \_)} \\
  \menv &= [\mathit{cmp} \mapsto \wrapv]
\end{align*}

The final state has a negated nullable contract, which we regard as a failing state.
%
This is because a regular expression derives to $\epsilon$ through a string $w$ iff $w$ is accepted by the regular expression.
%
We interpret non-empty regular expressions as contracts that are following an allowed path, and have not yet violated it.
%
This interpretation implies prefix closure of our partial trace semantics in \autoref{sec:temporal-semantics}.

\subsection{Syntax of contracts}

The general forms for expressing structural contracts and monitoring values are in \autoref{fig:scontract-syntax}.
%
The other forms of the language are irrelevant, but we of course assume the existence of $\lambda$ expressions.
%
Labels are used to express the identities of parties engaged in a software contract.
%
We have three parties: the provider of a value ($k$, the server, or positive party), the consumer of a value ($l$, the client, or negative party), and the provider of the contract ($j$, the contract, or self party).
%
Timelines, $\mtimeline$, are associated with unique runtime monitors, and are generated by evaluating {\tt tmon} expressions.
%%

\begin{figure}
  \begin{align*}
    \mscon \in \SContract &::= \sflat{\mexp} \alt \sarr{\mtoplevelname}{\mscon}{\mscon} \alt \sconsc{\mscon}{\mscon}
\\
    \mexp \in \Expr &::= \sTMon{k}{l}{j}{\mscon}{\mtcon}{\mexp}
                    \alt \sSMon{k}{l}{j}{\mtimeline}{\mscon}{\mexp}
                    \alt \text{other forms}
\\
\mmlab,k,l,j \in \Label&\text{ an infinite set} \\
\mtimeline \in \Timeline &\text{ an infinite set}
  \end{align*}
  \caption{Syntax of structural contracts with labels}
  \label{fig:scontract-syntax}
\end{figure}

%%
A structural monitor $\sSMon{k}{l}{j}{\mtimeline}{\mscon}{\mexp}$ is a behavioral monitor \citep{ianjohnson:dthf:complete}: the structural contract is given by $\mscon$, where actions of contracted values will be sent to the runtime monitor associated with the given timeline $\mtimeline$.
%
A temporal monitor $\sTMon{k}{l}{j}{\mscon}{\mtcon}{\mexp}$ produces a timeline $\mtimeline$, contracts $\mexp$ with $\mscon$ with respect to the timeline, and associates $\mtcon$ with $\mtimeline$.

\begin{figure}
  \begin{align*}
  \motcon \in \TContract^\circ &::=
      \mevent \alt \snonevent{\mevent}
 \alt \epsilon
 \alt \stnot{\motcon}
 \alt \stseq{\motcon}{\motcon}
 \alt \stmany{\motcon{}} \\
&\alt \stOr{\isset{\motcon}}
 \alt \stAnd{\isset{\motcon}}
 \alt \stbind{\mevent}{\motcon}
\\
\mtcon \in \TContract &= \text{same rules as } \TContract^\circ \text{ for } \mtcon \text{ plus } \alt \motcon, \menv \\
\menv \in \Env &= \Var \to \wp(\Value) \\
\mevent \in \Action &::= \scevev{\mname}{\mvpat} \alt \sany \\
\mcev \in \FunctionAction &::= {\tt call} \alt {\tt ret} \\
\mtrace \in \mathit{Trace} &= \Action^* \\
\mvpat \in \VPat &::= \mval \alt \sbind{\mvar} \alt \mname \alt \scons(\mvpat,\mvpat) \alt \snegpat{\mvpat} \alt \sany \alt \snone \\
\mvar \in \Var &\quad\text{an infinite set} \\
\mname \in \Name &::= \mvar \alt \mtoplevelname
  \end{align*}
  \caption{Syntax of temporal contracts}
  \label{fig:tcontract-syntax}
\end{figure}

%
The syntax is presented in \autoref{fig:tcontract-syntax}.
%
%
We notate sets of some element type with metavariable $e$ as $\isset{e}$, and lists as $\many{e}$.
%
We write the interpolation of sets and lists into many arguments as $e\ldots$, following the rules of \citet{dvanhorn:Kohlbecker1987Macrobyexample}.
%
The if-then-else syntax we use follows Dijkstra ($\mathit{guard} \to \mathit{then}, \mathit{else}$), where anything non-$\bot$ is considered true.
%
We use $x\equiv S(y,\ldots)$ to mean ``$x$ matches $S(y,\ldots)$;'' alternatively, there exist elements $y\ldots$ such that $x = S(y,\ldots)$.
%
For brevity, we will use $\mname(\mvpat)$ and $\retof{\mname}(\mvpat)$ to mean $\scallev{\mname}{\mvpat}$ and $\sretev{\mname}{\mvpat}$ respectively.
%%

%%
Structural contracts include subsets of first-order data that satisfy a particular predicate ($\sflat{\mexp}$), functions with associated structural contracts on their domain and range in addition to a label to reference within the temporal contract ($\sarr{\mtoplevelname}{\mscon_D}{\mscon_R}$), and {\tt cons} pairs whose components are contracted ($\sconsc{\mscon_A}{\mscon_D}$).
%%
Temporal contracts include actions ($\mevent$), negated actions ($\snonevent{\mevent}$), action matching scoped over a following contract ($\stbind{\mevent}{\mtcon}$), concatenation ($\stseq{\mtcon}{\mtcon}$) (often written using juxtaposition), negated contracts ($\stnot{\mtcon}$), Kleene closure of events ($\stmany{\mtcon}$), union ($\stOr{\isset{\mtcon}}$), intersection ($\stAnd{\isset{\mtcon}}$), the empty temporal contract ($\epsilon$), and an open temporal contract closed by an environment ($\motcon, \menv$).
%
We consider the fail contract $\bot$ as a macro for $\stOr{\setof{}}$, and \dfm's universal contract $\sddd$ a macro for $\stAnd{\setof{}}$.
%
The difference between $\snonevent{\mevent}$ and $\stnot{\mtcon}$ is that the first must be an action and force time to step forward once, whereas the second may match arbitrarily many actions.
%%

%%
Actions themselves are expressed as patterns denoting calls ($\scallev{\mname}{\mvpat}$) or returns ($\sretev{\mname}{\mvpat}$), with respect to a particular function named $\mname$ and with its argument or result matching a pattern $\mvpat$.
%
If $\mname$ is a label ($\mtoplevelname$), we simply check that the monitor wrapping the function has the same label.
%
Arrow contract monitors impose their label on the contracted function.
%
However, if $\mname$ is a variable ($\mvar$), then we consult a binding environment that the monitoring system builds as we pass binding actions to determine if the function is exactly equal to the value bound.
%
Patterns can match values ($\mval$), variable bindings and references ($\sbind{\mvar}$, $\mvar$), labeled functions ($\mtoplevelname$), structured data ($\scons(\mvpat,\mvpat)$), negated patterns that do not bind ($\snegpat{\mvpat}$), anything or nothing ($\sany$, $\snone$).
%%
%
%%
%
%%
\iflong{
\subsection{File example} \label{sec:file}
%
\begin{align*}
 &FileSystemContract\, =\, open\, :\, String\, \to\, FileContract \\
 &FileContract =~ Record \\
 &\begin{array}{ @{\quad~}l@{\ :} @{~}l@{\ \to\ } l }
  read & Unit & String \\
  write & String & Unit \\
  close & Unit & Unit
 \end{array} \\
 &\text{where}\quad \sddd~ \ret{close}
 %\caption{File example}
 %\label{fig:file}
\end{align*}
%
%%
This example gives the contract for a hypothetical file system, which can be used to open files by giving the $open$ function a filename (a $String$); the client is then given a file handle contracted by $FileContract$.
%
A file handle, in turn, is a record of functions which interact with the file: $read$, $write$, and $close$, all which perform the expected behaviors.
%%
%
%%
The temporal contract is what is interesting: it is not phrased in terms of a negation, but rather an affirmation.
%
Its goal is to state that a user of the file is forbidden from making use of the file handle (through the use of its component functions) after the user has $close$d the file.
%
It is phrased such that there is no ``$\sddd$'' at the end of its trace; this means that the last reference one can make to such a monitored record is returning from $close$; after that, it cannot be used.
%
%TODO: Do we need to define prefix-closure?
Note that this is not a \emph{liveness property}; this does not mean that a return from $close$ \emph{must} happen, as traces are \emph{prefix-closed}.
%
Instead, the property is a \emph{safety property}, though expressed in the affirmative.
%%
%
%%
\subsection{TCP example} \label{sec:tcp}
%
\newcommand*{\tcpc}{\mathit{TCPConnection}}
\newcommand*{\tcpcc}{\mathit{TCPConnectionContract}}
\newcommand*{\tcpsock}{\mathit{TCPSocket}}
\newcommand*{\tcpsendc}{\mathit{TCPSendContract}}
\newcommand*{\tcprecvc}{\mathit{TCPRecvContract}}
\newcommand*{\tcpdata}{\mathit{TCPData}}
\newcommand*{\tcpstyle}[1]{\texttt{#1}}
%\FloatBarrier
\begin{figure}
 \newcommand*{\send}[1]{\scallev{send}{#1}}
 \newcommand*{\rcv}[1]{\sretev{recv}{#1}}
 \newcommand*{\notclose}{\snonevent{\call{close}}}
 \newcommand*{\tcpsyn}{\tcpstyle{SYN}}
 \newcommand*{\tcpack}{\tcpstyle{ACK}}
 \newcommand*{\tcpsynack}{\tcpstyle{SYN\&ACK}}
 \newcommand*{\tcpfin}{\tcpstyle{FIN}}
 %\newcommand*{\tcpfinack}{\tcpstyle{FIN\&ACK}}
 $\tcpcc\, =~ Record$ \\
 $\begin{array}{ @{\quad~}l@{\ :} @{~}l @{\ \to\ }l }
  open & \tcpsock & \tcpsendc \\
  listen & \tcpsock & \tcprecvc
 \end{array}$ \\
 $\tcpdata =$
 $~ \tcpsyn \mid \tcpack \mid \tcpsynack \mid \tcpfin \mid \tcpstyle{Data}(\_)$
%
 $\tcpc =~ Record$ \\
 $\begin{array}{ @{\quad~}l@{\ :} @{~}l@{\ \to\ } l }
  send & \tcpdata & Unit \\
  recv & Unit & \tcpdata \\
  timeout & Duration & Unit \\
  close & Unit & Unit
 \end{array}$ \\
%
 $\tcpsendc =~ \tcpc$ \\
 $\begin{array}{ @{~}r@{\quad} l @{}l }
  \text{where}
  &&\send{\tcpsyn}~ \notclose \\
  &&\rcv{\tcpsynack}~ \notclose \\
  &&\send{\tcpack}~ \stmany{\notclose} \\
%
  &&\cup
  \begin{aligned}
   &\left(
    \begin{aligned}
     &\rcv{\tcpfin}~ \send{\tcpack} \\
     &\send{\tcpfin}~ \rcv{\tcpack}
    \end{aligned}
   \right) \\
   &\left(
    \begin{aligned}
     &\call{close}~ \send{\tcpfin} \\
     &\rcv{\tcpack}~ \rcv{\tcpfin}~ \send{\tcpack}
    \end{aligned}
   \right)
  \end{aligned} \\
  &&\ret{timeout}~ \ret{close}
 \end{array}$
%
 $\tcprecvc =~ \tcpc$ \\
 $\begin{array}{ @{~}r@{\quad} l @{}l @{}l }
  \text{where}
  &&\rcv{\tcpsyn}~ \notclose \\
  &&\send{\tcpsynack}~ \notclose \\
  &&\rcv{\tcpack}~ \stmany{\notclose} \\
%
  &&\cup
  \begin{aligned}
   &\left(
    \begin{aligned}
     &\call{close}~ \send{\tcpfin} \\
     &\rcv{\tcpack}~ \rcv{\tcpfin}~ \send{\tcpack}
    \end{aligned}
   \right) \\
   &\left(
    \begin{aligned}
     &\rcv{\tcpfin}~ \send{\tcpack} \\
     &\call{close}~ \send{\tcpfin}~ \rcv{\tcpack}
    \end{aligned}
   \right)
  \end{aligned} \\
  &&\ret{timeout}~ \ret{close}
 \end{array}$
%
 \caption{TCP example}
 \label{fig:tcp}
\end{figure}
%
% TODO: Source this from WP
\begin{figure}
 \centering
 \fontsize{4}{5} \selectfont
 \def \svgwidth{\columnwidth}
 \input{tcp-fsm.pdf_tex}
 \caption{Simplified TCP FSM for \autoref{fig:tcp}}
 \label{fig:tcp-fsm}
\end{figure}
%
%%
In \autoref{fig:tcp} we show the contract for a hypothetical TCP connection module.
%
A client of this module may $open$ a $\tcpsock$ for initiating a connection or may $listen$ to a $\tcpsock$ for passively connecting.
%
A $\tcpc$ is defined similarly to our file system example; it is a record of functions which interact with the connection: $send$, $recv$, $timeout$, and $close$.
%
Notably, $send$ and $recv$ interact with data of the form $\tcpdata$, which can be one of the special packets used in the TCP protocol or can simply be some amount of user data ($\tcpstyle{Data}(\_)$).
%
%TODO: maybe express timing out in a better manner?
The $timeout$ function is unimportant to our discussion; it is used internally by the module to set timeouts for interactions with the other end of the TCP connection; clients do not directly use it and it is included solely for the ability to reason over timeouts in the temporal contract.
%%
%
%%
A socket on the ``sending'' end of the TCP connection (having used $open$) is obliged to use the connection according to the temporal component of $\tcpsendc$.
%
Likewise, a socket on the ``receiving'' end of the TCP connection (having used $listen$) is constrained by $\tcprecvc$.
%
The contracts are very involved, but they implement a simplified version of the TCP connection life cycle given in \autoref{fig:tcp-fsm}.
%
Note that the temporal clause in each contract is phrased in the affirmative, as in \autoref{sec:file}; however, unlike that example, this property is indeed a (restricted) kind of \emph{liveness property}.
%
It states that the client uses the socket in a manner consistent with the TCP protocol and can expect the other end to likewise adhere.
%
The key difference between this property and a traditional liveness property is that it speaks of a liveness property with respect to \emph{a particular contract-monitoring} of a $\tcpc$; prefix-closure is still present at the top level (cf. \autoref{fig:tcontract-denotation}) and thus the protocol noted in the temporal contract may not occur if no $\tcpc$ is used.
%%
}
%%
\section{Semantics of Temporal Contracts} \label{sec:temporal-semantics}

%% FIXME
We present and analyze a slightly different formulation than \dfm's temporal contracts that allows for more precise specification of value-use.
%
But first, we must discuss why we do not import \dfm's semantics directly.
%
\subsection{\dfm's semantics}
\newcommand{\denotedfm}[2]{\denote{#1}_{#2}}
\begin{figure}
  \begin{align*}
    \denotedfm{\bullet}{\bullet} &: \TContract^\circ \times \Env \to \wp(\mathit{Trace})
    \\
    E \in \Env &::= \epsilon \alt E, \mvar : \rho \mscon
    \\
    \rho &\in \setof{{\tt send}, {\tt recv}} \text{, and }\sim \text{ a notion of ``matches.''}
  \end{align*}
  \begin{align*}
    \denotedfm{\scevdfm{\mtoplevelname}{\mvpat}}{E} &=
      \setbuild{\rho.\scevdfm{y}{d}}{E(y) \equiv \sarr{\mtoplevelname}{\_}{\_}, \mvpat \sim d}
    \\
    \denotedfm{\snonevent{\mevent}}{E} &= \Event \setminus \denotedfm{\mevent}{E}
    \\
    \denotedfm{\stseq{\motcon_0}{\motcon_1}}{E} &= \denotedfm{\motcon_0}{E} \semseq \denotedfm{\motcon_1}{E}
    \\
    \denotedfm{\stmany{\motcon{}}}{E} &= \stmany{\denotedfm{\motcon}{E}}
    \\
    \denotedfm{\stnot{\motcon}}{E} &= \mathit{Trace} \setminus \denotedfm{\motcon}{E}
    \\
    \denotedfm{\stOr{\isset{\motcon}}}{E} &= \bigcup\denotedfm{\motcon}{E}\ldots
    \\
    \denotedfm{\sddd}{E} &= \mathit{Trace}
    \\
    \denotedfm{\stbind{\scevdfm{\mtoplevelname}{? \mvar}}{\motcon}}{E} &= \setbuild{\rho.\scevdfm{y}{c}\mtrace}{E(y) \equiv \sarr{\mtoplevelname}{\_}{\_}, \mtrace \in \denotedfm{\motcon[\mvar := c]}{E}}
  \end{align*}
  \caption{\dfm's semantics of temporal contracts}
  \label{fig:dfm-semantics}
\end{figure}
%
The semantics presented in \autoref{fig:dfm-semantics} is a recollection of the denotational semantics that \dfm{} gives to temporal contracts.
%
They use a module semantics based on an $\mathit{EF}$ machine that tracks the bindings shared across module boundaries, $E$, and a stack of module boundaries to return to, $F$.
%
Regardless of how this machine works, the denotation of a temporal contract attached to a structural contract, $\denote{\mscon\ {\tt where}\ \motcon}$, is generated by traces of $\mathit{EF}$ that are driven by sent or received calls and return events (roughly):
\begin{align*}
 \left\{
   \begin{array}{l}
    {\tt send.ret}(\mathit{start},h)\mtrace \in \prefixes(\denotedfm{\motcon}{E}) :
 \\ \qquad\langle E_0, \mathit{start}\rangle \Rightarrow^\mtrace \langle E, F\rangle \wedge E_0 = \epsilon, h : {\tt send}S
\end{array}\right\}
\end{align*}

The use of $\prefixes$ in this definition is problematic, and negation is the culprit.
%
Contracts that state anything about how a trace may not end would allow just such traces since \emph{extensions} to such ``bad traces'' are acceptable, and prefix closure will throw the ``bad traces'' back into what is acceptable.
%
%Additionally, if one writes a contract more carefully to reject extensions of bad traces, there isn't an obvious operational interpretation that allows early failure.
%
For example, the denotation of temporal contracts from \dfm{} allows $\mevent\mevent \in \prefixes(\denote{\stnot{\mevent}})$, and because of prefix closure, $\mevent \in \prefixes(\denote{\stnot{\mevent}})$!
%
Temporal contract failure should not be contingent on future observations; an effective monitor should blame \emph{as soon as} a contract is not satisfied.
%%

%%
\dfm's semantics for referring to functions is additionally problematic; we give a slightly different account that captures the spirit of their prose describing their system, and indeed mirrors their actual implementation.
%
The temporal component of the example discussed in \autoref{sec:sort} was originally the following:
\begin{align*}
 \stnot{(\sddd~ \sortid(\_,\_)~ \stmany{\snonevent{\retof{\sortid}(\_)}}~ \sortid(\_,\_))} 
 \ \cap\  \stnot{\sddd~ \retof{\sortid}(\_) \sddd~ \mathit{cmp}(\_,\_)}
\end{align*}

In contrast to our restatement, the flat use of \emph{labels} instead of \emph{bindings} would cause a second call to a supposedly-correct $\sortid$ to fail, since it internally calls the comparator of the same label, but of a different monitor construction.
%
Their implementation works around this by additionally adding a monitor-wrapping event, that generates a new label to pair with the function label to uniquely identify it.
%
Notice also the use of $\sddd$ instead of $\stmany{\snonevent{\mathit{following\text{-}event}}}$.
%%

%% TODO: restate
A third shortcoming is that \dfm's universal contract $\sddd$ was intended to be interpreted lazily, though was not in the formal semantics.
%
Without laziness, the original statement of $\sortid$'s temporal contract does not have a definitive blame point: by the second clause, we cannot eventually call $\mathit{cmp}$, but is the first call of $\mathit{cmp}$ the one that was meant, or is it part of the $\sddd$? To be sure, let's keep going.
%
Laziness makes the first sight of a call to $\mathit{cmp}$ a definite transition out of the $\sddd$.
%
We do not directly address the issue of laziness in this work, but rather restate contracts to work around this interpretation.
%
We conjecture that this nuance in the semantics is why a program that blatantly violates this contract by storing the comparison function in a global, and calling it afterwards, is why \dfm's implementation did not blame.
% laziness poses challenges to the derivative parsing approach we have taken, so we leave the study of laziness to future work.
%%
%% FIXME: location
%\dfm's implementation also includes call/return matching via a different binding at the call site to eventually use in the matching return.
%
%We leave call/return matching to future work, as it is not necessary to verify the motivating examples.
%%

\subsection{Our semantics}
%
As noted in \autoref{sec:sort}, temporal contracts are associated with structural contracts that label function components within them.
%
For simple exposition, we will consider tuples as the main organizational tool for contracting the interactions between multiple functions.
%
Since we consider monitor constructions as a more basic notion of equality, we also see each temporal monitor construction as starting its own \emph{timeline}, which sees its own filtered view of events in the system.
%
Thus, as values flow through contract boundaries, they are considered on different timelines.
%
\dfm{} formalized their semantics in terms of a nondeterministic machine that defined its interactions on all event streams, and thus their machine was on a single timeline.
%
The semantics of temporal contracts that we propose uses a weak equality for comparisons of non-primitive data, which we evaluate as structural equality up to closures, where we use pointer-equality.
%
Our semantics makes interaction between temporal contract monitors explicit, allowing us to verify whole programs.
%%

%
To combat the problems introduced by \dfm's original use of $\prefixes$ on top of a semantics of full traces, we give a different semantics for temporal contracts that alternates between \emph{full traces} and \emph{partial traces}.
%
Additionally, a negated temporal contract will reject all non-empty full traces of the given contract, as well as any extension of such traces.
%
This semantics of negation does not satisfy double-negation elimination (DNE), but we find that to be an acceptable trade-off; we do not need DNE in order to implement or verify temporal contracts.
%
We claim that this semantics is what \dfm{} intended their system to mean, as it matches up with the expectations of their prose, the test cases in their implementation\footnote{The functional test cases, in particular, since our model does not handle Racket's object system.}, and additionally raises blame on programs that \dfm{} were surprised their implementation accepted.
%
Since our semantics catches more ``bad'' behavior than their monitoring system, we have not simply formalized their implementation.
%%

%%
The denotational semantics in \autoref{fig:tcontract-denotation} lends itself nicely to online monitoring via a derivative parsing approach.
%
For each event $\mevent$ we come across on a timeline $\mtimeline$, we set $\mTMons(\mtimeline)$ to $\derive{\mevent}{\mTMons(\mtimeline)}$, as long as the derivative isn't failing.
%
We can show that the partial trace semantics is prefix-closed (\autoref{thm:prefix-closed}), and thus we can pinpoint the cause of an error as soon as it is sent to the monitor --- a necessary condition for effective blame management in this realm.
%%

%%
We made the design choice that call events for bound variables ($\scallev{\mvar}{\mvpat}$) will only be sent to the temporal monitor if the value bound to $\mvar$ is itself contracted on the monitor's timeline.
%
The reason for this is that control should flow back to the timeline considered in order for an event to affect that timeline.
%
It is easy enough to amend the structural contracts to reflect the fact that a binding is considered a function in the temporal contract.

\begin{theorem}[Prefix closure]\label{thm:prefix-closed}
  $\prefixes(\denotetcone{\mtcon}{\menv}) = \denotetcone{\mtcon}{\menv}$
\end{theorem}

%\FloatBarrier
\newcommand*{\tconsemfigs}[4]{
 \iftwocolumn{\begin{figure}#1 #2\end{figure}
              \begin{figure}#3 #4\end{figure}}
             {\begin{figure}
              \begin{minipage}[b]{.55\linewidth}#1 #2\hrule height 0pt\end{minipage}
              \begin{minipage}[b]{.40\linewidth}#3 #4\hrule height 0pt\end{minipage}
             \end{figure}}}
\tconsemfigs{
    \begin{align*}
      \denotetconbothe{\stOr{\isset{\motcon}}}{\menv} &=
      \bigcup\denotetconbothe{\motcon}{\menv}\ldots
      \\
      \denotetconbothe{\stAnd{\isset{\motcon}}}{\menv} &=
      \bigcap\denotetconbothe{\motcon}{\menv}\ldots
      \\
      \denotetconbothe{\epsilon}{\menv} &= \setof{\epsilon}
      \\
      \denotetconbothe{\stnot{\motcon}}{\menv} &=
      \semneg{\denotetconfulle{\motcon}{\menv}}
      \\[2pt]
      \denotetcone{\stmany{\motcon{}}}{\menv} &=
      \denotetconfulle{\stmany{\motcon{}}}{\menv} \semseq
      \denotetcone{\motcon}{\menv}
      \\
      \denotetcone{\stseq{\motcon_0}{\motcon_1}}{\menv} &=
      \denotetcone{\motcon_0}{\menv} \cup
      \denotetconfulle{\motcon_0}{\menv}\semseq
      \denotetcone{\motcon_1}{\menv}
      \\
     \denotetcone{\stbind{\mevent}{\motcon}}{\menv} &=
      \setof{\epsilon}\cup
        \{ \mdata \mtrace : \mdata,\menv' \in \denoteevent{\mevent}{\menv},
        \\&\phantom{= \setof{\epsilon}\cup \{ \mdata\mtrace : } \mtrace \in \denotetcone{\motcon}{\menv'} \}
     \\
      \denotetcone{\mevent}{\menv} &= \setof{\epsilon}\cup\denotetconfulle{\mevent}{\menv}
      \\[2pt]
      \denotetconfulle{\stseq{\motcon_0}{\motcon_1}}{\menv} &=
      \denotetconfulle{\motcon_0}{\menv} \semseq
      \denotetconfulle{\motcon_1}{\menv}
      \\
      \denotetconfulle{\stmany{\motcon{}}}{\menv} &=
      \setbuild{\mtrace^i}{i \le \omega, \mtrace \in
        \denotetconfulle{\motcon}{\menv}}
      \\
     \denotetconfulle{\stbind{\mevent}{\motcon}}{\menv} &=
     \setbuild{\mdata \mtrace}{\mdata,\menv' \in \denoteevent{\mevent}{\menv}, \mtrace \in \denotetconfulle{\motcon}{\menv'}}
     \\
      \denotetconfulle{\mevent}{\menv} &= \setbuild{\mdata}{\mdata,\menv' \in \denoteevent{\mevent}{\menv}}
      \\[2pt]
      \denoteevent{\mevent}{\menv} &= \setbuild{\mdata,\menv'}{\menv'_\must = \matches(\mevent, \mdata, \menv)}
      \\
      \semneg{\Pi} &= \setbuild{\mtrace}{\forall \mtrace' \in
        \Pi\setminus\setof{\epsilon}. \mtrace' \nleq \mtrace}
      \\
      \Pi \semseq \Pi' &= \setbuild{\mtrace \cdot \mtrace'}{\mtrace
        \in \Pi, \mtrace' \in \Pi'}
    \end{align*}}{\caption{Denotational Semantics of Temporal Contracts ($B$ means both $P$ and $F$)}\label{fig:tcontract-denotation}}
  {\begin{align*}
      \derivee{\mdata}{\epsilon}{\menv} &= \bot
      \\
      \derivee{\mdata}{\mevent}{\menv} &= \left\{
        \begin{array}{ll}
          \epsilon & \text{if } \menv'_\must = \matches(\mevent, \mdata, \menv) \\
          \bot & \text{otherwise}
        \end{array}\right.
      \\
      \derivee{\mdata}{\stbind{\mevent}{\motcon}}{\menv} &=
      \left\{\begin{array}{ll}
          \motcon,\menv' & \text{if } \menv'_\must = \matches(\mevent, \mdata, \menv) \\
          \bot & \text{otherwise}
        \end{array}\right.
      \\
      \derivee{\mdata}{\stseq{\motcon_0}{\motcon_1}}{\menv} &=
      \stOr{\setof{\derivee{\mdata}{\motcon_0}{\menv},\
          \stseq{\nullable(\motcon_0)}{\derivee{\mdata}{\motcon_1}{\menv}}}}
      \\
      \derivee{\mdata}{\stOr{\isset{\motcon}}}{\menv} &=
      \stOr{\derivee{\mdata}{\motcon}{\menv}\ldots}
      \\
      \derivee{\mdata}{\stAnd{\isset{\motcon}}}{\menv} &=
      \stAnd{\derivee{\mdata}{\motcon}{\menv}\ldots}
      \\
      \derivee{\mdata}{\stmany{\motcon{}}}{\menv} &=
      \stseq{\derivee{\mdata}{\motcon}{\menv}}{\stmany{\motcon{}}}
      \\
      \derivee{\mdata}{\stnot{\motcon}}{\menv} &=
      \nullable(\derivee{\mdata}{\motcon}{\menv}) \to \bot,
      \stnot{\derivee{\mdata}{\motcon}{\menv}}
      \\[2pt]
      \nullable(\epsilon) &= \nullable(\stmany{\motcon{}}) =
      \nullable(\stnot{\motcon}) = \epsilon
      \\
      \nullable(\stbind{\mevent}{\motcon}) &= \nullable(\mevent) = \bot
      \\
      \nullable(\stOr{\isset{\motcon}}) &=
      \bigvee{\nullable(\motcon)\ldots}
      \\
      \nullable(\stAnd{\isset{\motcon}}) &=
      \bigwedge{\nullable(\motcon)\ldots}
      \\
      \nullable(\stseq{\motcon_0}{\motcon_1}) &=
      \nullable(\motcon_0)\wedge \nullable(\motcon_1)
      \\
      \nullable(\motcon,\menv) &= \nullable(\motcon)
    \end{align*}}{\caption{Derivatives of Temporal Contracts}\label{fig:tcon-deriv}}

The semantics and derivatives here are simplified to the concrete case, for space.
%
The extension to support the approximate matching semantics uses the appropriate logical connectives across $\cup$, $\cap$ and $\neg$ to compute the overall valuation of the temporal contract following a derivation.
%
Multiple possible matches (given from $\matches$) lead to multiple possible derivations, and the temporal contract connectives are lifted over these sets of derivatives.
%
Thus, derivatation produces a set of possible derivations and a valuation indicating how sure we are to \emph{not} blame.
%%

\newcommand*{\matchsemfigs}[4]{
 \iftwocolumn{\begin{figure} #1 #2 \end{figure}
              \begin{figure} #3 #4 \end{figure}}
             {\begin{figure}
              \begin{minipage}[b]{.50\linewidth}#1 #2\hrule height 0pt\end{minipage}
              \begin{minipage}[b]{.45\linewidth}#3 #4\hrule height 0pt\end{minipage}
             \end{figure}}} 
\matchsemfigs{
\setlength{\abovedisplayskip}{0pt}
\setlength{\belowdisplayskip}{4pt}
\setlength{\abovedisplayshortskip}{0pt}
\setlength{\belowdisplayshortskip}{8pt}
  \begin{align*}
    \matches &: \Pattern \times \Qualified \times \Env \to \wp(\MatchResult) \\
    \Delta &: \wp(\MatchResult) \to \MatchResult \\
    \delta &: \wp(\Valuation) \to \Valuation
  \end{align*}
    \begin{align*}
      \mvaluation \in \Valuation &::= \may \alt \must \alt \bot \\
      \mpat \in \Pattern &::= \VPat \text{ rules plus } \alt \mconstructor(\many{\mpat}) \\
      S \subset \MatchResult &::= \menvs_\mvaluation \\
      \menvs_\bot &= \emptyset_\bot = \emptyset_\mvaluation \\
      \mdata \in \Qualified &= \isset{\mdata} \alt \mval \alt \mconstructor(\many{\mdata}) \\
      \mconstructor \in \Constructors &= \setof{{\tt call}, {\tt ret}, {\tt cons}}
      \\[2pt]
      \must \wedge \must &= \must \\
      \bot \wedge \_ &= \_ \wedge \bot = \bot \\
      \may \wedge \_ &= \_ \wedge \may = \may
      \\[2pt]
      \neg \must &= \bot \\
      \neg \may &= \may \\
      \neg \bot &= \must
    \\[2pt]
    \menvs \bowtie \menvs' &= \setbuild{\combinef{\menv}{\menv'}}{\menv \in \menvs, \menv' \in \menvs'} \\
    \combinef{\menv}{\menv'} &= \lambda x. x \in \dom(\menv') \to \menv'(x), \menv(x)
    \end{align*}}
  {\caption{Spaces and functions for matching}\label{fig:matchspace}}
  {\begin{align*}
%    \matches &: \Pattern \times \Qualified \times \Env \to \MatchResult \\
    \matches(\sany, \_, \menv) &= \menv_\must \\
    \matches(\snone, \_, \menv) &= \emptyset_\bot \\
    \matches(\mtoplevelname, \snlam{\mtoplevelname}{\mvar}{\mexp}, \menv) &= \menv_\must \\
    \matches(\snegpat{\mvpat}, \mdata, \menv) &= \menv_{\neg \mvaluation} \\
    \text{where }& \matches(\mvpat, \mdata, \menv) = \menv'_\mvaluation \\
    \matches(\sbind{\mvar}, \mdata, \menv) &= \menv[\mvar \mapsto \mdata]_\must \\
    \matches(\mvar, \mdata, \menv) &= \menv_{\menv(\mvar) \simeq \mdata} \\
    \matches(\mconstructor(\many{\mpat}), \mconstructor(\many{\mdata}), \menv) &= (\Bowtie \menvs \ldots)_{\bigwedge \mvaluation \ldots} \\
    \text{where }& \menvs_\mvaluation \ldots = \matches(\mvpat, \mdata, \menv) \ldots \\
    \matches(\mpat, \isset{\mdata}, \menv) &= \Delta\setbuild{\matches(\mpat, \mdata, \menv)}{\mdata \in \isset{\mdata}} \\
    \matches(\mdata, \mdata', \menv) &= \menv_{\mdata \simeq \mdata'} \\
    \matches(\mpat, \mdata, \menv) &= \emptyset_\bot \quad
    \text{otherwise}
    \\[2pt]
    \Delta S &= (\bigcup\limits_{\menvs_{\_} \in S}\menvs)_{\delta \setbuild{\mvaluation}{{\_}_\mvaluation \in S}} \\
    \delta \emptyset &= \bot \\
    \delta \setof{\mvaluation} &= \mvaluation \\
    \delta \isset{t} &= \may \quad \text{otherwise}
  \end{align*}}{\caption{Semantics of matching}\label{fig:matchsem}}

We define our semantics of matching in anticipation of abstraction, where we do not always know when two values are equal.
%
When matching against a set of possible values, we might have a $\must$ match and a $\bot$ match, in which case the entire match should be considered $\may$ matching (this is the significance of the $\Delta$ metafunction).
%
The $\triangleleft$ operator extends the left environment with the bindings of the right, though the order doesn't matter considering that binding patterns may not bind the same variable twice.
%
Matching against sets of values makes it possible that we have several possible matches.
%
Thus $\matches$ returns a set of environments possible from matching a given pattern against some data.
%
At the leaves, when considering values of the language equal, $\matches$ appeals to a weak equality function, $\simeq$, where $\mval \simeq \mval' = \must$ implies $\mval = \mval'$ in the concrete semantics, and $\mval \simeq \mval' = \bot$ implies $\mval \neq \mval'$ in the concrete semantics.
%%

%%
We say $\mexp$ satisfies a temporal contract $\mtcon$ if its event trace filtered by the timeline to which the contract is attached is in the denotation of the temporal contract ($\denotetcon{\mtcon}$).
%
Since monitors are generated during reduction, the proof is mostly technical that our monitoring system ensures an expression either satisfies its contract or blames, and hinges mainly on the correctness of derivatives:
%

\begin{theorem}[Full]\label{thm:full}
 $\denotetconfull{\derivee{\mdata}{\motcon}{\menv}} = \setbuild{\mtrace}{\mdata\mtrace \in \denotetconfulle{\motcon}{\menv}}$
\end{theorem}

\begin{theorem}[Partial]\label{thm:partial}
 $\denotetcon{\derivee{\mdata}{\motcon}{\menv}} = \setbuild{\mtrace}{\mdata\mtrace \in \denotetcone{\motcon}{\menv}}$
\end{theorem}

\begin{theorem}[Top level full]\label{thm:top-full}
 $\denotetconfull{\derive{\mdata}{\mtcon}} = \setbuild{\mtrace}{\mdata\mtrace \in \denotetconfull{\mtcon}}$
\end{theorem}

\begin{theorem}[Top level partial]\label{thm:top-partial}
 $\denotetcon{\derive{\mdata}{\mtcon}} = \setbuild{\mtrace}{\mdata\mtrace \in \denotetcon{\mtcon}}$
\end{theorem}

The latter three fall out of the first from simple inductions.
%
The first theorem depends on the following lemma, which follows from a simple induction.

\begin{lemma}[Nullability]\label{lem:nullability}
  $\nullable(\motcon) = \epsilon \iff \epsilon \in \denotetconfulle{\motcon}{\menv}$
\end{lemma}

\autoref{thm:full} has a straightforward proof except in the $\neg$ case, which requires some tricky reasoning.
%
\begin{byCases}
  \iftwocolumn{}
  {\fontsize{8pt}{9pt}\selectfont}
  \case{\motcon \equiv \stnot{\motcon{}'}}{
    \begin{byCases}
      \case{H : \nullable(\derivee{\mdata}{\motcon{}'}{\menv}) = \epsilon}{
        \begin{pfsteps*}
          \item{$\denotetconfull{\derivee{\mdata}{\motcon}{\menv}} = \emptyset$} \BY{computation}
          \item{$\epsilon \in \denotetconfull{\derivee{\mdata}{\motcon{}'}{\menv}}$}
            \BY{$H$, lemma \ref{lem:nullability}} \pflabel{deriveeps}
          \item{$\mdata \in \denotetconfulle{\motcon{}'}{\menv}$} \BY{IH, \pfref{deriveeps}}
        \end{pfsteps*}
        To show $\setbuild{\mtrace}{\mdata\mtrace \in \denotetconfulle{\motcon}{\menv}} = \emptyset$, we suppose $\mtrace \in \semneg{\denotetconfulle{\motcon{}'}{\menv}}$ and show $\mtrace \nequiv \mdata\mtrace'$:
        \begin{byCases}
          \case{\mtrace \equiv \mdata\mtrace'}{
            Since $\mdata \in \denotetconfulle{\motcon{}'}{\menv}$, by definition of $\neg$, contradiction.}
          \otherwise{$\mtrace$ not prefixed by $\mdata$}
        \end{byCases}}
      \case{H : \nullable(\derivee{\mdata}{\motcon{}'}{\menv}) = \bot}{
\newcommand{\lhs}{\semneg{\setbuild{\mtrace}{\mdata\mtrace \in \denotetconfulle{\motcon{}'}{\menv}}}}
\newcommand{\rhs}{\semneg{\denotetconfulle{\motcon{}'}{\menv}}}
        \begin{pfsteps*}
          \item{$\epsilon \notin \denotetconfull{\derivee{\mdata}{\motcon{}'}{\menv}}$}
             \BY{lemma \ref{lem:nullability}} \pflabel{derivenoeps}
          \item{$\setbuild{\mtrace}{\mdata\mtrace \in \denotetconfulle{\motcon{}'}{\menv}} = \denotetconfull{\derivee{\mdata}{\motcon{}'}{\menv}}$} \BY{IH} \pflabel{IH}
          \item{$\mdata \notin \denotetconfulle{\motcon{}'}{\menv}$} \BY{\pfref{derivenoeps}, \pfref{IH}}
          \item{Goal is $\lhs = \setbuild{\mtrace}{\mevent\mtrace \in \rhs}$} \BY{computation}
        \end{pfsteps*}
        We prove this goal by bi-containment:
        \begin{byCases}
          \case{\mathit{Hs} : \mtrace \in \lhs}{
            \begin{pfsteps*}
              \item{$\forall \mtrace' \in \setbuild{\mtrace}{\mevent\mtrace \in \denotetconfulle{\motcon{}'}{\menv}}\setminus\setof{\epsilon}. \mtrace' \nleq \mtrace$}
                  \BY{$\mathit{Hs}$ and inversion} \pflabel{Hinv}
               \item{Suppose $\mtrace' \in \denotetconfulle{\motcon{}'}{\menv}$} \pflabel{let}
               \item{$\mtrace' \nleq \mevent\mtrace$}
                 \BY{\pfref{Hinv}, \pfref{let}, prefix cancellation} \pflabel{concl}
               \item{$\mtrace \in \rhs$} \BY{\pfref{concl}}
            \end{pfsteps*}}
          \case{\mathit{Hs} : \mtrace \in \rhs}{
            \begin{pfsteps*}
              \item{$\forall \mtrace' \in \denotetconfulle{\motcon{}'}{\menv}\setminus\setof{\epsilon}. \mtrace' \nleq \mevent\mtrace$}
                \BY{$\mathit{Hs}$, inversion} \pflabel{Hsinv1}
              \item{Suppose $\mtrace' \in \setbuild{\mtrace}{\mevent\mtrace \in \denotetconfulle{\motcon{}'}{\menv}} \setminus\setof{\epsilon}$} \pflabel{let}
              \item{$\mevent\mtrace' \in \denotetconfulle{\motcon{}'}{\menv}$} \BY{\pfref{let}} \pflabel{in}
              \item{$\mtrace' \nleq \mtrace$} \BY{\pfref{Hsinv1}, \pfref{in}, prefix cancellation} \pflabel{concl}
              \item{$\mtrace \in \lhs$} \BY{\pfref{concl}}
            \end{pfsteps*}}
        \end{byCases}}
    \end{byCases}}
\end{byCases}
%z

An interesting corollary relating paths to repeated derivation:
\begin{corollary}
  $\mtrace \in \denotetcon{\mtcon} \iff \nullable(\derive{\mtrace}{\mtcon}) = \epsilon$
\end{corollary}
% %
% Depends on
% \begin{lemma}[Flat emptiness]
%   $\flatempty(\mtcon) \implies \denotetconfull{\mtcon} = \emptyset$
% \end{lemma}
% where $\flatempty$ and $\flatempty_\menv$ are defined as (where $[_\menv]$ denotes the definition applies to both)
% \begin{align*}
%   \flatempty[_\menv](T) &= \bot \quad\text{if } T\equiv\epsilon,\, T\equiv(\stmany{\mtcon}),\,\text{or }T\equiv\stnot{\mtcon} \\
%   \flatempty[_\menv](\stbind{\mevent}{\mtcon}) &= \flatempty[_\menv](\mevent) \\
%   \flatempty[_\menv](\stOr{\isset{\mtcon}}) &= \bigwedge \flatempty[_\menv](\mtcon)\ldots \\
%   \flatempty[_\menv](\stAnd{\isset{\mtcon}}) &= \bigvee \flatempty[_\menv](\mtcon)\ldots \\
%   \flatempty[_\menv](\stseq{\mtcon_0}{\mtcon_1}) &= \flatempty[_\menv](\mtcon_0) \vee \flatempty[_\menv](\mtcon_1) \\
%   \flatempty(\mtcon,\menv) &= \flatempty_\menv(\mtcon) \\[2pt]
%   \flatempty[_\menv](\snone) &= \top \\
%   \flatempty_\menv(\mvar) &= (\menv(\mvar) \overset{?}{=} \emptyset) \\
%   \flatempty[_\menv](\snegpat{\mvpat}) &= \top \quad\text{if } \mvpat\equiv\sany,\, \mvpat\equiv\sbind{\mvar}, \text{ or } \mvpat\equiv\snegpat{\mvpat'} \text{ and } \flatempty[_\menv](\mvpat') \\
%   \flatempty[_\menv](\mvpat) &= \bot \quad\text{otherwise}
% \end{align*}
% %
% Our implementation reduces temporal contracts at construction time, so that the following invariant holds of represented contracts:
% \begin{equation*}
%  \flatempty(\mtcon) \iff \mtcon = \bot
% \end{equation*}


%%
\section{Semantics}\label{sec:technical}

Now that we have a clear view of the behavior of temporal contracts, we nail down a formal semantics that we use to prove the correctness of our monitoring system.
%
The semantics we present is in the style of Felleisen's reduction semantics~\citep{ianjohnson:Felleisen:2009:SEP:1795772}, which can be systematically transformed into an abstract machine in the form presented in~\citet{dvanhorn:VanHorn2010Abstracting}.
%%

%FIXME: This is from the LTL days...
%%
Wrapping monitors around values associates entirely fresh NFA states.
%
The names given to bindings are really associated with the bindings.
%
We have to give up some precision for decidability --- we use a monovariant allocation scheme for monitor allocation.
%
This scheme allows us to lift all textual temporal contracts to the top level and check individually.
%
A polyvariant scheme would require either knowing how many abstract bindings are allocated during execution, or an exhaustive enumeration of possible bindings as propositions to check.

%%
\subsection{Syntax}
\FloatBarrier

\begin{figure}
\begin{align*}
\mexp \in \Expr &::=
      \slit{\mconstant}
 \alt \svar{\mvar}
 \alt \sapp{\mexp}{\mexp}
 \alt \slam{\mvar}{\mexp}
 \alt \sif{\mexp}{\mexp}{\mexp} \\
&\alt \sSMon{\mmlab}{\mmlab}{\mlab}{\mtimeline}{\mscon}{\mexp}
 \alt \sTMon{\mmlab}{\mmlab}{\mlab}{\mscon}{\mtcon}{\mexp} \\
\mconstant \in \Constant &=
 -1 \alt 0 \alt 1 \alt \ldots \alt + \alt - \alt \ldots \\
&\alt \strue \alt \sfalse \alt \vee \alt \wedge \alt \ldots \\
&\alt \sunit \alt \scons \alt \scar \alt \scdr \alt \ldots \\
\mvar \in \Var &\quad\text{ an infinite set} \\
\mmlab \in \Label \alt \toplevel &\quad\text{ an infinite set, where $\toplevel$ is the top-level}
\end{align*}
\caption{Syntax}
\label{fig:syntax}
\end{figure}

%
\autoref{fig:syntax} presents the core, desugared syntax of our language model.
%
Note that the surface syntax is better represented by the pseudocode examples given in \cref{sec:sort,sec:file,sec:tcp}; for example, a user would not be able to use $\mathtt{smon}$ or $\mathtt{tmon}$ in expressions.
%
While the syntax is largely conventional, the nature of monitoring and labelling are worth discussion.

%
A structural monitor $\sSMon{k}{\ell}{j}{\mtimeline}{\mscon}{\mexp}$ is like a traditional monitor \cite{ianjohnson:dthf:complete}: the structural contract is given by $\mscon$ and the contract parties are given by labels: $k$ is the server, $l$ is the client, and $j$ is the contract itself.
%
However, our structural monitors additionally feature a timeline ($\mtimeline$), used to temporally reason over values as they flow through contract boundaries (cf. \autoref{sec:temporal-semantics}).
%

%
A temporal monitor $\sTMon{k}{\ell}{j}{\mscon}{\mtcon}{\mexp}$ has a structural component as well ($\mscon$), but importantly it also considers a temporal contract given by $\mtcon$.
%
These kinds of monitors are only introduced at the top level, and thus need not be qualified over a timeline, as it is effectively $\toplevel$.
%
At this point we would like to note that our usage of labels differs slightly from existing literature \cite{ianjohnson:dthf:complete}: we have a notion of the ``top level label'', denoted by $\toplevel$; this is similar to \citeauthor{ianjohnson:dthf:complete}'s use of $\ell_0$.
%%

%%
\subsection{Value Space and Evaluation Contexts}

\begin{figure}
\begin{align*}
\mexp &::= \ldots
 \alt \sblame{\mmlab}{\mmlab}
 \alt \schk*{\mmlab}{\mmlab}{\mtimeline}{\mexp}{\mval}
 \alt \sown{\mexp}{\mmlab} \\
&\alt \sret{\mmlab}{\mtimeline}{\mexp}
\\
\mval \in \Value &::=
      \mconstant
 \alt \vcons{\mval}{\mval}
 \alt \sown{\mval}{\mmlab}
 \alt \maddr \\
&\alt \bclos{\mmlab}{\mscon}{\mmlab}{\mmlab}{\mtimeline}{\mscon}{\mmlab}{\maddr}
\\
\maddr \in \Addr &\quad\text{an infinite set} \\
\mstimeline \in \Timeline &::= \maddr \alt \toplevel
\\
\mscon &::= \ldots
\alt \sown*{\sflat{\mexp}}{\mmlab}
\\
\sOwn{\mexp}{\mmlab} = \sown{ \ldots \sown{\mexp}{\mmlab} \ldots }{\mmlab}
&\quad\text{where $\forall$ labels $k$ and terms $\mexp^\prime$, $\mexp \neq \sown{\mexp^\prime}{k}$}
\\
\sown*{\sflat{\mexp}}{\mmlab} &\quad\text{denotes the set of responsible parties $\overline{\mmlab}$}
\end{align*}
\caption{Value space}
\label{fig:values}
\end{figure}
%%

%
\autoref{fig:values} presents the value space for our language model.
%
For evaluation purposes, we extend our definition of expressions ($\mexp$) to include intermediate terms for contract checking.
%
Expressions of the form $\sret{\mmlab}{\mtimeline}{\mexp}$ denote the production of an return event $\sretev{\mmlab}{\mexp}$, which may be rejected by the temporal monitors in place.
%
The $\sblame{\ell}{j}$ expression denotes a contract failure (of the contract $j$), which is a fatal error which blames $\ell$.
%
An $\schk*{\ell}{j}{\mtimeline}{\mexp}{\mval}$ expression denotes an obligation to check the contract of $j$ given by $\mexp$ against the value $\mval$, blaming $\ell$ if the contract is violated.
%
The timeline component of this expression ($\mtimeline$) is used to pass along the timeline to any monitors inside of $\mexp$ which might need to reason over it; given complete monitoring \cite{ianjohnson:dthf:complete}, the contract itself is still subject and a party to the program's contracts.
%
The expression $\sown{\mexp}{\mmlab}$ denotes that the $\mexp$ is \emph{owned} \cite{ianjohnson:dthf:complete} by $\mmlab$; this is used to keep track of value migration for the purposes of ensuring \emph{complete monitoring}, using the concept of \emph{single ownership}.
%
We also extend our definition of structural contracts to include ownership of flat contracts ($\sown*{\sflat{\mexp}}{\mmlab}$).

%
\NB We provide a brief explanation of some notation at the bottom of the figure.
%
Of particular note, we use $\sOwn{\mexp}{\mmlab}$ to denote that $\mexp$ may have no ownership annotations, but if it has one then it is $\ell$ for all such annotations.
%
Our concept and usage of ownership is identical to that of \citet{ianjohnson:dthf:complete}, which we recommend reading to fully understand this topic; as it is orthogonal to our work, we do not delve into it in much detail in this paper.
%

%
Values ($\mval$) are conventional except for ownership ($\sown{\mval}{\mmlab}$) and blessed closures ($\bclos{\mmlab}{\mscon}{\mmlab}{\mmlab}{\mtimeline}{\mscon}{\mmlab}{\maddr}$).
%
Though we've discussed timelines ($\mtimeline$) (cf. \autoref{sec:temporal-semantics}), we show here that our representation for them is either as an address ($\maddr$) or the top level ($\toplevel$).
%

%
Our evaluation contexts in \autoref{fig:ctx} follow straightforwardly and are ownership-sensitive \cite{ianjohnson:dthf:complete}.
%
A machine state ($\mstate$) is composed of the evaluation context ($\mctx{\mmlab}$), store ($\msto$), and store of PMSM states ($\mTMons$).
%
The former maps addresses to sets of values and the latter maps timelines to PMSMs.
%

%%
\begin{figure}
\begin{align*}
\mctx{\mmlab} \in \ECtx &::=
      \sapp{\mctx{\mmlab}}{\mexp}
 \alt \sapp{\mval}{\mctx{\mmlab}}
 \alt \sif{\mctx{\mmlab}}{\mexp}{\mexp} \\
&\alt \vcons{\mctx{\mmlab}}{\mexp}
 \alt \vcons{\mval}{\mctx{\mmlab}} \\
&\alt \sSMon{\ell}{k}{j}{\mtimeline}{\mscon}{\mctx*}
 \alt \sSMon{\ell^\prime}{k}{j}{\mtimeline}{\mscon}{\mctx{\ell}} \\
&\alt \sTMon{\ell}{k}{j}{\mscon}{\mtcon}{\mctx*}
 \alt \sTMon{\ell^\prime}{k}{j}{\mscon}{\mtcon}{\mctx{\ell}} \\
&\alt \schk*{k}{\ell}{\mtimeline}{\mctx*}{\mval}
 \alt \schk*{k}{\ell^\prime}{\mtimeline}{\mctx{\ell}}{\mval} \\
&\alt \sret{\ell}{\mtimeline}{\mctx*}
 \alt \sret{\ell^\prime}{\mtimeline}{\mctx{\ell}} \\
&\alt \sown{\mctx*}{\ell}
 \alt \sOwn{\mctx{\ell}}{\ell^\prime}
\\
\mctx* \in \ECtx* &::=
      \hole
 \alt \sapp{\mctx*}{\mexp}
 \alt \sapp{\mval}{\mctx*}
 \alt \sif{\mctx*}{\mexp}{\mexp} \\
&\alt \vcons{\mctx*}{\mexp}
 \alt \vcons{\mval}{\mctx*} \\
&\alt \scall{\mmlab}{\mtimeline}{\mctx*}
 \alt \sret{\mmlab}{\mtimeline}{\mctx*} \\
\ECtx* &\subset \ECtx^\mmlab
\\
\mstate \in \State &= \chevron{\mctx{\mmlab}, \msto, \mTMons} \quad\text{machine state}
\\
\msto \in \Store &= \Addr \parto \wp(\Value)
\\
\mTMons \in \TMons &= \Timeline \parto \wp(\PMSM)
\end{align*}
\caption{Evaluation contexts}
\label{fig:ctx}
\end{figure}
%%

%%
\subsection{Reduction} \label{sec:reduction}

% TODO:
% * Should addresses be owned? Does this imply stores need to be owned...?
% * Should conses be owned?
\newcommand*{\namefmt}[1]{\textit{\textsc{#1}}}
\begin{figure*}
\newcommand*{\update}[3]{#1^\prime = #1[#2 \mapsto #1(#2)\sqcup#3]}
\newcommand*{\name}[1]{&\text{[\namefmt{#1}]}}
\newcommand*{\where}{\text{where }}
\newcommand*{\cwhere}{\phantom{\where}}
\centering
$\begin{array}{@{}l @{\ }c@{\ } l@{} r}
\mstate \chevron{\mctx{\mmlab}[\ldots], \msto, \mTMons}
&\machstep&
%TODO: Can mmlab change? For now, yup.
\mstate^\prime \chevron{\mctx{\mmlab^\prime}[\ldots], \msto^\prime, \mTMons^\prime}
\name{rule-name}

\\ \hline % If true
\sif{ \sOwn{\strue}{\mmlab} }{\mexp_1}{\mexp_2}
&\machstep&
\mexp_1
\name{if-true}

\\ % If false
\sif{ \sOwn{\sfalse}{\mmlab} }{\mexp_1}{\mexp_2}
&\machstep&
\mexp_2
\name{if-false}

\\ % Apply
\sapp{ \sOwn{\slam{\mvar}{\mexp}}{\mmlab} }{ \sOwn{\mval}{\mmlab} }
&\machstep&
\sown{ \sapp{\{ \sown{\mval}{\mmlab}/\mvar \}}{\mexp} }{\mmlab}
\name{apply}

\\ % Temporal monitor
\sTMon{k}{\mmlab}{j}{\mscon}{\mtcon}{\mval}
&\machstep&
\sSMon{k}{\mmlab}{j}{\mtimeline}{\mscon}{\mval}
\name{tmon} \\
&&\where \update{\mTMons}{\mtimeline}{\smachine} \\
&&\cwhere \smachine = \compile{\mtcon}{\mtimeline} \ne \bot \\
&&\cwhere \mtimeline = \salloc{\mstate} \\
&&\text{given } \mmlab = \toplevel

\\ % Temporal monitor compilation fail
\ditto
&\machstep&
\sblame{j}{j}
\name{tmon-fail} \\
&&\where \bot = \compile{\mtcon}{\mtimeline} \\
&&\cwhere \mtimeline = \salloc{\mstate} \\
&&\text{given } \mmlab = \toplevel 

\\ % Flat monitor
\sSMon{k}{\mmlab}{j}{\mtimeline}{ \sown*{\sflat{\mexp}}{\mmlab^{\prime\prime}} }{ \sOwn{\mconstant}{\mmlab^{\prime\prime}} }
&\machstep&
\schk*{k}{j}{\mtimeline}{ \sapp{\mexp}{\mconstant} }{\mconstant}
\name{smon-flat}

\\ % Cons monitor
\sSMon{k}{\mmlab}{j}{\mtimeline}{ \sconsc{\mscon_A}{\mscon_D} }{ \sOwn{\vcons{\mval_A}{\mval_D}}{\mmlab} }
&\machstep&
\sown{\vcons{ \sSMon{k}{\mmlab}{j}{\mtimeline}{\mscon_A}{\mval_A} }{ \sSMon{k}{\mmlab}{j}{\mtimeline}{\mscon_D}{\mval_D} }}{\mmlab}
\name{smon-cons}

\\ % Arrow monitor
\sSMon{k}{\mmlab}{j}{\mtimeline}{ \sarr{\mmlab_\lambda}{\mscon_D}{\mscon_R} }{\mval}
&\machstep&
\sown{\bclos{\mmlab_\lambda}{\mscon_D}{k}{j}{\mtimeline}{\mscon_R}{\mmlab}{\maddr}}{\mmlab}
\name{smon-arrow} \\
&&\where \maddr = \salloc{\mstate} \\
&&\cwhere \update{\msto}{\maddr}{\mval} \\
&&\text{given } \mval = \slam{\mvar}{\mexp}

\\ % Arrow monitor fail
\ditto
&\machstep&
\sblame{k}{j} \qquad\text{given } \mval \neq \slam{\mvar}{\mexp}
\name{smon-arrow-fail}

\\ % Check true
\schk{k}{j}{ \sOwn{\strue}{j} }{\mval}
&\machstep&
\mval
\name{chk-true}

\\ % Check false
\schk{k}{j}{ \sOwn{\sfalse}{j} }{\mval}
&\machstep&
\sblame{k}{j}
\name{chk-false}

\\ % Blessed application (call)
\sapp{ \sOwn{\bclos{\mmlab_\lambda}{\mscon_D}{k}{j}{\mtimeline}{\mscon_R}{\mmlab^{\prime\prime}}{\maddr}}{\mmlab} }{ \sOwn{\mval}{\mmlab} }
&\machstep&
%TODO: Should the timeline be different?
\mexp_{ret}
\name{call} \\
&&\where \mexp_{ret} =
\sret{\mmlab_\lambda}{\mtimeline}{ \sSMon{k}{\mmlab^{\prime\prime}}{j}{\mtimeline}{\mscon_R}{\mexp_{call}} } \\
&&\cwhere \mexp_{call} = \sapp{\mval_\lambda}{\sSMon{\mmlab^{\prime\prime}}{k}{j}{\mtimeline}{\mscon_D}{\mval}} \\
&&\cwhere \mval_\lambda = \slam{\mvar}{\mexp} \in \msto(\maddr) \\
&&\cwhere \mTMons^\prime \in \sstep{\mTMons}{\mtimeline}{\scallev{\mmlab_\lambda}{\mval}} \\
&&\cwhere \mmlab^\prime = \mmlab_\lambda

\\ % Call fail
\ditto
&\machstep&
\sblame{\mmlab}{j}
\qquad\text{given } \bot \in \sstep{\mTMons}{\mtimeline}{\scallev{\mmlab_\lambda}{\mval}}
\name{call-fail}

\\ % Return
\sret{\mmlab_{ret}}{\mtimeline}{ \sOwn{\mval}{\mmlab} }
&\machstep&
\sown{\mval}{\mmlab_{ret}}
\name{return} \\
&&\where \mTMons^\prime \in \sstep{\mTMons}{\mtimeline}{\sretev{\mmlab}{\mval}} \\
&&\cwhere \mmlab^\prime = \mmlab_{ret}

\\ % Return fail
\ditto
&\machstep&
\sblame{\mmlab}{\j}
\qquad\text{given } \bot \in \sstep{\mTMons}{\mtimeline}{\sretev{\mmlab}{\mval}}
\name{return-fail}

\\ \hline % Blame
\mctx{\mmlab}[\sblame{k}{j}]
&\machstep&
\sblame{k}{j}
\name{blame}

\end{array}$
\caption{Reduction rules}
\label{fig:reduction}
\end{figure*}

%
Our reduction relation given in \autoref{fig:reduction} makes use of the the single owner policy by reducing redexes only if the label of the hole matches the owner of the pieces of the redex.
%
We elide the rules for the primitive functions as they are standard and straightforward.
%
Note, however, that all of our primitives (save for $\scons$) take only addresses as arguments and each rule dereferences the addresses in the store.
%

%
The \namefmt{tmon} rule compiles the temporal contract inside of the $\mathtt{tmon}$ ($\mtcon$) and stores it inside the PMSM store ($\mTMons$), associated with a fresh timeline ($\mtimeline$) and joined with any extant PMSMs associated with the timeline ($\mTMons(\mtimeline)$).
%
The structural component of the contract ($\mscon$) is then used to rewrap the monitored value ($\mval$) inside of a structural monitor ($\mathtt{smon}$) associated with the timeline.
%
Temporal monitors only exist at the top level, and so the context's ownership label \emph{must} be $\toplevel$.
%
Should the compilation of the temporal contract fail --- which would happen \eg if the PMSM is initially not accepting --- then the reduction would instead blame the contract in \namefmt{tmon-fail}.
%

%
The \namefmt{smon} rules handle the reduction of structural monitors.
%
Flat contracts (in \namefmt{smon-flat}) on constants are easily translated into a check on the contract.
%
Cons-list contracts (in \namefmt{smon-cons}) are also simple given our evaluation contexts: they are translated into an owned $\mathtt{cons}$, with each component appropriately wrapped by its respective structural contract: $\mscon_A$ for the \texttt{car} and $\mscon_D$ for the \texttt{cdr}.
%
%
Arrow contracts (in \namefmt{smon-arrow}), however, are more nuanced.
%
As we our temporal contracts reason over the calls and returns of functions, we cannot simply translate this into \eg\ $\slam{\mvar}{ \sSMon{k}{\mmlab}{j}{\mtimeline}{\mscon_R}{ \sapp{\mval}{\sSMon{\mmlab}{k}{j}{\mtimeline}{\mscon_D}{\mvar} }}}$.
%
Instead, we construct a blessed closure with all of the relevant information, and store the monitored value in the store at a fresh address (\maddr) and joined with any extant values at that same address ($\msto(\maddr)$).
%
We will perform the relevant checking upon application of the blessed closure (in \namefmt{call}).
%
Nevertheless, if the monitored value is not a function, then we fail and blame in a straightforward manner (in \namefmt{smon-arrow-fail}).
%

%
Blessed application (in \namefmt{call}) is quite involved.
%
As we discussed regarding arrow contracts, we cannot simply check the structural domain and range.
%
Instead, we must first check with any relevant PMSMs (through the $\delta{}step$ function) whether the call event which represents this application is permissible; if not, then the relevant rule is instead \namefmt{call-fail}, which blames the caller.
%
Assuming that the call is temporally permissible, then we construct an application expression ($\mexp_{call}$) which wraps the argument in a structural domain monitor.
%
In turn, we wrap \emph{that} expression in a structural range monitor, which itself is wrapped in a return event production (which will be handled in \namefmt{return}/\namefmt{return-fail}).
%
This formulation guarantees that a call event is produced \emph{before} the call actually occurs, which is obviously critical to the correctness of these temporal contracts.
%
Finally, note that the evaluation context's owner becomes the callee after this step (in line with the call).
%

%
To complete our modeling of temporal monitors, we must also represent return events; we do so in \namefmt{return}.
%
Once we have reduced an application to a value, we must note that the function which was called will now return.
%
Again, we check with any relevant PMSMs---using the $\delta{}step$ function---whether the return event is permissible; if not, then the relevant rule is instead \namefmt{return-fail}, which blames the callee.
%
Otherwise, we reduce to the return value.
%
This formulation guarantees that a return event is produced \emph{before} the return actually completes, again critical for correctness.
%
After this reduction step, the evaluation context's owner becomes that of the original caller again (in line with the return).
%
Note also that the returning context takes ownership of the returned value as well.
%

%
\paragraph{Complete Monitoring}
Our dynamic semantics preserves complete monitoring given the introduction of temporal contracts.
%
This is guaranteed via our extension of the use of the single-ownership principle to our use of temporal contracts.
%
We are also mindful of the change of context ownership, given that we consider temporally-tracked functions to be modules and thus temporally-sensitive calls and returns become module boundaries that must be correctly monitored.
%
%%

\section{Abstract semantics}\label{sec:abstract-semantics}

We use the AAM approach to soundly approximate our concrete semantics.
%
This first requires a CESK-like machine to transform, but there are known ways to systematically transform Felleisen-style reduction semantics into corresponding CESK machines~\citep{dvanhorn:Danvy-Nielsen:RS-04-26}.
%
The important aspect of the abstraction process is not this transformation, but the fact that the space of values becomes \emph{finite}, meaning the space of temporal contract derivatives is finite (proven via co-induction).
%
The additional important step is how to handle the uncertainy allowed in the definition of $\matches$ in \autoref{sec:temporal-semantics}: $\partial$'s codomain becomes $\wp(\TContract \times \Valuation)$, where we let $\motcon_\bot = \bot_\bot$.
%
The important cases are defined in \autoref{fig:abstract-tcon}.
%
Here $\Valuation$ stands for how sure we are to \emph{not} blame.
\begin{figure}
  \begin{tabular}{lr}
    $\begin{array}{rl}
    \derivee{\mdata}{\mevent}{\menv} &= 
      \setbuild{\epsilon_\mvaluation}{\menv'_\mvaluation \in \matches(\mevent, \mdata, \menv)} \\
%
    \derivee{\mdata}{\stbind{\mevent}{\motcon}}{\menv} &=
      \setbuild{(\motcon,\menv)_\mvaluation}{\menv_\mvaluation \in \matches(\menv,\mdata,\menv)} \\
%
    \derivee{\mdata}{\stOr{\isset{\motcon}}}{\menv} &= \bigoplus \derivee{\mdata}{\motcon}{\menv}\ldots \\
%
    \derivee{\mdata}{\stAnd{\isset{\motcon}}}{\menv} &= \bigotimes \derivee{\mdata}{\motcon}{\menv}\ldots \\
%
    \mtcons \otimes \mtcons' &= \setbuild{\stAnd{\setof{\mtcon,\mtcon'}}_{\mvaluation \wedge \mvaluation'}}{\mtcon_\mvaluation \in \mtcons, \mtcon'_{\mvaluation'} \in \mtcons'}
  \end{array}$
  &
  $\begin{array}{rl}
    \derivee{\mdata}{\stnot{\motcon}}{\menv} &=
     \left\{
       \begin{array}{ll}
         \begin{array}{l}
           \mathit{case}\ \nullable(\mtcon) \\
           \alt \epsilon \Rightarrow \stnot{\mtcon}_{\neg \mvaluation} \\
           \alt \bot \Rightarrow \stnot{\mtcon}_\must
         \end{array}
         &: \mtcon_\mvaluation \in \derivee{\mdata}{\motcon}{\menv}
       \end{array}
       \right\}
\\
    \mtcons \oplus \emptyset &= \emptyset \oplus \mtcons = \mtcons \\
%
    \mtcons \oplus \mtcons' &= \setbuild{\stOr{\setof{\mtcon,\mtcon'}}_{\mvaluation \vee \mvaluation'}}{\mtcon_\mvaluation \in \mtcons, \mtcon'_{\mvaluation'} \in \mtcons'} \\
%
  \end{array}$
  \end{tabular}
  \caption{Uncertainty in derivatives}
  \label{fig:abstract-tcon}
\end{figure}

%
In our semantics with just closures, conses, booleans and integers, we consider:
\begin{itemize}
 \item syntactically equal closures and conses to be $\may$ equal;
 \item equal booleans and integers to be $\must$ equal;
 \item integer comparisons with the abstract $\mathbf{Int}$ element to be $\may$ equal (assuming a simple flat abstraction of the integers);
 \item and all other possibilities to be never equal, $\bot$.
\end{itemize}
%
Any time a temporal contract is in a \may{} state, we blame conservatively.
%
Since timelines can be abstractly allocated more than once, each time we send an action to the monitor, we have to both step and not step the contract, which can kill precision and performance.
%
On top a basic abstraction from AAM, we evaluate the following techniques to improve the precision of the analysis:
\begin{itemize}
\item{($\mu$) use abstract counting ~\citep{dvanhorn:Might:2006:GammaCFA} to improve equality checking of allocated data and provide a non-trivial $\gamma_1?$;}
\item{($\Gamma$) use abstract garbage collection~\citep{dvanhorn:Might:2006:GammaCFA} to remove dead temporal monitors;}
\item{($\Gamma_\tau$) use abstract garbage collection to remove bindings to dead values in live temporal monitors.
%
This means we reduce $\mtcon,\menv$ to $\mtcon,\menv'$, where $\menv'$ restricts variables to the values in $\menv$ that touch only reachable addresses.
%
The concrete semantics would operate the same way regardless of GC, but this is akin to using weak boxes to maintain values in temporal contract environments;}
\item{($\Xi$) separate continuation management to a pushdown abstraction~\citep{dvanhorn:Vardoulakis2011CFA2}.}%; or}
%\item{use a polyvariant allocation strategy such as $m$-CFA ~\citep{dvanhorn:Might2010Resolving} or polymorphic splitting~\citep{dvanhorn:wright-jagannathan-toplas98}.}
\end{itemize}
%%

\section{Evaluation}\label{sec:evaluation}
%%

\begin{figure}
  \input{bench-overview}
  \caption{Benchmark results. Numbers are run-time (sec) and $\frac{\text{spurious blames}}{\text{possible blames}}$}
  \label{fig:evaluation}
\end{figure}
%%
We built temporal higher-order contracts into our existing framework for analyzing a subset of Scheme~\citep{ianjohnson:oaam:icfp2013} \footnote{Model and benchmarks available online \url{http://github.com/dvanhorn/oaam/tree/thocon}}.
%
Our benchmarks are a collection of examples from \dfm's paper and implementation:
\paragraph{{\tt sort1}:}{\dfm's motivating example of a non-re-entrant sort function where the given comparison function may not be called after sort returns\footnote{If sort stashes the comparator in a global and we call it afterwards, \dfm's implementation does not raise blame, whereas ours does.}.
%
The structural contract for {\tt sort1} is weak: {\tt (listof integer?) -> any/c}.
}
\paragraph{{\tt sort2}:}{like {\tt sort1}, but with a stronger structural contract that will confuse a regular analysis due to merged return flows to the flat contract: {\tt (listof integer?) -> (listof integer?)}.}
\paragraph{{\tt sort3}:}{like {\tt sort2}, but call sort more than once (mapped over a list of lists), in order to confuse non-$\Gamma_\tau$ analyses. If a dead, wrapped $\mathit{cmp}$ is not collected, then additional calls to $\sortid$ will conflate the wrappings and conservatively blame.}
\paragraph{{\tt malloc}:}{A pair of two functions, malloc and free, where malloc returns an ever-growing counter and free does nothing. They are contracted so that free may not be called with addresses that were previously freed and not subsequently returned by malloc.}
\paragraph{{\tt file}:}{A function, {\tt open}, which given a file path, produces a list of read, write and close functions. They are contracted such that none of these functions may be called after close. Using this interface, we copy the contents of files in one list and write them to the files in the other.}
%\paragraph{{\tt FTP}:}{An FTP server built on TCP. \textbf{TODO: Alex}}
%%

%%
The analysis results and run times for our benchmark programs are in \autoref{fig:evaluation}, categorized by the additional machinery as labeled in the previous section ($m$ means used more than 2GiB limit).
%
Analyses without garbage collection used the global store widening to accelerate convergence.
%%

%%
The minor variations to the sort example motivated our additional analytical machinery.
%, \textbf{TO BE SEEN} and proved strong enough to verify the more complicated example of the FTP interaction.
%
No technique we employed could fully verify the {\tt malloc} example, since the protocol depends on being able to prove equality between values that are mutated and grown without bound; our abstraction for numbers is simple, so the mutated counter immediately jumps to \textbf{Int}$\top$, where comparisons conservatively both succeed and fail.
%
The {\tt file} example could be verified had we used a polyvariant allocation strategy, since it involved opening two files at the same time with the same handle allocation code.
%%

%%
Notice that without a pushdown abstraction, abstract garbage collection searched a state space orders of magnitude more than with.
%
The $\Gamma$ and $\Gamma_\tau$ implementations do not use a pushdown abstraction, but they still can verify {\tt sort2}, which was meant to confound non-pushdown analyses.
%
The reason for this is that in AAM, continuations are allocated on the heap, so GC can reclaim them for better return flow prediction; recursive calls do not get the same precision boost.
%
The topic of making abstract GC more performant and feasible is an active area of research, most recently extended to pushdown abstractions \citep{dvanhorn:Earl2012Introspective}.
%
Although greater precision can increase the possible state space, clever abstractions combined with typical program structure can actually reduce the explored state space, as is evident by the dearth of visited blame sites in our most precise implementation.

%%
\section{Related Work}\label{sec:related}
%%

%%
Model-checking and contract verification (proving functional correctness) are huge fields and it is important to view our work in the greater context of these worlds of research.
%
The main separating factor between this work and the model-checking literature is that temporal property verification is \emph{extra-linguistic}, meaning there is no mechanism in the object language that could monitor for the properties.
%
Furthermore, work that focuses on the linguistic mechanism does not also focus on the verification mechanism.
%%

%%
\paragraph{Runtime monitoring:}
Monitoring sequences of actions at runtime is a mature and active area of research.
%
This area has similarities to temporal contracts due to the use of runtime monitoring and of languages for describing execution traces, but nothing in the area has the concept of blame or has a static story.
%
The notion of an action is reminiscent of aspect-oriented programming's notion of a \emph{join-point}, and thus we see several systems built on AspectJ \citep{ianjohnson:aspectj} that offer a domain-specific language for running \emph{advice} when the action trace matches a specified pattern, \eg, Tracematches \citep{ianjohnson:Allan05addingtrace} and J-LO ~\citep{ianjohnson:jlo}.
%
Tracematches use a language similar to temporal contracts but do not support negation; they also have a different purpose: execute advice at more specific times based on the program history, and not to offer a high-level specification system with blame that contracts provide.
%
J-LO on the other hand offers a monitoring system based on LTL propositions with binding constructs that tracks the satisfaction of the LTL proposition with the assumption that future portions of the proposition hold.
%
If the \emph{now} portion of the proposition fails to hold, the monitor raises a failure (it does not blame).
%
The language for temporal contracts is an extension of regular expressions with back references, as many desirable properties are difficult to express in LTL with back-references (DLTL ~\cite{ianjohnson:jlo})\footnote{A sentiment expressed by DLTL's inventor~\citep{boddenadmission}}.
%
Temporal contracts and DLTL can talk about value flow and use over time via binding in the specification --- this is not something that LTL can locally express, and DLTL does not currently have any model checking tool-support.
%
J-LO's goal is closer to temporal contracts, but its language is not; conversely, tracematches match the language and not the goal.
%
In both cases, the only static analysis is on the specification itself in order to improve runtime performance, and not on the monitored program's adherence to the specification.
%
Both systems are also tied to Java's class structure, so they cannot express properties of higher-order behavior or refinements on values.

\paragraph{Higher-order model-checking:}
Java and C++ both have several high-quality model-checking tools \citep{ianjohnson:bandera, ianjohnson:java-pathfinder, ianjohnson:LLBMC}, some of which are bounded model-checkers; meaning they cannot fully verify temporal properties --- only present possible counter-examples.
%
Bandera~\citep{ianjohnson:bandera} is a collection of tools that uses static analysis techniques to extract a finite model from a Java program to feed to select back-end model-checkers.
%
Similar to our approach, Bandera employs flow analysis in order to produce compact models.
%
Unsimilarly, it does not synthesize checkers for runtime monitoring the expressed properties, nor does it natively support higher-order functions.
%
A complete separation of model generation and model-checking also degrades precision, since the more in-depth constraint solving typical model-checkers do can help prune the control-flow space; our approach is amenable to integrated constraint solving and is left to future work.
%%

%%
A technique that specifically targets higher-order languages, higher-order recursion schemes (HORS)~\citep{ianjohnson:Knapik:2002:HPT:646794.704852}, is rooted in simply-typed, call-by-name lambda terms, but has model-checking solutions that have been extended to call-by-value ~\citep{ianjohnson:DBLP:journals/jacm/Kobayashi13} and untyped ~\citep{dvanhorn:Tsukada2010Untyped} languages, through heavy type-theoretic machinery.
%
Model-checking an untyped HORS is undecidable, and such model checkers make various approximations biased to soundly model-check programs in traditional type systems rather than traditional untyped languages.
%
Our technique is lighter weight and more transparently correct since it follows from a systematic transformation of a standard semantics.
%
Additionally, the AAM approach makes extensions to more complex language features straightforward, whereas in HORS one would need to CPS, double CPS, or perform a functional encoding of a new form of data; all of which impose additional proof obligations and points-of-failure for the analysis implementor.
%
Finally, HORS do not synthesize runtime monitors or have a notion of blame, unlike our system.
%%

%%
\paragraph{Behavioral contract verification:}
In the world of static sotware contract verification, there is more closely related work.
%
There have been many successful efforts in the realm of first-order contract verification~\citep{ianjohnson:fahndrich:contracts:2011,ianjohnson:vcc:2009}, but the techniques employed are inherently first-order: the only values are booleans.
%
\citet{ianjohnson:Flanagan:2006:HTC:1111037.1111059}'s notion of hybrid type checking is one way to state the problem: dynamic types are essentially flat contracts, and are treated as subtypes of anything during static checking.
%
If an external theorem prover can prove that the flat contracts always hold, the dynamic checks can be safely removed.
%
\citet{dvanhorn:Xu2012Hybrid} describes a higher-order contract verification system for OCaml by inlining all contract monitors and relying on a system of simplifications further enhanced by an SMT solver to optimize away dynamic checks.
%
\citet{dvanhorn:TobinHochstadt2012Higherorder} use AAM on a module semantics with higher-order contracts and is the most related to this work.
%
However, whereas their work focuses on a concrete semantics for handling unknown values and an external theorem prover to show contract containment, our work focuses on an orthogonal issue of temporal contract monitoring.
%
Our techniques should smoothly integrate with theirs when considering partial programs, and is left to future work (\cref{sec:conclusion}).
%%

%%
\section{Conclusion and future work} \label{sec:conclusion}
%%

%%
We demonstrated that a linguistic construction for monitoring temporal properties of higher-order programs (temporal higher-order contracts) can be transparently abstracted to provide a sound verification algorithm, almost for free, given the AAM approach.
%
Our preliminary evaluation provides evidence that this can be an effective approach to verifying programs with finite protocols, given the right combination of existing analysis machinery.
%
The way forward is clear: build a quality temporal monitoring library and contract a large project to find weaknesses in the language of temporal contracts, and finally evaluate the verification algorithm on this large example.
%
We can identify the following dimensions that can strengthen our work:
\begin{itemize}
\item{targeted blame assignment, with deontic logic \citep{ianjohnson:DBLP:conf/dagstuhl/MeyerWD98}:
%
currently, the party that generates the action that causes a contract to fail gets blamed, and there is no way to express that some action must happen before the end of the monitor's lifetime.
%
The action-emitting party might be innocent, just working at the behest of a different party that is obliged not to violate the contract.
%
Instead of attaching a temporal contract to a single structural contract with the {\tt tmon} form, we instead allow structural contracts to \emph{emit} temporal contracts as obligations of a given party, and additionally blame for unfinished obligations at monitor collection time, end of execution, or after a specified amount of time;
}
%
\item{calls and returns don't match in the temporal contracts themselves:
%
the actions are currently interpreted without a notion of matching, which could turn out to be too flat and limiting for proper specifications --- the implementation from \dfm{} allows labeling calls so that return actions can refer to the matching call.
%
It is unclear whether an analysis can precisely match calls and returns in both the contracts and control-flow, due to the several stacks that apparently entails, but at least a ``good enough'' approximation should exist;
}
%
\item{some specifications are better suited to DLTL or state machines:
%
some examples in \dfm's test suite included using side effects in the temporal monitor's matching to encode state transitions.
To more adequately handle this idiom without sophisticated handling of side-effects, we may wish to mix and match ways in which we express properties.
%
For example, a contract in a different language is considered primitive, and only continues when the contract is accepting (much like sequence contracts work now).
}
%
\item{whole program analysis is intractable at scale:
%
we can verify on a per-module basis if we add temporal contracts to a semantics of partial programs, such as in \citet{dvanhorn:TobinHochstadt2012Higherorder}.
%
The primary difficulty there is a sound definition of {\tt havoc} in the presence of temporal contracts, which simulates all possible interactions with a module.
%
Additional concerns lie in adequately maintaining enough information about unknown values to prune search space, without also exploding the search space with all the distinguished unknown values, but there is existing work on this that we can leverage~\citep{ianjohnson:DBLP:journals/cacm/DilligDA10}.
%
Within the confines of small programs, the implementation for derivatives is a simple translation of \autoref{fig:tcon-deriv} and \autoref{fig:matchsem} with some algebraic reductions, and is an easy target for more optimization.
}
\end{itemize}
%
%
