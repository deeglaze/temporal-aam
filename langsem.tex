Now that we have a clear view of the behavior of temporal contracts, we nail down a formal semantics that we use to prove the correctness of our monitoring system.
%
We present a reduction semantics~\citep{ianjohnson:Felleisen:2009:SEP:1795772}, which can be systematically transformed into an abstract machine in the form presented in~\citet{dvanhorn:VanHorn2010Abstracting}.
%%

The language itself is standard ISWIM, just extended with contract-monitoring forms, whose syntax and semantics we will focus on.
%
%%
\subsection{Additional syntactic forms}
%\FloatBarrier

We extend the language with additional forms to communicate actions with the temporal monitors, which can be naturally extended to allow $n$-ary functions such as in our initial example.
%
\begin{figure}
\begin{align*}
\mexp \in \Expr &::=
      \snlam{\mtoplevelname}{\mvar}{\mexp}
% These are already in overview
% \alt \sSMon{k}{l}{k}{\mtimeline}{\mscon}{\mexp}
% \alt \sTMon{k}{l}{j}{\mscon}{\mtcon}{\mexp}
 \alt \stblame{k}{j}
 \alt \scev{k}{j}{\mtimeline}{\mval_f}{\mexp}
 \alt \text{other forms}
\end{align*}
\caption{Syntax additions}
\label{fig:syntax}
\end{figure}

%
\iflong{At this point we would like to note that our usage of labels differs slightly from existing literature \cite{ianjohnson:dthf:complete}: we have a notion of the ``top level label'', denoted by $\toplevel$; this is similar to \citeauthor{ianjohnson:dthf:complete}'s use of $l_0$.}
%%
%%
Expressions of the form $\sret{k}{j}{\mtimeline}{\mval}{\mexp}$ denote the production of a return action on the $\mtimeline$ timeline, blaming $k$ for failures, where $\mval$ is the function being returned from, and $\mexp$ has evaluated to the return value.
%
Call actions are similar.
%
The $\stblame{l}{j}$ expression denotes a temporal contract failure (of the contract $j$), which is a fatal error which blames $l$.
%%
%
After arguments to a call, or return values are wrapped, we get a blessing from the temporal monitor that the call or return is acceptable.
%
\newcommand*{\namefmt}[1]{\textit{\textsc{#1}}}
This is embodied in the $\namefmt{wrap}$ and $\namefmt{action}$ rules in \autoref{fig:reduction}.
%
Functions wrapped in arrow contracts are annotated with the name in the contract ($\snlam{\mtoplevelname}{\mvar}{\mexp}$) for matching purposes, but otherwise these functions have the same behavior as non-annotated functions.
%
%
\iflong{
Our evaluation contexts in \autoref{fig:ctx} follow straightforwardly and are ownership-sensitive \cite{ianjohnson:dthf:complete}, though we leave out discussion of this topic, as we feel it is orthogonal to our work.}
%
%

%%
\subsection{Reduction} \label{sec:reduction}

A machine state ($\mstate$) is composed of the expression ($\mexp$), and store of temporal contracts ($\mTMons : \Timeline \to \wp(\TContract)$).
%
\begin{figure*}
\newcommand*{\update}[3]{#1^\prime = #1[#2 \mapsto #1(#2)\mathrel{\sqcup}{#3}]}
\newcommand*{\name}[1]{&\text{[\namefmt{#1}]}}
\newcommand*{\where}{\text{where }}
\newcommand*{\cwhere}{\phantom{\where}}
\centering
$\begin{array}{@{}l @{\ }c@{\ } l@{} r}
\mstate := \chevron{\mctx{\mmlab}[\mathit{redex}], \mTMons}
&\machstep&
\mstate^\prime := \chevron{\mctx{\mmlab^\prime}[\mathit{reduct}], \mTMons^\prime}
\name{rule-name}

\iflong*{
\\ \hline % If true
\sif{ \sOwn{\strue}{\mmlab} }{\mexp_1}{\mexp_2}
&\machstep&
\mexp_1
\name{if-true}
%
\\ % If false
\sif{ \sOwn{\sfalse}{\mmlab} }{\mexp_1}{\mexp_2}
&\machstep&
\mexp_2
\name{if-false}
%
\\ % Apply
\sapp{ \sOwn{\slam{\mvar}{\mexp}}{\mmlab} }{ \sOwn{\mval}{\mmlab} }
&\machstep&
\sown{ \sapp{\{ \sown{\mval}{\mmlab}/\mvar \}}{\mexp} }{\mmlab}
\name{apply}

\\
}
% Temporal monitor
{\\ \hline}
\sTMon{k}{\mmlab}{j}{\mscon}{\mtcon}{\mval}
&\machstep&
\sSMon{k}{\mmlab}{j}{\mtimeline}{\mscon}{\mval}
\name{tmon} \\
&&\where \update{\mTMons}{\mtimeline}{\setof{\mtcon}} \\
&&\cwhere \mtimeline = \salloc{\mstate}

\\ % Arrow monitor
\sSMon{k}{l}{j}{\mtimeline}{\sarr{\mtoplevelname}{\mscon_D}{\mscon_R}}{\mval}
&\machstep&
\self \qquad \text{given } \mval \equiv \slam{\mvar}{\mexp}
\name{wrap} \\
&&\where \self = \snlam{\mtoplevelname}{\mvar}{\sret{k}{j}{\mtimeline}{\self}{\mexp_{\mathit{rng}}}} \\
&&\cwhere \mexp_{\mathit{rng}} = \sSMon{k}{l}{j}{\mtimeline}{\mscon_R}{
                               \sapp{\mval}{\mexp_{\mathit{call}}}} \\
&&\cwhere \mexp_{\mathit{call}} = \scall{l}{j}{\mtimeline}{\self}{\mexp_{\mathit{arg}}} \\
&&\cwhere \mexp_{\mathit{arg}} = \sSMon{l}{k}{j}{\mtimeline}{\mscon_D}{\mvar}

\\ % Arrow monitor fail
\ditto
&\machstep&
\sblame{k}{j} \qquad\text{given } \mval \nequiv \slam{\mvar}{\mexp}
\name{wrap-fail}

\iflong{
\\ % Flat monitor
\sSMon{k}{\mmlab}{j}{\mtimeline}{ \sown*{\sflat{\mexp}}{\mmlab^{\prime\prime}} }{ \sOwn{\mval}{\mmlab^{\prime\prime}} }
&\machstep&
\schk*{k}{j}{\mtimeline}{ \sapp{\mexp}{\mval} }{\mval}
\name{smon-flat}
%
\\ % Cons monitor
\sSMon{k}{\mmlab}{j}{\mtimeline}{ \sconsc{\mscon_A}{\mscon_D} }{ \sOwn{\vcons{\mval_A}{\mval_D}}{\mmlab} }
&\machstep&
\sown{\vcons{ \sSMon{k}{\mmlab}{j}{\mtimeline}{\mscon_A}{\mval_A} }{ \sSMon{k}{\mmlab}{j}{\mtimeline}{\mscon_D}{\mval_D} }}{\mmlab}
\name{smon-cons}
}

\iflong{
\\ % Check true
\schk{k}{j}{ \sOwn{\strue}{j} }{\mval}
&\machstep&
\mval
\name{chk-true}
%
\\ % Check false
\schk{k}{j}{ \sOwn{\sfalse}{j} }{\mval}
&\machstep&
\sblame{k}{j}
\name{chk-false}
}

\\ % Call/Return
\scev{k}{j}{\mtimeline}{\mval_f}{\sown{\mval}{o}}
&\machstep&
\sown{\mval}{k}
\name{action} \\
&&\where \mtcons_\mvaluation = \Delta(\bigcup\setbuild{\derive{\scevev{\mval_f}{\mval}}{\mtcon}}{\mtcon\in\mTMons(\mtimeline)}) \\
&&\cwhere \mTMons' = \mTMons[\mtimeline \mapsto \gamma_1?(\mtimeline) \to \mtcons, \mTMons(\mtimeline) \sqcup \mtcons] \\
&&\text{given } \mvaluation \neq \bot

\\ % Call/Return fail
\ditto
&\machstep&
\stblame{k}{j}
\quad\where \ditto, \text{given } \mvaluation \neq \must
\name{TCon-fail}

\\ \hline % Blame from structural monitor
\mctx{\mmlab}[\sblame{k}{j}]
&\machstep&
\sblame{k}{j}
\name{blame}

\\ % Blame from temporal monitor
\mctx{\mmlab}[\stblame{k}{j}]
&\machstep&
\stblame{k}{j}
\name{TCon-blame}

\end{array}$
\caption{Reduction rules}
\label{fig:reduction}
\end{figure*}

%
We give our reduction relation in \autoref{fig:reduction}.
%
In the interest of space, we only present and discuss the rules which we feel are most relevant to our discussion of temporal contracts.
%

%
The \namefmt{tmon} rule stores the given temporal contract inside the store ($\mTMons$), associated with an allocated timeline ($\mtimeline$) and joined with any pre-existing temporal contracts on that timeline (as abstract allocation cannot always produce fresh timelines).
%
The structural component of the contract ($\mscon$) is then used to rewrap the monitored value ($\mval$) inside of a structural monitor (${\tt smon}$) associated with the timeline.
%

%
%The \namefmt{wrap} rules handle the reduction of structural monitors.
%
\iflong{%
Flat contracts (in \namefmt{smon-flat}) on constants are easily translated into a check on the contract.
%
Cons-list contracts (in \namefmt{smon-cons}) are also simple given our evaluation contexts: they are translated into an owned $\mathtt{cons}$, with each component appropriately wrapped by its respective structural contract: $\mscon_A$ for the \texttt{car} and $\mscon_D$ for the \texttt{cdr}.
%
}
%
The $\namefmt{action}$ rule states that when an action is permissible by the temporal monitor, we reduce to the wrapped value.
%
An action is permissible if the temporal contracts on the timeline do not derive to the empty contract.
%
\iflong{Finally, note that the evaluation context's owner becomes the callee after this step (in line with the call).}
%%
%
%%
Since multiple temporal contracts can be on a timeline, and a timeline might be aliased in the abstract, we have two points to focus on in the $\namefmt{action}$ rule.
%
We use the following $\Delta$ function in order to combine the possible outcomes of derivation due to weak equality; any failure means we must conservatively blame:
\begin{center}
  \begin{tabular}{lr}
    $\begin{array}{rl}
      \Delta(\setof{\bot_\must}) &= \emptyset_\bot \\
      \Delta(\setof{\mtcon_\mvaluation\ldots}) &= \setof{\mtcon' \ldots}_{\delta(\mvaluation\ldots)} \\
      \text{where } \isset{\mtcon'} &= \isset{\mtcon} \setminus \setof{\bot_\must, \bot_\may}
    \end{array}$
    &
    $\begin{array}{rl}
      \delta(\setof{\mvaluation}) &= \mvaluation \\
      \delta(\setof{\mvaluation\ldots}) &= \may
    \end{array}$
  \end{tabular}
\end{center}
%
When a timeline corresponds to one concrete timeline (given by $\gamma_1?$, always true in the concrete), we can destructively update the timeline state.
%
If the timeline is aliased, then we abstractly cannot decide which contract belongs to which alias of the timeline (they can even be identical contracts on two different timelines), so we join with the derivation to simultaneously step and not step all contracts.
%
%
\iflong{
An expression's action trace is defined as the string of actions that would be passed to the temporal monitor, were we to leave in the ${\tt call}$ and ${\tt ret}$ redexes.
%
We restrict it to a particular timeline in the following co-recursive definition:
\begin{align*}
 \denote{\mexp}_\mtimeline &= \prefixes\left(\left\{
   \begin{array}{ll}
      \setbuild{\scevev{\mval_f}{\mval}\mtrace}
               {\mtrace \in \denote{\mctx{\mmlab^\prime}[\mval]}_\mtimeline}
       &\text{if } \mexp\equiv\mctx{\mmlab^\prime}[\scev{k}{j}{\mtimeline}{\mval_f}{\mval}] \\
      \bigcup\setbuild{\denote{\mexp'}}{\mexp \machstep \mexp'} &\text{otherwise}\end{array}\right.\right)
\end{align*}
%%
Since timelines get allocated freshly and temporal contracts get introduced in later computation, our monitor's correctness condition must take into account all possible monitor constructions along execution.
%
Therefore, we state correctness as a correspondence between blame and (full) adherence to contracts after any amount of computation has produced some amount of temporal monitors.
%
The well-formedness criterion ($\wf$) requires emptiness to mean blame, and forces sane initial conditions; if an expression adheres to a temporal contract, it can never lead to an empty monitor by definition.
\begin{align*}
 \wf(\chevron{\mexp, \mTMons}) &= \forall \mtimeline\in\dom(\mTMons). \mTMons(\mtimeline) = \emptyset \implies \mexp \equiv \mctx{\mmlab^\prime}[\stblame{k}{j}] \\
 \adhere(\chevron{\mexp, \mTMons}) &= \forall \mtimeline\in\dom(\mTMons). \denote{\mexp}_\mtimeline \subseteq \bigcup\limits_{\mtcon \in \mTMons(\mtimeline)}{\denotetcon{\mtcon}} \\
 \good(\mtimeline, \mstate) &= \forall \mstate'\equiv\chevron{E[\scev{k}{j}{\mtimeline}{\mval_f}{\mval}], \mTMons}. \mstate \machstep^* \mstate' \implies \\
                            &\qquad \forall \mTMons'. \mstate' \centernot\machstep \chevron{E[\stblame{k}{j}], \mTMons'} \\
 \alias(\mtimeline, \mstate) &= \exists \mstate'\equiv\chevron{E[\sTMon{k}{l}{j}{\mscon}{\mtcon}{\mval}], \mTMons}. \\
                           &\qquad\mstate \machstep^* \mstate' \machstep \chevron{E[\sSMon{k}{l}{j}{\mtimeline}{\mscon}{\mval}], \mTMons\sqcup[\mtimeline \mapsto \mTMons(\mtimeline) \sqcup \setof{\mtcon}]} \\
 \blame(\mtimeline, \mstate) &= \exists \mstate'\equiv\chevron{E[\scev{k}{j}{\mtimeline}{\mval_f}{\mval}], \mTMons}, \mTMons'. \\
                             &\qquad\mstate \machstep^* \mstate' \machstep \chevron{E[\stblame{k}{j}], \mTMons'} \\
\end{align*}
%
%%
\begin{lemma}[Well-formedness preserved]\label{lem:wf}
  If $\wf(\mstate)$ and $\mstate \machstep \mstate'$ then $\wf(\mstate')$.
\end{lemma}
\begin{proof}
  Cases on $\machstep$.
\end{proof}
%
\begin{lemma}[Adherence means aliasing or no blame]
  For all $\mstate\equiv\chevron{\mexp, \mTMons}$,
  if $\adhere(\mstate)$ then
  $\forall \mtimeline \in \dom(\mTMons). \alias(\mtimeline, \mstate) \vee \good(\mtimeline,\mstate)$.
\end{lemma}
%
\begin{lemma}[Later blame means eventual non-adherence]
  For all $\mstate\equiv\chevron{\mexp, \mTMons}$,
  if $\exists \mtimeline \in \dom(\mTMons). \blame(\mtimeline,\mstate)$ then
  $\exists \mstate'. \neg\adhere(\mstate') \wedge \mstate \machstep^* \mstate'$.
\end{lemma}
%
\begin{theorem}[Monitor soundness]
  If $\overline{\mstate} = \chevron{\mexp,\mTMons} \machstep^* \chevron{\mctx{\mmlab^\prime}[\stblame{k}{j}], \mTMons'}$, $\wf(\chevron{\mexp,\mTMons})$ and $\inv(\chevron{\mexp, \mTMons})$ then $\exists \mstate \in \overline{\mstate}.\neg\inv(\mstate)$
\end{theorem}
Proof by induction and \autoref{thm:top-partial}.
\iflong{
  \begin{proof}
    \begin{byCases}
      \case{\text{Base}}{Vacuously true by definition of $\inv$.}
      \case{\text{Induction step }\overline{\mstate} \equiv
        \chevron{\mexp, \mTMons} \machstep^* \chevron{\mexp'', \mTMons''} \machstep \chevron{\mexp', \mTMons'}}{
        By cases on the final step
         $\chevron{\mctx{\mmlab^\prime}[\mexp''], \mTMons''} \machstep
          \chevron{\mctx{\mmlab^{\prime\prime}}[\mexp'], \mTMons'}$:
        \begin{byCases}
          \case{\scev{k}{j}{\mtimeline}{\mval_f}{\sown{\mval}{o}} \machstep \stblame{k}{j}}{
            Absurd by definition of $\inv$ and \autoref{thm:top-partial}.}
          \otherwise{No blame}
        \end{byCases}
      }
    \end{byCases}
  \end{proof}}
%
\begin{theorem}[Monitor completeness]
  For all $\mstate\equiv\chevron{\mexp,\mTMons}$,
  if $\neg\inv(\mstate)$ and $\wf(\mstate)$. then $\exists k,j, \mTMons'. \mstate \machstep^* \chevron{\mctx{\mmlab^\prime}[\stblame{k}{j}], \mTMons'}$.
\end{theorem}
\iflong{
  \begin{proof}
    Trivial in the case that $\exists \mtimeline. \mTMons(\mtimeline) = \emptyset$.
    By definition of $\inv$, there must be some timeline $\mtimeline$, temporal contract $\mtcon$, and action trace $\mtrace$ such that $\mtrace \notin \denotetcon{\mTMons(\mtimeline)}$
  \end{proof}
}
%
Notice that the invariant about adherence is not limited by the number of steps it is allowed to take.
%
Since the expression throughout reduction adheres to each contract on each created timeline, after no amount of steps will the monitor be able to blame.
}