Now that we have a clear view of the behavior of temporal contracts, we nail down a formal semantics that we use to prove the correctness of our monitoring system.
%
The semantics we present is in the style of Felleisen's reduction semantics~\citep{ianjohnson:Felleisen:2009:SEP:1795772}, which can be systematically transformed into an abstract machine in the form presented in~\citet{dvanhorn:VanHorn2010Abstracting}.
%%

%FIXME: This is from the LTL days...
%%
Wrapping monitors around values associates entirely fresh NFA states.
%
The names given to bindings are really associated with the bindings.
%
We have to give up some precision for decidability --- we use a monovariant allocation scheme for monitor allocation.
%
This scheme allows us to lift all textual temporal contracts to the top level and check individually.
%
A polyvariant scheme would require either knowing how many abstract bindings are allocated during execution, or an exhaustive enumeration of possible bindings as propositions to check.

%%
\subsection{Syntax}
%\FloatBarrier

\begin{figure}
\begin{align*}
\mexp \in \Expr &::=
      \slit{\mconstant}
 \alt \svar{\mvar}
 \alt \sapp{\mexp}{\mexp}
 \alt \slam{\mvar}{\mexp}
 \alt \sif{\mexp}{\mexp}{\mexp} \\
&\alt \sSMon{\mmlab}{\mmlab}{\mlab}{\mtimeline}{\mscon}{\mexp}
 \alt \sTMon{\mmlab}{\mmlab}{\mlab}{\mscon}{\mtcon}{\mexp} \\
\mconstant \in \Constant &=
 -1 \alt 0 \alt 1 \alt \ldots \alt + \alt - \alt \ldots \\
&\alt \strue \alt \sfalse \alt \vee \alt \wedge \alt \ldots \\
&\alt \sunit \alt \scons \alt \scar \alt \scdr \alt \ldots \\
\mvar \in \Var &\quad\text{ an infinite set} \\
\mmlab \in \Label \alt \toplevel &\quad\text{ an infinite set, where $\toplevel$ is the top-level}
\end{align*}
\caption{Syntax}
\label{fig:syntax}
\end{figure}

%
\autoref{fig:syntax} presents the core, desugared syntax of our language model.
%
Note that the surface syntax is better represented by the pseudocode examples given in \cref{sec:sort,sec:file,sec:tcp}; for example, a user would not be able to use $\mathtt{smon}$ or $\mathtt{tmon}$ in expressions.
%
While the syntax is largely conventional, the nature of monitoring and labelling are worth discussion.

%
A structural monitor $\sSMon{k}{\ell}{j}{\mtimeline}{\mscon}{\mexp}$ is like a traditional monitor \cite{ianjohnson:dthf:complete}: the structural contract is given by $\mscon$ and the contract parties are given by labels: $k$ is the server, $l$ is the client, and $j$ is the contract itself.
%
However, our structural monitors additionally feature a timeline ($\mtimeline$), used to temporally reason over values as they flow through contract boundaries (cf. \autoref{sec:temporal-semantics}).
%

%
A temporal monitor $\sTMon{k}{\ell}{j}{\mscon}{\mtcon}{\mexp}$ has a structural component as well ($\mscon$), but importantly it also considers a temporal contract given by $\mtcon$.
%
These kinds of monitors are only introduced at the top level, and thus need not be qualified over a timeline, as it is effectively $\toplevel$.
%
At this point we would like to note that our usage of labels differs slightly from existing literature \cite{ianjohnson:dthf:complete}: we have a notion of the ``top level label'', denoted by $\toplevel$; this is similar to \citeauthor{ianjohnson:dthf:complete}'s use of $\ell_0$.
%%

%%
\subsection{Value Space and Evaluation Contexts}

\begin{figure}
\begin{align*}
\mexp &::= \ldots
 \alt \sblame{\mmlab}{\mmlab} \\
&\alt \schk*{\mmlab}{\mmlab}{\mtimeline}{\mexp}{\mval}
 \alt \sret{\mmlab}{\mtimeline}{\mexp}
\\
\mval \in \Value &::=
      \mconstant
 \alt \vcons{\mval}{\mval}
 \alt \maddr \\
&\alt \bclos{\mmlab}{\mscon}{\mmlab}{\mmlab}{\mtimeline}{\mscon}{\mmlab}{\maddr}
\\
\maddr \in \Addr &\quad\text{an infinite set} \\
\mstimeline \in \Timeline &::= \maddr \alt \toplevel
\end{align*}
\caption{Value space}
\label{fig:values}
\end{figure}
%%

%
\autoref{fig:values} presents the value space for our language model.
%
For evaluation purposes, we extend our definition of expressions ($\mexp$) to include intermediate terms for contract checking.
%
Expressions of the form $\sret{\mmlab}{\mtimeline}{\mexp}$ denote the production of an return event $\sretev{\mmlab}{\mexp}$, which may be rejected by the temporal monitors in place.
%
The $\sblame{\ell}{j}$ expression denotes a contract failure (of the contract $j$), which is a fatal error which blames $\ell$.
%
An $\schk*{\ell}{j}{\mtimeline}{\mexp}{\mval}$ expression denotes an obligation to check the contract of $j$ given by $\mexp$ against the value $\mval$, blaming $\ell$ if the contract is violated.
%
The timeline component of this expression ($\mtimeline$) is used to pass along the timeline to any monitors inside of $\mexp$ which might need to reason over it; given complete monitoring \cite{ianjohnson:dthf:complete}, the contract itself is still subject and a party to the program's contracts.
%

%
Values ($\mval$) are conventional except for blessed closures ($\bclos{\mmlab}{\mscon}{\mmlab}{\mmlab}{\mtimeline}{\mscon}{\mmlab}{\maddr}$).
%
Though we've discussed timelines ($\mtimeline$) (cf. \autoref{sec:temporal-semantics}), we show here that our representation for them is either as an address ($\maddr$) or the top level ($\toplevel$).
%

%
Our evaluation contexts in \autoref{fig:ctx} follow straightforwardly and are ownership-sensitive \cite{ianjohnson:dthf:complete}, though we leave out discussion of this topic, as we feel it is orthogonal to our work.
%
A machine state ($\mstate$) is composed of the evaluation context ($\mctx{\mmlab}$), store ($\msto$), and store of PMSM states ($\mTMons$).
%
The former maps addresses to sets of values and the latter maps timelines to PMSMs.
%

%%
\begin{figure}
\begin{align*}
\mctx{\mmlab} \in \ECtx &::=
      \sapp{\mctx{\mmlab}}{\mexp}
 \alt \sapp{\mval}{\mctx{\mmlab}}
 \alt \sif{\mctx{\mmlab}}{\mexp}{\mexp} \\
&\alt \vcons{\mctx{\mmlab}}{\mexp}
 \alt \vcons{\mval}{\mctx{\mmlab}} \\
&\alt \sSMon{\ell}{k}{j}{\mtimeline}{\mscon}{\mctx*} \\
&\alt \sTMon{\ell}{k}{j}{\mscon}{\mtcon}{\mctx*} \\
&\alt \schk*{k}{\ell}{\mtimeline}{\mctx*}{\mval}
\alt \sret{\ell}{\mtimeline}{\mctx*}
\\
\mctx* \in \ECtx* &::=
      \hole
 \alt \sapp{\mctx*}{\mexp}
 \alt \sapp{\mval}{\mctx*}
 \alt \sif{\mctx*}{\mexp}{\mexp} \\
&\alt \vcons{\mctx*}{\mexp}
 \alt \vcons{\mval}{\mctx*} \\
&\alt \scall{\mmlab}{\mtimeline}{\mctx*}
 \alt \sret{\mmlab}{\mtimeline}{\mctx*} \\
\ECtx* &\subset \ECtx
\\
\mstate \in \State &= \chevron{\mctx{\mmlab}, \msto, \mTMons} \quad\text{machine state}
\\
\msto \in \Store &= \Addr \parto \wp(\Value)
\\
\mTMons \in \TMons &= \Timeline \parto \wp(\PMSM)
\end{align*}
\caption{Evaluation contexts}
\label{fig:ctx}
\end{figure}
%%

%%
\subsection{Reduction} \label{sec:reduction}

\newcommand*{\namefmt}[1]{\textit{\textsc{#1}}}
\begin{figure*}
\newcommand*{\update}[3]{#1^\prime = #1[#2 \mapsto #1(#2)\sqcup#3]}
\newcommand*{\name}[1]{&\text{[\namefmt{#1}]}}
\newcommand*{\where}{\text{where }}
\newcommand*{\cwhere}{\phantom{\where}}
\centering
$\begin{array}{@{}l @{\ }c@{\ } l@{} r}
\mstate \chevron{\mctx{\mmlab}[\ldots], \msto, \mTMons}
&\machstep&
\mstate^\prime \chevron{\mctx{\mmlab^\prime}[\ldots], \msto^\prime, \mTMons^\prime}
\name{rule-name}

\iflong*{
\\ \hline % If true
\sif{ \sOwn{\strue}{\mmlab} }{\mexp_1}{\mexp_2}
&\machstep&
\mexp_1
\name{if-true}
%
\\ % If false
\sif{ \sOwn{\sfalse}{\mmlab} }{\mexp_1}{\mexp_2}
&\machstep&
\mexp_2
\name{if-false}
%
\\ % Apply
\sapp{ \sOwn{\slam{\mvar}{\mexp}}{\mmlab} }{ \sOwn{\mval}{\mmlab} }
&\machstep&
\sown{ \sapp{\{ \sown{\mval}{\mmlab}/\mvar \}}{\mexp} }{\mmlab}
\name{apply}

\\ % Temporal monitor
}
{\\ \hline}
\sTMon{k}{\mmlab}{j}{\mscon}{\mtcon}{\mval}
&\machstep&
\sSMon{k}{\mmlab}{j}{\mtimeline}{\mscon}{\mval}
\name{tmon} \\
&&\where \update{\mTMons}{\mtimeline}{\smachine} \\
&&\cwhere \smachine = \compile{\mtcon}{\mtimeline} \ne \bot \\
&&\cwhere \mtimeline = \salloc{\mstate} \\
&&\text{given } \mmlab = \toplevel

\\ % Temporal monitor compilation fail
\ditto
&\machstep&
\sblame{j}{j}
\name{tmon-fail} \\
&&\where \bot = \compile{\mtcon}{\mtimeline} \\
&&\cwhere \mtimeline = \salloc{\mstate} \\
&&\text{given } \mmlab = \toplevel 

\iflong{
\\ % Flat monitor
\sSMon{k}{\mmlab}{j}{\mtimeline}{ \sown*{\sflat{\mexp}}{\mmlab^{\prime\prime}} }{ \sOwn{\mconstant}{\mmlab^{\prime\prime}} }
&\machstep&
\schk*{k}{j}{\mtimeline}{ \sapp{\mexp}{\mconstant} }{\mconstant}
\name{smon-flat}
%
\\ % Cons monitor
\sSMon{k}{\mmlab}{j}{\mtimeline}{ \sconsc{\mscon_A}{\mscon_D} }{ \sOwn{\vcons{\mval_A}{\mval_D}}{\mmlab} }
&\machstep&
\sown{\vcons{ \sSMon{k}{\mmlab}{j}{\mtimeline}{\mscon_A}{\mval_A} }{ \sSMon{k}{\mmlab}{j}{\mtimeline}{\mscon_D}{\mval_D} }}{\mmlab}
\name{smon-cons}
}

\\ % Arrow monitor
\sSMon{k}{\mmlab}{j}{\mtimeline}{ \sarr{\mmlab_\lambda}{\mscon_D}{\mscon_R} }{\mval}
&\machstep&
\sown{\bclos{\mmlab_\lambda}{\mscon_D}{k}{j}{\mtimeline}{\mscon_R}{\mmlab}{\maddr}}{\mmlab}
\name{smon-arrow} \\
&&\where \maddr = \salloc{\mstate} \\
&&\cwhere \update{\msto}{\maddr}{\mval} \\
&&\text{given } \mval = \slam{\mvar}{\mexp}

\\ % Arrow monitor fail
\ditto
&\machstep&
\sblame{k}{j} \qquad\text{given } \mval \neq \slam{\mvar}{\mexp}
\name{smon-arrow-fail}

\iflong{
\\ % Check true
\schk{k}{j}{ \sOwn{\strue}{j} }{\mval}
&\machstep&
\mval
\name{chk-true}
%
\\ % Check false
\schk{k}{j}{ \sOwn{\sfalse}{j} }{\mval}
&\machstep&
\sblame{k}{j}
\name{chk-false}
}

\\ % Blessed application (call)
\sapp{ \sOwn{\bclos{\mmlab_\lambda}{\mscon_D}{k}{j}{\mtimeline}{\mscon_R}{\mmlab^{\prime\prime}}{\maddr}}{\mmlab} }{ \sOwn{\mval}{\mmlab} }
&\machstep&
%TODO: Should the timeline be different?
\mexp_{ret}
\name{call} \\
&&\where \mexp_{ret} =
\sret{\mmlab_\lambda}{\mtimeline}{ \sSMon{k}{\mmlab^{\prime\prime}}{j}{\mtimeline}{\mscon_R}{\mexp_{call}} } \\
&&\cwhere \mexp_{call} = \sapp{\mval_\lambda}{\sSMon{\mmlab^{\prime\prime}}{k}{j}{\mtimeline}{\mscon_D}{\mval}} \\
&&\cwhere \mval_\lambda = \slam{\mvar}{\mexp} \in \msto(\maddr) \\
&&\cwhere \mTMons^\prime \in \sstep{\mTMons}{\mtimeline}{\scallev{\mmlab_\lambda}{\mval}}

\\ % Call fail
\ditto
&\machstep&
\sblame{\mmlab}{j}
\qquad\text{given } \bot \in \sstep{\mTMons}{\mtimeline}{\scallev{\mmlab_\lambda}{\mval}}
\name{call-fail}

\\ % Return
\sret{\mmlab_{ret}}{\mtimeline}{ \sOwn{\mval}{\mmlab} }
&\machstep&
\sown{\mval}{\mmlab_{ret}}
\qquad\where \mTMons^\prime \in \sstep{\mTMons}{\mtimeline}{\sretev{\mmlab}{\mval}}
\name{return}

\\ % Return fail
\ditto
&\machstep&
\sblame{\mmlab}{\j}
\qquad\text{given } \bot \in \sstep{\mTMons}{\mtimeline}{\sretev{\mmlab}{\mval}}
\name{return-fail}

\\ \hline % Blame
\mctx{\mmlab}[\sblame{k}{j}]
&\machstep&
\sblame{k}{j}
\name{blame}

\end{array}$
\caption{Reduction rules}
\label{fig:reduction}
\end{figure*}

%
We give our reduction relation in \autoref{fig:reduction}.
%
In the interest of space, we only present and discuss the rules which we feel are most relevant to our discussion of temporal contracts.
%
We elide the rules for the primitive functions as they are standard and straightforward.
%
Note, however, that all of our primitives (save for $\scons$) take only addresses as arguments and each rule dereferences the addresses in the store.
%

%
The \namefmt{tmon} rule compiles the temporal contract inside of the $\mathtt{tmon}$ ($\mtcon$) and stores it inside the PMSM store ($\mTMons$), associated with a fresh timeline ($\mtimeline$) and joined with any extant PMSMs associated with the timeline ($\mTMons(\mtimeline)$).
%
The structural component of the contract ($\mscon$) is then used to rewrap the monitored value ($\mval$) inside of a structural monitor ($\mathtt{smon}$) associated with the timeline.
%
Should the compilation of the temporal contract fail --- which would happen \eg if the PMSM is initially not accepting --- then the reduction would instead blame the contract in \namefmt{tmon-fail}.
%

%
The \namefmt{smon} rules handle the reduction of structural monitors.
%
\iflong{%
Flat contracts (in \namefmt{smon-flat}) on constants are easily translated into a check on the contract.
%
Cons-list contracts (in \namefmt{smon-cons}) are also simple given our evaluation contexts: they are translated into an owned $\mathtt{cons}$, with each component appropriately wrapped by its respective structural contract: $\mscon_A$ for the \texttt{car} and $\mscon_D$ for the \texttt{cdr}.
%
}
%
Arrow contracts (in \namefmt{smon-arrow}), however, are more nuanced.
%
As we our temporal contracts reason over the calls and returns of functions, we cannot simply translate this into \eg\ $\slam{\mvar}{ \sSMon{k}{\mmlab}{j}{\mtimeline}{\mscon_R}{ \sapp{\mval}{\sSMon{\mmlab}{k}{j}{\mtimeline}{\mscon_D}{\mvar} }}}$.
%
Instead, we construct a blessed closure with all of the relevant information, and store the monitored value in the store at a fresh address (\maddr) and joined with any extant values at that same address ($\msto(\maddr)$).
%
We will perform the relevant checking upon application of the blessed closure (in \namefmt{call}).
%
Nevertheless, if the monitored value is not a function, then we fail and blame in a straightforward manner (in \namefmt{smon-arrow-fail}).
%

%
Blessed application (in \namefmt{call}) is quite involved.
%
As we discussed regarding arrow contracts, we cannot simply check the structural domain and range.
%
Instead, we must first check with any relevant PMSMs (through the $\delta{}step$ function) whether the call event which represents this application is permissible; if not, then the relevant rule is instead \namefmt{call-fail}, which blames the caller.
%
Assuming that the call is temporally permissible, then we construct an application expression ($\mexp_{call}$) which wraps the argument in a structural domain monitor.
%
In turn, we wrap \emph{that} expression in a structural range monitor, which itself is wrapped in a return event production (which will be handled in \namefmt{return}/\namefmt{return-fail}).
%
This formulation guarantees that a call event is produced \emph{before} the call actually occurs, which is obviously critical to the correctness of these temporal contracts.
%
Finally, note that the evaluation context's owner becomes the callee after this step (in line with the call).
%

%
To complete our modeling of temporal monitors, we must also represent return events; we do so in \namefmt{return}.
%
Once we have reduced an application to a value, we must note that the function which was called will now return.
%
Again, we check with any relevant PMSMs---using the $\delta{}step$ function---whether the return event is permissible; if not, then the relevant rule is instead \namefmt{return-fail}, which blames the callee.
%
Otherwise, we reduce to the return value.
%
This formulation guarantees that a return event is produced \emph{before} the return actually completes, again critical for correctness.
%
After this reduction step, the evaluation context's owner becomes that of the original caller again (in line with the return).
%