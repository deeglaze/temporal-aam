%%
An important aspect of contract systems is that we don't have full knowledge of the program's future.
%
We don't have a full list of values that will be constructed, nor is it necessary to have this full list.
%
Thus, we separate out the control states---governing the progression through satisfying a temporal contract---from the bindings of names to values we encounter along the way.
%
Values will come from event constructors, and are allowed to be unbounded.
%
Unlike a conventional FSM, instead of transitioning by matching against atomic symbols, we allow for pattern matching against constructed data; this allows for the binding of subdata to variables for later equality checking.
%
We also scope bindings by simply $\alpha$-converting contracts to always use unique binders.
%

To do this, we consider a machine state to be a tuple of a control state and a binding environment, where transitions may change/use the environment when pattern matching on input.
%
Additionally, since we plan to later abstract our semantics (and values), we need a weaker, tunable notion of equality (\eg, is 1 equal to $\mathbf{Int}$? We must be conservative and consider both valid).
%
The atomic symbols therefore come equipped with a may/must/never equality-checking function.
%
A PMSM $\pmsmMachine$ is thus a tuple $(q_0, \simeq, \ary, \delta)$ where:
\begin{align*}
  \Sigma &\text{ a set of atoms} \\
  \simeq : \Sigma \times \Sigma \to \setof{\top, \bot, ?} &\text{ weak equality function}\\
  Q &\text{ a finite set of control states} \\
  C &\text{ a finite set of constructors} \\
  q_0 \in Q &\text{ initial state} \\
  \ary : C \to \Nat&\text{ the arity of constructors} \\
  \delta \subseteq Q \times \Pattern \times Q & \text{ transition relation}\\[2pt]
\end{align*}

\begin{property}
  For all $\matom, \matom' \in \Sigma$. $a \simeq a' = \top$ implies $a = a'$ and $a \simeq a' = \bot$ implies $a \neq a'$.
\end{property}
%
We consider all reachable states as accepting, except for the $\stfail$ state.
%
The pattern language consists of binding patterns, equality with reference patterns, wildcard and equality with a constant atom.
%
The syntax and semantics of matching patterns is in \autoref{fig:pattern-syntax}; notice that we do not change the binding environment for anti-matching ($!$) patterns.
%
Crucially, since we will need to form intersections between machines, we need a way to state that different variables will be bound to the same value, so each position in a constructor is allowed a \emph{set} of patterns to match.
%
Additionally, $\combinefalone$ on $\mbindenv$s is a right-biased finite function combination; we don't allow non-linear patterns (though it would be an easy extension) so the bias doesn't matter.
%
We lift $\simeq$ over constructors ($\simeq^\mconstructor$) by folding $\wedge$ over the data to compare atoms in the leaves.
%
\begin{figure}
  \begin{align*}
    \mpat \in \Pattern &::= \mconstructor(\mpat, \ldots_{\ary(\mconstructor)}) \alt \snonevent{\mconstructor(\mpat, \ldots_{\ary(\mconstructor)})} \\
     & \alt \swc \alt \sref{\mvar} \alt \sbind{\mvar} \alt \matom \alt \mpatset
      \quad\text{ where } \matom \in \Sigma
\\
    \mpatset \in \wp(\Pattern)
\\
    \mdata \in \Qualified &::= \mconstructor(\mdata, \ldots_{\ary(\mconstructor)}) \alt \matom
\\
    \mbindenv \in \BindEnv &= \Var \to \Qualified
\\
    t \in \Valuation &::= \top \alt\ ?
\\
    \MatchResult &::= \mfail \alt \smatch{\mbindenv}{t}
\\
    \match &: \Pattern \times \Qualified \times \BindEnv \to \MatchResult
\\
    \match(\swc, \mdata, \mbindenv) &= \smust{\mbindenv}
\\
    \match(\mpatset, \mdata, \mbindenv) &= \bigsqcup\limits_{\mpat \in \mpatset}{\match(\mpat, \mdata, \mbindenv)}
\\
    \match(\mconstructor(\mpat \ldots), \mconstructor(\mdata \ldots), \mbindenv) &=
      \bigsqcup\limits_i{\match(\mpat_i, \mdata_i, \mbindenv)}
\\
    \match(\snonevent{\mconstructor(\mpat \ldots)}, \mdata, \mbindenv) &=
      \left\{\begin{array}{ll}
               \smatch{\mbindenv}{\top} & \text{if } \mfail = \mres \\
               \smatch{\mbindenv}{?} & \text{if } \smatch{\mbindenv'}{?} = \mres \\
               \mfail & \text{if } \smatch{\mbindenv'}{\top} = \mres \\
             \end{array}\right.
    \\ \text{where }& \mres = \match(\mconstructor(\mpatset \ldots), \mdata, \mbindenv)
\\
    \match(\swc, \mdata, \mbindenv) &= \smatch{\mbindenv}{\top}
\\
    \match(\sref{\mvar}, \mdata, \mbindenv) &= \weakif(\mbindenv, \mdata \simeq^\mconstructor \mbindenv(\mvar))
\\
    \match(\sbind{\mvar}, \mdata, \mbindenv) &= \smatch{\mbindenv[\mvar \mapsto \mdata]}{\top}
\\
    \match(\matom, \matom', \mbindenv) &= \weakif(\mbindenv, \matom \simeq \matom')
\\
    \match(\mpat, \mdata, \mbindenv) &= \mfail \text{ otherwise}
\\[2pt]
    \weakif(\mbindenv, \bot) &= \mfail
\\
    \weakif(\mbindenv, t) &= \smatch{\mbindenv}{t} \text{ otherwise}
\\[2pt]
    \mfail \sqcup \mbindenv &= \mbindenv \sqcup \mfail = \mfail
\\
    \smatch{\mbindenv}{t} \sqcup \smatch{\mbindenv'}{t'} &= \smatch{\combinef{\mbindenv}{\mbindenv'}}{t \vee t'}
  \end{align*}
  \caption{Pattern syntax and semantics}
  \label{fig:pattern-syntax}
\end{figure}

%TODO: Do we need to define input-driven?
The machine is \emph{input-driven}, and a path through the machine is defined in \autoref{fig:pmsm-semantics}.
%
We will associate each temporal contract monitor with a state machine we generate from the temporal contract, where the data fed to the machine are call and return events that pass through the structural monitors associated with the timeline.
%
\begin{figure}
  \begin{align*}
    \mstate \in \State &::= \pmsmstate{\pmsmq, \mbindenv, t} \alt \stfail
  \end{align*}
  \begin{mathpar}
    \infer{(\pmsmq, \mpat, \pmsmq') \in \delta \\
             \smatch{\mbindenv'}{t'} = \match(\mpat, \mdata, \mbindenv)}
          {\pmsmstate{\pmsmq, \mbindenv, t} \pmsmstepd{\mdata}{\delta} \pmsmstate{\pmsmq', \mbindenv', t'}}
\\
    \infer{(\pmsmq, \mpat, \pmsmq') \in \delta \\
             \mfail = \match(\mpat, \mdata, \mbindenv)}
          {\pmsmstate{\pmsmq, \mbindenv, t} \pmsmstepd{\mdata}{\delta} \stfail}
\\
  \inferrule{ }{\mstate \multistepd{\epsilon}{\delta} \mstate}
\qquad
  \inferrule{\mstate \multistepd{\mtrace}{\delta} \mstate' \\
             \mstate' \pmsmstepd{\mdata}{\delta} \mstate''}
            {\mstate \multistepd{\mtrace\mdata}{\delta} \mstate''}
  \end{mathpar}
  \caption{PMSM semantics}
  \label{fig:pmsm-semantics}
\end{figure}

\subsection{Compiling temporal contracts to PMSMs}
%
The compilation process of a temporal contract to a PMSM is very similar to that of a regular expression to a NFA.
%
Negation is still cumbersome to deal with, so we cope with them by converting temporal contracts into a negative normal form before we start the translation.
%
\begin{figure}
  \begin{align*}
    \nnf(\stnot{(\stseq{\mtcon_0}{\mtcon_1})}) &=
      \stOr{\nnf(\stnot{\mtcon_0})}{\stseq{\nnf(\mtcon_0)}{\nnf(\stnot{\mtcon_1})}}
% XXX Is this the right def?
%      \stOr{(\stseq{\nnf(\stnot{\mtcon_0})}{\sddd})}{\stseq{\nnf(\mtcon_0)}{\nnf(\stnot{\mtcon_1})}}
\\
    \nnf(\stnot{\stnot{\mtcon}}) &= \stseq{\nnf(\mtcon)}{\sddd}
\\
    \nnf(\stnot{(\stOr{\mtcon_0}{\mtcon_1})}) &= \stAnd{\nnf(\stnot{\mtcon_0})}{\nnf(\stnot{\mtcon_1})}
\\
    \nnf(\stnot{(\stAnd{\mtcon_0}{\mtcon_1})}) &= \stOr{\nnf(\stnot{\mtcon_0})}{\nnf(\stnot{\mtcon_1})}
\\
    \nnf(\stnot{\mevent}) &= \stseq{\snonevent{\mevent}}{\sddd}
\\
    \nnf(\stnot{\snonevent{\mevent}}) &= \stseq{\mevent}{\sddd}
\\
    \nnf(\stnot{\stmany{\mtcon}}) &= \nnf(\stnot{\mtcon})
\\
    \nnf(\stnot{\stcall{\mname}{\mvar}{\mtcon}}) &= \stOr{(\stseq{\snonevent{\scallev{\mname}{\swc}}}{\sddd})}
                                                         {\stcall{\mname}{\mvar}{\nnf(\stnot{\mtcon})}}
\\
    \nnf(\stnot{\stret{\mname}{\mvar}{\mtcon}}) &= \stOr{(\stseq{\snonevent{\sretev{\mname}{\swc}}}{\sddd})}
                                                        {\stret{\mname}{\mvar}{\nnf(\stnot{\mtcon})}}
\\
    \nnf(\stnot{\sddd}) &= \stfail
\\
    \nnf(\stnot{\stfail}) &= \sddd
  \end{align*}
  Otherwise structurally distribute $\nnf$
  \caption{Temporal contract NNF translation}
  \label{fig:nnf}
\end{figure}
%%
\begin{theorem}
  $\denotetcon{\mtcon}{\mtimeline}{\menv} = \denotetcon{\nnf(\mtcon)}{\mtimeline}{\menv}$
\end{theorem}
\ifproof{
\begin{proof}
  By induction on the $\nnf$'s recursion scheme.
  \begin{byCases}
    \case{\stnot{(\stseq{\mtcon_0}{\mtcon_1})}}{
      By IH, it suffices to show
$\forall \mtcon_0 \mtcon_1 \mtimeline \menv. \denotetcon{\stnot{(\stseq{\mtcon_0}{\mtcon_1})}}{\mtimeline}{\menv} = \denotetcon{\stOr{\stnot{\mtcon_0}}{\stseq{\mtcon_0}{\stnot{\mtcon_1}}}}{\mtimeline}{\menv}$
($\Rightarrow$) Let $\mtrace \in \denotetcon{\stnot{(\stseq{\mtcon_0}{\mtcon_1})}}{\mtimeline}{\menv}$ be arbitrary.
By cases on $\mtrace$ prefixed by some $\mtrace_p \in \denotetcon{\mtcon_0}{\mtimeline}{\menv}$:
\begin{byCases}
  \case{\exists \mtrace_r. \mtrace = \mtrace_p \cdot \mtrace_r}{
   By cases on $\mtrace_r$ prefixed by a trace in $\denotetcon{\mtcon_1}{\mtimeline}{\menv}$:
   \begin{byCases}
     \case{\exists \mtrace' \in \denotetcon{\mtcon_1}{\mtimeline}{\menv}, \mtrace''. \mtrace_r = \mtrace'\cdot\mtrace''}{
       Contradiction, since $\mtrace_p\cdot\mtrace' \in \denotetcon{\stseq{\mtcon_0}{\mtcon_1}}{\mtimeline}{\menv}$ is a prefix of $\mtrace$}
     \case{\forall \mtrace' \in \denotetcon{\mtcon_1}{\mtimeline}{\menv}, \mtrace''. \mtrace_r \neq \mtrace'\cdot\mtrace''}{
       It follows that $\mtrace_r \in \denotetcon{\stnot{\mtcon_1}}{\mtimeline}{\menv}$ and thus $\mtrace \in \denotetcon{\stseq{\mtcon_0}{\stnot{\mtcon_1}}}{\mtimeline}{\menv}$.
     }\end{byCases}
   }
  \case{\forall \mtrace_p \in \denotetcon{\mtcon_0}{\mtimeline}{\menv}, \mtrace_r. \mtrace \neq \mtrace_p \cdot \mtrace_r}{This is another way to state $\mtrace \in \denotetcon{\stnot{\mtcon_0}}{\mtimeline}{\menv}$.}
\end{byCases}
($\Leftarrow$) Let $\mtrace \in \denotetcon{\stOr{\stnot{\mtcon_0}}{\stseq{\mtcon_0}{\stnot{\mtcon_1}}}}{\mtimeline}{\menv}$ be arbitrary.
    Let $\mtrace' \in \denotetcon{\stseq{\mtcon_0}{\mtcon_1}}{\mtimeline}{\menv}$ be arbitrary.
    If $\exists \mtrace_r. \mtrace = \mtrace' \cdot \mtrace_r$,
    then since $\exists \mtrace_p\in \denotetcon{\mtcon_0}{\mtimeline}{\menv}, \mtrace_q\in \denotetcon{\mtcon_1}{\mtimeline}{\menv} . \mtrace' = \mtrace_p \cdot \mtrace_q$,
    $\mtrace = \mtrace_p\cdot\mtrace_q\cdot\mtrace_r$, which is a contradiction.}
%
  \case{\stnot{\stnot{\mtcon}}}{
    By IH, it suffices to show $\forall \mtcon, \mtimeline,\menv. \denotetcon{\stnot{\stnot{\mtcon}}}{\mtimeline}{\menv} = \denotetcon{\mtcon}{\mtimeline}{\menv}$.
\\
($\Rightarrow$) Let $\mtrace \in \denotetcon{\stnot{\stnot{\mtcon}}}{\mtimeline}{\menv}$ be arbitrary.
We show $\exists \mtrace_p \in \denotetcon{\mtcon}{\mtimeline}{\menv}, \mtrace_r. \mtrace = \mtrace_p \cdot \mtrace_r$.
Suppose not for contradiction.
Thus $\forall \mtrace_p \in \denotetcon{\mtcon}{\mtimeline}{\menv}, \mtrace_r. \mtrace \neq \mtrace_p \cdot \mtrace_r$.
This means, by definition, $\mtrace \in \denotetcon{\stnot{\mtcon}}{\mtimeline}{\menv}$.
However, by case assumption $\forall \mtrace_p \in \denotetcon{\stnot{\mtcon}}{\mtimeline}{\menv}, \mtrace_r. \mtrace \neq \mtrace_p \cdot \mtrace_r$, so with $\mtrace_p = \mtrace, \mtrace_r = \epsilon$ we derive a contradiction.
Therefore $\mtrace \in \denotetcon{\stseq{\mtcon}{\sddd}}{\mtimeline}{\menv}$.
\\
($\Leftarrow$) Let $\mtrace \in \denotetcon{\stseq{\mtcon}{\sddd}}{\mtimeline}{\menv}$ be arbitrary.
By definition, $\exists \mtrace_p\in\denotetcon{\mtcon}{\mtimeline}{\menv},\mtrace_r.\mtrace = \mtrace_p \cdot \mtrace_r$.
Let $\mtrace_n \in \denotetcon{\stnot{\mtcon}}{\mtimeline}{\menv}$ be arbitrary.
We show $\mtrace_n$ is not a prefix of $\mtrace$.
Suppose it were, for contradiction.
Then, $\exists \mtrace_q. \mtrace = \mtrace_n \cdot\mtrace_q$.
Since $\forall \mtrace_u \in \denotetcon{\mtrace}{\mtimeline}{\menv}$, $\mtrace_u$ is not a prefix of $\mtrace_n$
    }
%
    \case{\stnot{(\stOr{\mtcon_0}{\mtcon_1})}}{...}
%
    \case{\stnot{(\stAnd{\mtcon_0}{\mtcon_1})}}{...}
%
    \case{\stnot{\mevent}}{...}
%
    \case{\stnot{\snonevent{\mevent}}}{...}
%
    \case{\stnot{\stmany{\mtcon}}}{...}
%
    \case{\stnot{\stcall{\mname}{\mvar}{\mtcon}}}{...}
%
    \case{\stnot{\stret{\mname}{\mvar}{\mtcon}}}{...}
%
    \case{\stnot{\sddd}}{...}
%
    \case{\stnot{\stfail}}{...}
%
    \otherwise{Structural cases follow by direct application of the induction hypothesis.}
  \end{byCases}
\end{proof}
}
The translation easily follows if we temporarily allow $\epsilon$ transitions that we later remove (that is, the $\Pattern$ component of $\delta$ may also be an $\epsilon$).
%
To remove $\epsilon$ transitions, we elaborate the transition relation in the following way:
\begin{align*}
  \epsclose(\delta) &= \setbuild{(\pmsmq, \mpat, \pmsmq')}{(\pmsmq_0, \mpat, \pmsmq_1) \in \delta, \pmsmq \in \pmsmq_0^\epsilon, \pmsmq' \in \pmsmq_1^\epsilon}
\end{align*}
where $\pmsmq_0 \in \pmsmq_1^\epsilon$ is defined as the least fixed point of the following rules:
\begin{mathpar}
  \inferrule{ }{\pmsmq \in \pmsmq^\epsilon} \qquad
  \inferrule{(\pmsmq, \epsilon, \pmsmq') \in \delta}{\pmsmq' \in \pmsmq^\epsilon} \qquad
  \inferrule{\pmsmq_0 \in \pmsmq^\epsilon \\ \pmsmq_1 \in \pmsmq_0^\epsilon}{\pmsmq_1 \in \pmsmq^\epsilon}
\end{mathpar}

\begin{lemma}
  $\matchstate \multistepd{\mtrace}{\delta} \matchstate'$ iff
  $\matchstate \multistepd{\mtrace}{\epsclose(\delta)} \matchstate'$
\end{lemma}
\begin{proof}
  By induction on $\mtrace$.
\end{proof}
%
\begin{align*}
 s \in \QState &::= q_i \alt \pmsmpair{s}{s} \text{ where } i \in \Nat
\\
 \translateeps &: \TContract \times \Timeline \to \PMSM
\\
 \translateeps(\mtcon, \mtimeline) &= (\pmsmq_0, \lambda x,y . x = y, \ary, \epsclose(\delta)) \\
 \text{where }
   (\delta, \pmsmqlast, k) &= \epshelp^\mtimeline(\mtcon, \pmsmq_0, 1, \emptyset, \setof{\pmsmq_0})
\\
   \ary(\apvcallalone) &= 2 \\
   \ary(\apvretalone) &= 2 \\
   \ary(\bclos{\swc}{\swc}{\swc}{\swc}{\swc}{\swc}{\swc}{\swc}) &= 8 \\
\\
\begin{array}{l}
    \epshelp :  \Timeline \times \TContract \times \QState \times \Nat \\ \phantom{\epshelp :}\to \mathit{Transition} \times \wp(\QState) \times \QState \times \Nat
\end{array}\span\omit
\end{align*}

%FIXME: Pretty terrible
and:

\begin{align*}
\begin{array}{l}
  \epshelp^\mtimeline(\scallev{\mvar}{\mvpat'}, \pmsmqcur, k, \delta) = \\
\quad           (\delta
            \cup \setof{(\pmsmqcur,
                         \apvcall{\sref{\mvar}}{\vpatToPattern(\mvpat)},
                         \pmsmq_k)}, \\
\quad            \ \pmsmq_k, k+1)
\\[2pt]
%
  \epshelp^\mtimeline(\scallev{\mtoplevelname}{\mvpat'}, \pmsmqcur, k) = \\
\quad           (\setof{(\pmsmqcur,
                         \apvcall{\bclos{\mtoplevelname}{\swc}{\swc}{\swc}{\mtimeline}{\swc}{\swc}{\swc}}{\vpatToPattern(\mvpat)},
                         \pmsmq_k)}, \\
\quad            \ \setof{\pmsmqcur, \pmsmq_k}, \pmsmq_k, k+1)
\\
\begin{array}{ll}
  \vpatToPattern(\swc, \mtimeline) &= \swc
\\
  \vpatToPattern(\mvar, \mtimeline) &= \sref{\mvar}
\\
  \vpatToPattern(\mtoplevelname, \mtimeline) &= \bclos{\mtoplevelname}{\swc}{\swc}{\swc}{\mtimeline}{\swc}{\swc}{\swc} \\
  \vpatToPattern(\mconstant, \mtimeline) &= \mconstant
\end{array}
\end{array}
\end{align*}
Return events are similarly translated, and binding events are similar to their non-binding counterparts, only the created final state becomes the initial state for constructing the machine for the following contract.

The translation for ``or'' is similar to constructing the union of NFAs.
%
We simply thread through a source of freshness for constructing new states.
%
\begin{align*}
  \begin{array}{l}
  \epshelp^\mtimeline(\stOr{\mtcon_0}{\mtcon_1}, \pmsmqcur, k) = \\
    (\delta\cup \delta' \cup
    \left\{\begin{array}{ll}
        (\pmsmqlast^0, \epsilon, \pmsmq_{k''}),&
        (\pmsmqcur, \epsilon, \pmsmq_k),\\
        (\pmsmqlast^1, \epsilon, \pmsmq_{k''}),&
        (\pmsmqcur, \epsilon, \pmsmq_{k+1})
      \end{array}\right\},
    \\
    \ \pmsmq_{k''}, k''+ 1)
    \\
  %
    \text{where } (\delta, \pmsmqlast^0, k') = \epshelp^\mtimeline(\mtcon_0, \pmsmq_k, k+2) \\
    \phantom{\text{where }} (\delta', \pmsmqlast^1, k'') = \epshelp^\mtimeline(\mtcon_1, \pmsmq_{k+1}, k')
  \end{array}
\end{align*}

Kleene star is a simple pair of epsilon transitions:
\begin{align*}
  \begin{array}{l}
  \epshelp^\mtimeline(\stmany{\mtcon}, \pmsmqcur, k) = \\
  (\delta \cup \left\{
    \begin{array}{l}
      (\pmsmqcur, \epsilon, \pmsmqlast), \\
      (\pmsmqlast, \epsilon, \pmsmqcur)
    \end{array}\right\},
  \pmsmqlast, k')
 \\ \text{where } (\delta, \pmsmqlast, k') = \epshelp^\mtimeline(\mtcon, \pmsmqcur, k)
\end{array}
\end{align*}

Concatenation uses the first contract's final state as the initial state for the following contract's machine:
\begin{align*}
  \begin{array}{l}
    \epshelp^\mtimeline(\stseq{\mtcon_0}{\mtcon_1}, \pmsmqcur, k) = 
     (\delta \cup \delta', \pmsmqlast^1, k'')
   \\ \text{where } (\delta, \pmsmqlast^0, k') = \epshelp^\mtimeline(\mtcon_0, \pmsmqcur, k)
   \\\phantom{\text{where }} (\delta', \pmsmqlast^1, k'') = \epshelp^\mtimeline(\mtcon_1, \pmsmqlast^0, k')
  \end{array}
\end{align*}
%

Intersection has the most complex construction, so we saved it for last.
%
In order to avoid an exponential explosion of states that we must construct \emph{up-front}, we neither determinize the machines nor then perform the standard DFA intersection construction.
%
Instead, we simply remove $\epsilon$ transitions and construct the product automaton from that, since we will then make the two machines sync up in terms of input consumed across transitions.
%
Unlike NFAs, we do not have a finite alphabet to enumerate, so instead we combine patterns to check dynamically, which is where the sets of patterns come into play.
%
The final state of the construction is just the pair of each machine's final states.
\begin{align*}
  \begin{array}{l}
  \epshelp^\mtimeline(\stAnd{\mtcon_0}{\mtcon_1}, \pmsmqcur, k) = \\
  (\left\{(\pmsmpair{\pmsmqq_0}{\pmsmqq_1}, \mpat'', \pmsmpair{\pmsmqq'_0}{\pmsmqq'_1}) :
          \begin{array}{l}
                (\pmsmqq_0, \mpat, \pmsmqq'_0) \in \epsclose(\delta),\\
                (\pmsmqq_1, \mpat', \pmsmqq'_1) \in \epsclose(\delta'), \text{ and}\\
                \mpat'' = \combinepat{\mpat}{\mpat'}
           \end{array} \right\},\\
  \ \pmsmpair{\pmsmqlast^0}{\pmsmqlast^1}, k'') \\
  \text{where } (\delta, \pmsmqlast^0, k') = \epshelp^\mtimeline(\mtcon_0, \pmsmqcur, k) \\
  \phantom{\text{where }} (\delta', \pmsmqlast^1, k'') = \epshelp^\mtimeline(\mtcon_1, \pmsmqcur, k')
\end{array}
\end{align*}
%
Pattern combination ($\combinepatalone$) could just throw both patterns into a set together and wait to fail if there is no overlap.
%
However, we're a bit more proactive than that and rule out patterns that outright don't match, such as disequal constants or patterns with separate constructors.
%
The definition is a bit tedious and obvious, so we leave it informally specified for space.

\begin{theorem}[Soundness]
  If $\mtrace \in \denotetcon{\mtcon}{\mtimeline}{\menv}$ then $\exists \pmsmqq, \menv', t$.
%
  $\pmsmstate{\pmsmqcur, \menv, \top} \multistepd{\mtrace}{\delta} \pmsmstate{\pmsmqq, \menv', t}$ is a well-formed trace (the function positions of matched events are all on $\mtimeline$)
%
  where $(\delta, \pmsmqlast, k') = \epshelp^\mtimeline(\mtcon, \pmsmqcur, k)$
\end{theorem}
\begin{proof}
  By induction on $\mtcon$ and then $\mtrace$.
\end{proof}
%
\begin{theorem}[Concrete completeness]
  If $\exists \pmsmqq, \menv'$.
%
  $\pmsmstate{\pmsmqcur, \menv, \top} \multistepd{\mtrace}{\delta} \pmsmstate{\pmsmqq, \menv', \top}$ is a well-formed trace,
  then $\mtrace \in \denotetcon{\mtcon}{\mtimeline}{\menv}$.
  If $\pmsmstate{\pmsmqcur, \menv, \top} \multistepd{\mtrace}{\delta} \stfail$ is a well-formed trace, then $\mtrace \notin \denotetcon{\mtcon}{\mtimeline}{\menv}$
%
  where $(\delta, \pmsmqlast, k') = \epshelp^\mtimeline(\mtcon, \pmsmqcur, k)$
\end{theorem}
\begin{proof}
  By induction on $\mtcon$ and then $\mtrace$.
\end{proof}
%
When $\simeq$ never has uncertainty, such as in the concrete semantics, this machine accepts only traces that are in the denotation of the contract it was compiled from.
%
This theorem ensures us our monitoring system is correct (provided the semantics only provides well-formed traces, which it does).

The reduction semantics (\autoref{sec:reduction}) uses this monitor construction via the following metafunctions:
\begin{align*}
  \compile{\mtcon}{\mtimeline} &= \npmsm{\setof{\pmsmstate{\pmsmq_0, \bot, \top}}}{\translateeps(\mtcon, \mtimeline)} \\
  \sstep{\mTMons}{\mtimeline}{\mevent} &= T \cup F \\
    \text{where } {\mathcal S} &= \left\{
\ensuremath{
      \begin{array}{ll}
        \npmsm{S'}{\pmsmMachine} : &
        \begin{array}{l}
\npmsm{S}{\pmsmMachine} \in     \mTMons(\mtimeline) \\
 S' = \setbuild{\mstate'}{\mstate \in S, \mstate \pmsmstepd{\mevent}{\delta_\pmsmMachine} \mstate'}
\end{array}
      \end{array}
}\right\}
\\\phantom{\text{where }} T &= \setbuild{\mTMons[\mtimeline \mapsto {\mathcal S}]}{\npmsm{S}{\pmsmMachine} \in {\mathcal S}, S \neq \setof{\stfail}}
\\\phantom{\text{where }} F &= \left\{\stfail : \npmsm{S}{\pmsmMachine} \in {\mathcal S},
  \begin{array}{l}
S = \setof{\stfail} \text{ or } \\
\forall \mstate\equiv\pmsmstate{\pmsmqq, \mbindenv, t}\in S. t \neq \top
\end{array}\right\}
\end{align*}
We store several machines in one timeline in anticipation of abstract aliasing --- that is, we cannot always create fresh timelines, and thus merge two together by saying the timeline is in one of a set of possible states.
%
The additional set comprehension is because these machines are non-deterministic, so we must step all states in which that machine \emph{might} be.
%
If the only possible state in which the machine can be is $\stfail$, or there are no sure ways to accept (no state with $\top$ valuation), there is a cause for blame.
%
Since there are several machines that could describe the timeline at one time, we return a set of possibilities: either there is a definitive failure $\setof{\bot}$, definitive success $\setof{\mTMons'}$, or a possible success $\setof{\mTMons', \bot}$.
